\documentclass[12pt]{article}

\usepackage[margin=1in]{geometry}
%\usepackage{setspace}
%\doublespacing

\usepackage{fenna-files/packages}
\usepackage{fenna-files/commands}
\bibliography{fenna-files/references}{}

\title{The instrumental \tsc{r}-pronoun and postposition in Dutch}
\author{Fenna Bergsma\\Goethe-Universität Frankfurt}
\date{\today}

\begin{document}

\maketitle

\section{The problem}

Dutch has several types of inanimate pronouns.

\ex.
\ag. t\\
 it\\
 `it'
\bg. *dat\\
 that\\
 `that'
\bg. *wat\\
 what\\
 `what'
\bg. *iets\\
 something\\
 `something'
\bg. *alles\\
 everything\\
 `everything'

Dutch has a preposition for that expresses instrumental.

\exg. met een potlood\\
 with a pencil\\
 `with a pencil'

Inanimate pronouns in Dutch do not combine with the instrumental preposition.

\ex.
\ag. *met t\\
 with it\\
 `with it'
\bg. *met dat\\
 with that\\
 `with that'
\bg. *met wat\\
 with what\\
 `with what'
\bg. *met iets\\
 with something\\
 `with something'
\bg. *met alles\\
 with everything\\
 `with everything'

Instead, an \tsc{r}-pronoun plus postposition is used.

\ex.
\ag. (e)r mee\\
 there with\\
 `with it'
\bg. daar mee\\
 there with\\
 `with that'
\bg. waar mee\\
 where with\\
 `with what'
\bg. ergens mee\\
 somehwere with\\
 `with something'
\bg. overal mee\\
 everywhere with\\
 `with everything'


\section{This paper: questions and answers}

There is a big question to be asked:

\begin{itemize}
  \item Why *[P inanimate]?
\end{itemize}

And there are three questions with respect to the instrumental:

\begin{itemize}
  \item why is there an \tit{r}, even though there is no location involved
  \item why does the adposition change form?
  \item why does the adposition change position?
\end{itemize}


Answer to the big question:

\begin{itemize}
  \item the \tsc{r}-pronoun and postposition are the default spellout, as long as they form a proper constituent
\end{itemize}

Answers to the smaller questions:

\begin{itemize}
  \item the \tit{r} is ambiguous between location and (caseless) thing
  \item the preposition and postposition are two separate items in the lexcion so: (1) the form can be different, and (2) the position can be different
\end{itemize}


\section{Proof for proper constituent}

Let's compare a proper constituent vs. not a proper constituent

\ex.
\ag. Ze heeft (de brief) daar mee geschreven.\\
 she has the letter there with written\\
 `She wrote (the letter) with that.'
\bg. Ze heeft (de brief) met dat potlood geschreven\\
 she has the letter with that pencil written\\
 `She wrote (the letter) with that pencil.'

\tit{Daar mee} `there with' forms a proper constituent, \tit{met dat} `with that' does not.

\ex. topicalization
\a. Daar mee heeft ze (de brief) geschreven.
\b. *Met dat heeft ze (de brief) potlood geschreven.
\b. Met dat potlood heeft ze (de brief) geschreven.

\ex. coordination
\a. Ze heeft (de brief) hier mee en daar mee geschreven.
\b. *Ze heeft (de brief) met die en met dat potlood.
\b. Ze heeft (de brief) met die pen en met dat potlood geschreven.

\ex. fragment answer
\a. Waarmee heeft ze de brief geschreven?
\b. daarmee
\b. *met dat
\b. met dat potlood


\section{Technical implementation}

\ex. r = [loc [thing]]

\ex. met = binary entry

\ex. mee = unary entry


alternative fseq

\ex. [\tsc{f4} [\tsc{f3} [\tsc{f2} [\tsc{f1} [\tsc{thing} [ D [\tsc{deix}  ] ] ] ] ] ] ]


\subsection{Mobility of \tit{mee}}

\tit{daar} can move while stranding \tit{mee}. A consequence of letting \tit{mee} be a suffix makes it hard to only move \tit{daar}. Can \tit{d-} and \tit{a-} be lower in the hierarchy than \tit{r}? Then \tit{r} can be a suffix on \tit{da}. That way, \tit{d} and \tit{a} are not projecting as specifiers.

\subsection{Full DPs and animates}

Why do full DPs not take \tit{mee} as a suffix? Has to do with pronominal strength features and gaps. (?)


\section{Big-big question}

Should adpositions by default be able to be pre and post?

\subsection{pro}

lots are syncretic

\subsection{against}

several can only be pre or post

\ex. only pre: zonder, volgens

\ex. only post: vandaan

\subsection{two tables}

three instances of \tit{in}

\begin{table}[h]
  \begin{tabular}{ccc}
  \toprule
          & DP				& \tsc{r}	\\
  \midrule
    pre		& \tsc{loc} & *     	\\
    post	& \tsc{dir}	& \tsc{loc} \\
  \bottomrule
  \end{tabular}
\end{table}

twice \tit{mee} and once \tit{met}

\begin{table}[h]
  \begin{tabular}{ccc}
  \toprule
          & \tit{mee}	& \tit{met}	\\
  \midrule
    pre		& verb		 	& noun     	\\
    post	& noun			& *					 \\
  \bottomrule
  \end{tabular}
\end{table}






\printbibliography

\end{document}
