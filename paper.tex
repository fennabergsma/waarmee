\documentclass[12pt]{article}

\usepackage[margin=1in]{geometry}
%\usepackage{setspace}
%\doublespacing

\usepackage{fenna-files/packages}
\usepackage{fenna-files/commands}
\bibliography{fenna-files/references}{}

\title{The \tsc{r}-pronoun and postposition \tit{waarmee} in Dutch}
\author{Fenna Bergsma}
\date{\today}

\begin{document}

\maketitle



\section{Introduction}

Dutch has the preposition \tit{met} `with' that expresses instrumental case. In \ref{ex:metdp}, \tit{met} `with' is combined with a full DP. The inanimate pronoun in Dutch is \tit{'t} `it'.\footnote{The longer form that is mostly used in writing is \tit{het} `it'. I will use the spoken variant \tit{'t} throughout this paper.} \ref{ex:tverb} illustrates that \tit{'t} `it' can be used as the object of a verb.

\ex.
\ag. Ik schilder met een kwast.\\
 I paint with a brush\\
 `I am painting with a brush.'\label{ex:metdp}
 \bg. Ik zie 't.\\
  I see it\\
  `I see it.'\label{ex:tverb}

\tit{Met} `with' and \tit{'t} `it' do not appear together, as illustrated in \ref{ex:neemett}. Instead, Dutch uses the \tsc{r}-pronoun \tit{'r} and the postposition \tit{mee}, as shown in \ref{ex:jarmee}.\footnote{The \tsc{r}-pronoun \tit{'r} in \ref{ex:klimerop} can be written as \tit{er}, \tit{der} and \tit{'r} and pronounced as respectively /ɛr/, /dər/ or /ər/. As far as I am aware, there is no clear meaning difference between these forms. See \citet{wesseling2018} for discussion.
In my examples I use \tit{'r}, but the other two forms fit just as well.}$^,$\footnote{In this paper I do not make any claims about the distinction between prefixes and prepositions, or suffixes and postpositions.}

\ex.\label{ex:rmeeconst}
\ag. *Ik schilder met 't.\\
 I paint with it\\
 `I am painting with it.'\label{ex:neemett}
\bg. Ik schilder 'r -mee.\\
 I paint there -with\\
 `I am painting with it.'\label{ex:jarmee}

\tsc{R}-pronouns \citep{riemsdijk1978,koopman1994} are nominal elements that are syncretic with locative pronouns, which in Dutch means they contain the morpheme \tit{r}. The adposition it combines with obligatorily follows the \tsc{r}-pronoun, compare \ref{ex:jarmee} and \ref{ex:neemeer}.
Notice also that the preposition \tit{met} `with' has changed form into \tit{mee} `with' when it appears as a postposition (see \ref{ex:metdp} and \ref{ex:jarmee}).

\exg. *Ik schilder mee 'r.\\
 I paint with there\\
 `I am painting with it.'\label{ex:neemeer}

This chapter is concerned with the question of how to correctly rule out the construction in \ref{ex:neemett} and to let the \tsc{r}-pronoun construction in \ref{ex:jarmee} appear. I argue that \tsc{r}-pronouns are not something special, but a consequence of regular spellout. Just like \citet{riemsdijk1978} I analyze \tsc{r}-pronouns and postpositions as a type of allomorphy. What is different is that spellout rules out the ungrammaticality of the preposition and inanimate pronoun, rather than the stipulation of a filter.

This chapter is an in-depth study of the instrumental \tsc{r}-pronoun-postposition combination \tit{'r -mee} `with it' and \tit{waarmee} `with what' in Dutch. This instance is interesting for two reasons. First, just like for all \tsc{r}-pronouns, the \tsc{r}-pronoun is syncretic with the locative. In many of the \tsc{r}-pronoun-postpositions combinations, the meaning component of the locative is intuitive, as many prepositions express locations, directions etc. However, an instrumental expresses an instrument and does not have a meaning component assosicated with location.
The second reason why I focus on this particular \tsc{r}-pronoun and postposition has to do with the form of the adposition. The preposition \tit{met} `with' does not only turn into a postposition, but it also changes into \tit{-mee} `with' when it is combined with an \tsc{r}-pronoun.\footnote{I do not have anything to say about the distinction between prefixes and prepositions, or suffixes and postpositions.} This last observervation has so far remained unexplained.

%%here the basic idea%%
When all features form a proper constituent, then the \tsc{r}-pronoun surfaces. When there is a complement or whatever, it does not work.
In what follows I show that \tsc{r}-pronouns and regular preposition compete to spell out the same syntactic features. If all features form a proper constituent (i.e. a constituent to the exclusion of other features), the \tsc{r}-pronoun surfaces. If it is not a proper constituent, the preposition-pronoun combination shows up. This straightforwardly follows in a system in which spellout targets phrasal constituents: Nanosyntax \citep{starke2009}. --here more about the internal structure of r-pronouns, and the postposition preposition distinction--


%%here the structure of the paper%%
This paper is structured as follows. In Section \ref{sec:distribution} I discuss the distribution of \tit{waarmee} `waarmee' and \tit{met wat} `with what'. I show that the \tsc{r}-pronoun surfaces when it forms a proper constituent, and \tit{met wat} `with what' when it does not. I decompose \tit{waarmee} `with what' and \tit{met wat} `with what' to make sense of the phonological similarities, and I provide a more detailed analysis. Unmarked examples are constructed and have been verified by native speakers.





\section{\tsc{R}-pronouns are proper constituents}\label{sec:distribution}


\subsection{\tsc{R}-pronouns as default}\label{sec:rdefault}

The goal of this section is to show that \tit{waarmee} `with what' is the default as instrumental relative pronoun, a shown in a.o. \citet{riemsdijk1978,koopman2000}. In order to show that \tit{waarmee} `with what' is the default, I discuss the distribution of \tsc{r}-pronouns and regular pronouns in more general. I start with the personal pronouns and then return to the wh-pronouns.

Dutch has the personal pronouns \tit{haar} `her', \tit{hem} `him' and \tit{het} `it' that can be used as animate and inanimate objects of verbs, as illustrated in \ref{ex:objverb}.

 \ex. \label{ex:objverb}
 \ag. Ik zie haar/hem.\\
  I see her/him\\
  `I see her/him.'\label{ex:aniobj}
 \bg. Ik zie 't.\\
  I see it\\
  `I see it.'\label{ex:inaniobj}

The example in \ref{ex:prepani} shows that for animate objects the same pronouns (\tit{haar} `her' and \tit{hem} `him') appear as objects of prepositions. However, the inanimate personal pronoun \tit{het} `it' cannot be used as an object of a preposition, shown in \ref{ex:prephet}. Instead, an ʀ-pronoun appears. This is illustrated in \ref{ex:preper}. \ref{ex:erprep} shows that the \tsc{r}-pronoun obligatorily moves to the left of the pronoun.

\ex. \label{ex:objprep}
\ag. Ik schilder samen met haar/hem.\\
 I paint together with her/him\\
 `I am painting together with her/him.'\label{ex:prepani}
\bg. *Ik schilder met 't.\\
 I paint with it\\
 `I am painting with it.'\label{ex:prephet}
\bg. Ik schilder 'r mee.\\
 I paint there -with\\
 `I am painting with it.'\label{ex:preper}
\bg. *Ik schilder mee 'r.\\
 I paint with there\\
 `I am painting with it.'\label{ex:erprep}

\tit{Met} is not the only preposition with which this happens. \tit{Op} `on' and \tit{in} `in' do not combine with the inanimate personal pronoun \tit{'t}, but the \tsc{r}-pronoun is used obligatorily.

\ex.
\ag. Ik zit 'r op.\\
 I sit there on\\
 `I am sitting on it.
\bg. *Ik zit op 't.\\
 I sit on it\\
 `I am sitting on it.

\ex.
 \ag. Hij zwemt 'r in.\\
  he swims it-in\\
  `He is swimming in it.'
 \bg. *Hij zwemt in 't.\\
  he swims in it\\
  `He is swimming in it.'

The situation of the inanimate wh-pronouns resembles the inanimate personal pronouns. \tit{Wat} `what' can function as an object of a verb (see \ref{ex:wat}), but not as an object of a preposition \ref{ex:metwat}. In that case, the \tsc{r}-pronoun \tit{waarmee} `with what' is appears, as shown in \ref{ex:waarmee}.\footnote{The sentence in \ref{ex:metwat} is unacceptable with neutral intonation. It becomes is only acceptable if \tit{wat} `what' is stressed, for example in a context in which the speaker is highly surprised about the choice for the object hearer is painting with.}

\ex.
\ag. Wat zie jij?\\
 what see you\\
 `What do you see?'\label{ex:wat}
\bg. *Met wat schilder jij?\\
 with what paint you\\
 `What are you painting with?'\label{ex:metwat}
\bg. Waarmee schilder jij?\\
 {where with} paint you with\\
 `What are you painting with?'\label{ex:waarmee}

\tit{Waarmee} `with what' and not \tit{met wat} `with what' does not only appear in wh-questions, but also in other contexts. \ref{ex:headed} gives an example of a headed relative, and \ref{ex:headless} shows a free relative in which both predicates combine with an instrumental object. The use of \tit{met wat} `with what' is ungrammatical in both contexts, and \tit{waarmee} `with what' is used.

\ex.\label{ex:headed}
\ag. Ik schilder met de kwast waarmee jij ook schildert.\\
 I paint with the brush {where with} you also paint\\
 `I am painting with the brush that you are painting with too.'
\bg. *Ik schilder met de kwast met wat jij ook schildert.\\
 I paint with the brush with what you also paint\\
 `I am painting with the brush that you are painting with too.'

 \ex.\label{ex:headless}
 \ag. Ik schilder waarmee jij ook schildert.\\
  I paint {where with} you also paint\\
  `I am painting with what you are painting with too.'
 \bg. *Ik schilder met wat jij ook schildert.\\
  I paint with what you also paint\\
  `I am painting with what you are painting with too.'

\tit{'t} `it' and \tit{wat} `what' do not combine with prepositions. They are substituted by respectively \tit{'r} `there' and \tit{waar} `where'.

The next section discusses the role of constituency in \tsc{r}-pronouns.


\subsection{\tit{Met wat} `with what' shows up}

Sometimes \tit{met wat} shows up!

Evidence from that comes from the complementary distribution of \tit{waarmee} and \tit{met wat} in mismatching free relatives.

This is a free relative construction in which the two predicates (the one in the main clause and the one in the embedded clause) combine with two different cases (i.e. the case requirements do not match). I illustrate this in \ref{ex:gekochtwaarmee}. The predicate in the embedded clause, \tit{schildert} `paint', combines with an instrumental object. The predicate in the main clause clause, \tit{gekocht} `bought' combines with an accusative DP. The \tsc{r}-pronoun \tit{waarmee} `with what' is used here.\footnote{In this example, \tit{waar} `where' takes \tit{-mee} `with' to the left edge of the embedded clause. It is also possible for \tit{-mee} `with' to be stranded, and \tit{waar} `where' to be moved to the left edge of the embedded clause on its own.

\exg. Ik heb gekocht waar jij mee schildert.\\
 I have bought waar you with paint\\
 `I bought what you are painting with.'\label{ex:meealong}
\z.

\phantom{x}
}

\exg. Ik heb gekocht waarmee jij schildert.\\
 I have bought {where with} you paint\\
 `I bought what you are painting with.'\label{ex:gekochtwaarmee}

If the predicates are switched around between the clauses, the \tsc{r}-pronoun does not appear anymore. In \ref{ex:schildermet}, \tit{schilder} `paint' combines with an instrumental object in the main clause and \tit{gekocht} `bought' combines with an accusative object in the embedded clause. The use of an \tsc{r}-pronoun is ungrammatical, as indicated by the ungrammaticality of \ref{ex:schilderwaarmee}. Instead, a combination of the regular instrumental preposition \tit{met} `with' and the regular wh-pronoun \tit{was} `what' in used.

\ex.\label{ex:schildermet}
\ag. *Ik schilder waarmee jij hebt gekocht.\\
 I paint {where with} you have bought\\
 `I paint with what you bought.'\label{ex:schilderwaarmee}
\bg. Ik schilder met wat jij hebt gekocht.\\
 I paint with what you have bought\\
 `I paint with what you bought.'\label{ex:schildermetwat}

The use of \tit{met wat} `with what' is ungrammatical in the context in which \tit{waarmee} `with what' appeared in \ref{ex:gekochtwaarmee}. This is illustrated in \ref{ex:gekochtmetwat}.

\exg. *Ik heb gekocht met wat jij schildert.\\
 I have bought with what you paint\\
 `I bought what you are painting with.'\label{ex:gekochtmetwat}


In the introduction I discussed the distribution between \tit{waarmee} `with what' and \tit{met wat} `with what' in free relatives with predicates that combine with different cases. Table \ref{tbl:distribution} repeats the generalization. When the main clause predicate combines with an accusative and the embedded clause predicate with an instrumental, \tit{waarmee} `with what' is grammatical and \tit{met wat} `with what' is ungrammatical. When the main clause predicate combines with an instrumental and the embedded clause predicate with an accusative, \tit{waarmee} `with what' is ungrammatical and \tit{met wat} `with what' is used.

\begin{table}[ht]
	\center
	\caption {Distribution between \tit{waarmee} and \tit{met wat}}
	\begin{minipage}{0.45\linewidth}
		\begin{tabularx}{\textwidth}{ccc}
		\toprule
                              & \tit{waarmee} & \tit{met wat} \\
		\midrule
    m:\tsc{acc}, e:\tsc{ins}  & ✔             & *             \\
    m:\tsc{ins}, e:\tsc{acc}  & *             & ✔             \\
    \bottomrule
\end{tabularx}
\label{tbl:distribution}
\end{minipage}
\end{table}

In this section I first show that \tsc{r}-pronouns are the default complement of a preposition. Next, I illustrate that a necessary requirement for an \tsc{r}-pronoun is that is forms a proper constituent.

Let me now return to the mismatching free relatives. I repeat the relevant grammatical examples in \ref{ex:grammatical}.

\ex.\label{ex:grammatical}
\ag. Ik heb gekocht waarmee jij schildert.\\
 I have bought {where with} you paint\\
 `I bought what you are painting with.'\label{ex:grammaticalwaar}
\bg. Ik schilder met wat jij hebt gekocht.\\
 I paint with what you have bought\\
 `I paint with what you bought.'\label{ex:grammaticalwat}

In the preceding section I showed that ʀ-pronouns are expected in combinations with prepositions. This means that the use of \tit{waarmee} `with what' \ref{ex:grammaticalwaar} is not surprising. Something that is surprising is the use of \tit{met wat} `with what' in \ref{ex:grammaticalwat}, and this is the example something more needs to be said about. In the remainder of this section I argue that this `something more' is that the instrumental object in \ref{ex:grammaticalwat} does not form a proper constituent, i.e. it is not a constituent to the exclusion of any other elements. The other side of the coin is that constructions with \tsc{r}-pronouns contain a object that does form a proper constituent.

Below I repeat the examples with instrumentals I discussed so far in this paper.

\ex.
\ag. Ik schilder 'r mee.\\
 I paint there with\\
 `I am painting with it.'\label{ex:const1}
\bg. Waarmee schilder jij?\\
{where with} paint you with\\
 `What are you painting with?'\label{ex:const2}
\bg. Ik schilder met de kwast [waarmee jij ook schildert].\\
 I paint with the brush {where with} you also paint\\
 `I am painting with the brush that you are painting with too.'\label{ex:const3}
\bg. Ik schilder [waarmee jij ook schildert].\\
 I paint {where with} you also paint\\
 `I am painting with what you are painting with too.'\label{ex:const4}

In each of these examples the instrumental object forms a constituent at a certain point in the derivation. In \ref{ex:const1}, the instrumental object forms a proper constituent in the surface order, as shown in \ref{ex:const1stage}. In \ref{ex:const2}, the instrumental object forms a proper constituent before wh- and V2- movement, shown in \ref{ex:const2stage}.
The structure in \ref{ex:const3stage} represents a stage in the derivation of the embedded clauses in \ref{ex:const3} and \ref{ex:const4}. Again, in the stage, which comes before relative movement of the pronoun to the left periphery of the relative clause, the instrumental object forms a proper constituent.

\ex.
\a. [[ik] [[schilder] ['r mee]]]\label{ex:const1stage}
\b. [[jij] [[schilder] [waarmee]]]\label{ex:const2stage}
\b. [[jij] [[ook] [[schilder] [waarmee]]]]\label{ex:const3stage}

%There is additional evidence for the fact that \tit{waarmee} `with what' forms a constituent in the constructions above. In the example in \ref{ex:const2}-\ref{ex:const4} only \tit{waar} `where' was fronted, but \ref{ex:constwaarmee} shows that also the phrase containing \tit{-mee} with' can be moved. As wh-movement can only target constituents, it follows that \tit{waar} `where' and \tit{-mee} `with' have to form a constituent.

%\ex.\label{ex:constwaarmee}
%\ag. Waar -mee schilder jij?\\
 %{where -with} paint you\\
 %`What are you painting with?'
%\bg. Ik schilder [waar -mee jij ook schildert].\\
 %I paint {where -with} you also paint\\
 %`I am painting with what you are painting with too.'
%\bg. Ik schilder met de kwast [waar -mee jij ook schildert].\\
 %I paint with the brush {where -with} you also paint\\
 %`I am painting with the brush that you are painting with too.'

The mismatching free relative in \ref{ex:grammaticalwat} is not the only construction in which the string \tit{met wat} `with what' appears. I give examples of two more occurrences in \ref{ex:moremetwat}. In \ref{ex:watindef}, \tit{wat} `what' is the \tit{wat} `what' in the so-called \tit{wat voor} `what for'-construction \citep[cf.][]{corver1991}.
In \ref{ex:watindef}, \tit{wat} appears as a quantifier, and it means `some'. In both construction \tit{wat} `what' takes a complement and \tit{met wat} `with what' do not form a proper constituent. The brackets within the examples indicate the constituency.

\ex.\label{ex:moremetwat}
\ag. [Met [wat [voor [potloden]]] teken jij?\\
 with what for pencils draw you\\
 `What kind of pencils do you with?'\label{ex:watwasfur}
\bg. Ik wil graag thee [met [wat [suiker]]].\\
 I want please tea with some sugar\\
 `I would like to have tea with some sugar.'\label{ex:watindef}

Let me now show how this applies to the examples with the mismatching free relatives. The two predicates I used in the free relatives are \tit{kopen} `to buy' and \tit{schilderen}  `to paint'. \tit{Kopen} `to buy' takes an accusative DP as its object, illustrated in \ref{ex:kopen}. \tit{Schilderen} `to paint' can take an instrumental as its object, shown in \ref{ex:schilderen}.\footnote{\tit{Schilderen} also optionally takes an (accusative) object, but I am focussing on the instrumental object here.}

\ex.
\ag. Ik koop het schilderij.\\
 I buy the painting\\
 `I am buying the painting.'\label{ex:kopen}
\bg. Ik schilder met een kwast.\\
 I paint with a brush\\
 `I am painting with a brush.'\label{ex:schilderen}

I repeat the mismatching free relative in which \tit{waarmee} `met wat' appears in \ref{ex:mismatchwaarmee}. The predicate \tit{schildert} `paints' combines in the embedded clause with the instrumental object. The instrumental object forms a proper constituent within the embedded clause, and the it can be realized as the \tsc{r}-pronoun and postposition \tit{waarmee} `with what'.\footnote{I assume that the accusative case requirement of \tit{gekocht} `bought' is satisfying by grafting a subconstituent of \tit{waarmee} `with what' \citep{bergsma2019}.}

\exg. Ik heb gekocht [waarmee jij schildert].\\
 I have bought {where with} you paint\\
 `I bought what you are painting with.'\label{ex:mismatchwaarmee}

Next, we arrive at the mismatching free relative in which \tit{waarmee} `with what' cannot be used, but \tit{met wat} `with what' appears. The embedded clause predicate \tit{gekocht} `bought' combines with an accusative DP. The accusative object of a verb is always \tit{wat} `what', as I showed in \ref{ex:wat}. The instrumental only comes into the picture in the main clause, when \tit{schilder} `paint' combines with an instrumental object. At no point in the derivation does the instrumental object form a proper constituent, and \tit{waarmee} `with what' does not surface.

\exg. Ik schilder met [wat jij hebt gekocht].\\
 I paint with what you have bought\\
 `I paint with what you bought.'\label{ex:mismatchmetwat}

\ref{ex:summaryconst} summarizes what I showed in this section. \tit{Met wat} `with what' can never surface when \tit{met} `with' and \tit{wat} `what' form a proper constituent. It always becomes \tit{waarmee} `with what'. This is schematically shown in \ref{ex:waarmeefr}.
There are other contexts in which \tit{met wat} `with what' appears. This can be either when \tit{wat} `what' takes a complement, or when \tit{wat} is part of the a clause that \tit{met} `with' is not a part of. This last option is schematically showed in \ref{ex:metwatfr}, and it represents the mismatching free relative in \ref{ex:mismatchmetwat}.

\ex.\label{ex:summaryconst}
\a. [[met] [wat]] → [waarmee]\label{ex:waarmeefr}
\b. [met [wat [X]]]
\b. [met [[wat] [X]]]\label{ex:metwatfr}

\tit{Met} `with' is a preposition that combines with full DPs and animate pronouns. \tit{Wat} `what' is a wh-element that appears as subject or object.

In the next section I decompose \tit{waarmee} `with what' and \tit{met wat} `with what'. I show that both spell out the same set of features, but the distribution is different.




\section{Taking \tit{waarmee} and \tit{met wat} apart}

Along the way I introduce some background

Things to explain
*P pronoun-inanimate

\tit{r} is the locative
\tit{met} changes form

 The idea is that both expressions realize the same features, and \tit{waarmee} `with what' takes precedence when all features form a constituent. The morphemes I distinguish in \tit{waarmee} `with what' are \tit{w}, \tit{-aa-}, \tit{-r} and \tit{-mee}. Within \tit{met wat} `with what' I distinguish \tit{met}, \tit{w}, \tit{a} and \tit{-t}. \ref{ex:decompose} shows this as well.

\ex.\label{ex:decompose}
\a. w -aa -r -mee
\b. met w -a -t

In this section I investigate the internal structure of \tit{waarmee} `with what' and \tit{met wat} `with what' to capture the phonological similarities and differences between the two forms. First, I identify \tit{w} and \tit{a} as morphemes that appear in both expressions. Putting these two aside, I concentrate on \tit{'rmee} `with it' and \tit{met 't} `with it'.

The elements \tit{w} and \tit{a} express the same syntactic structure in both expressions. The elements \tit{met} and \tit{'t} together also express the same features as \tit{-mee} and \tit{'r} together, but the distribution differs. \tit{Met} expresses less structure than \tit{-mee}, and \tit{'t} expresses more structure than \tit{'r}. \tit{Met} is shown to be an preposition, and \tit{-mee} is a postposition. It also becomes clear that the \tit{'r} in fact corresponds to the locative \tit{r} in Dutch.


\subsection{Overlap: \tit{w-} and \tit{-a-}}

Let me start with the morphemes \tit{w} and \tit{a} that appear in both expressions. I assume that they correspond to the same syntactic structure in both \tit{waarmee} and \tit{met wat}. As I am interested in the differencs between the two expression, I do not discuss the featural content of \tit{w} and \tit{a} into depth.

For \tit{w} I follow \citet{hachem2015} who investigated \tit{d} and \tit{w} elements in German and Dutch. In her work, \tit{d} establishes a definite reference and \tit{w} triggers the construction of a set of alternatives in the sense of \citet{rooth1992} (see \citealt{hachem2015} for discussion).\footnote{Throughout the paper, ⇔ indicates the pairing between a lexical tree and a phonological form in a lexical entry, and ⇒ indicates how a node in the syntactic structure is spelled out.}

\ex. \begin{forest}
[WP
    [W, roof]
]
{\draw (.east) node[right]{⇔ \tit{w}}; }
\end{forest}\label{ex:entryw}

I follow several authors (cf. \citealt{lander2016,noonan2017dutch,wesseling2018}) in assuming the morpheme \tit{a} to be related to deixis. Dutch distinguishes between proximal by using \tit{ie} (/i:/) and \tit{i} (/ɪ/) and distal by using \tit{aa} (/aː/) and \tit{a} (/ɑ/), illustrated in \ref{ex:deixis}.\footnote{A question that remains open is why wh-elements only combine with the distal marker \tit{a}, and not with proximal marker \tit{i/ie}.} I analyze the transformation from /ɪ/ into /i:/ and /ɑ/ into /aː/ as a result of the final r.

\ex.\label{ex:deixis}
\ag. h-ie-r\\
 here\\
\bg. d-aa-r\\
 there\\
\bg. d-i-t\\
 this\\
\bg. d-a-t\\
 that\\
 \z.
 \z.

For the purpose of this paper I simply let \tit{a} correspond to \tsc{deixP}.

\ex. \begin{forest}
[deixP
    [deix, roof]
]
{\draw (.east) node[right]{⇔ \tit{a}}; }
\end{forest}\label{ex:entrya}

I put \tit{w} and \tit{a} aside for now, assuming they spell out the same syntactic structure in \tit{waarmee} `with what' and \tit{met wat} `with what'. This leaves \tit{'r -mee} `with it' and \tit{met 't} `with it'.

\ex.
\ag. 'r -mee\\
there with\\
\b. met 't\\
with it\\
\z.
\z.


\subsection{Differences: \tit{'rmee} vs. \tit{met 't}}

In this section I discuss the forms \tit{met 't} `with it` and \tit{'rmee} `with it'. I set up an account that makes the ungrammaticality of \tit{met 't} and the appearance of \tit{'rmee} follow from spellout. The analysis accounts for the following two observations, taking \tit{met 't} as the point of departure. First, \tit{met} `with' changes from being a preposition to being a postposition, and its form changes into \tit{-mee}. This process is restricted to inanimate pronouns, and it does not apply to full DPs and animate pronouns. Second, \tit{'t} is replaced by \tit{'r}, a morpheme that is associated with the locative in Dutch.


\subsubsection{\tit{'t} vs. \tit{'r}}

In this section I give the lexical entries for \tit{'t} and \tit{'r}, and I show that \tit{'r} is actually the base form and \tit{'t} a suppletive nominative, accusative and dative.

Let me start with the lexical entry for \tit{'t}. \tit{'t} `it' can be used as subject (associated with nominative), direct object (associated in accusative) and indirect object (associated with dative), as shown in \ref{ex:tsubobj}.

\ex.\label{ex:tsubobj}
\ag. 't Staat in de hal.\\
 \tsc{3sg.n.nom} stands in the hallway\\
 `It is standing in the hallway.'\label{ex:tnoclitic}
\bg. Ik zie 't.\\
 I see \tsc{3sg.n.acc}\\
 `I see it.'
\bg. Ik heb 't een klap gegeven.\\
 I have \tsc{3sg.n.dat} a hit given\\
 `I gave it a hit.'

 Pronouns in other genders alternate between nominative (non-oblique) and accusative/dative (oblique) in these contexts, illustrated in \ref{ex:hemsubobj}.

 \ex.\label{ex:hemsubobj}
 \ag. Hij staat in de hal.\\
  \tsc{3sg.m.nom} stands in the hallway\\
  `He is standing in the hallway.'
 \bg. Ik zie hem.\\
  I see \tsc{3sg.m.acc}\\
  `I see it.'
 \bg. Ik heb hem een klap gegeven.\\
  I have \tsc{3sg.m.acc} a hit given\\ with both alternatives. I work the proposal out with't  realizing all cases.

For case, I follow \citet{caha2009} that case features case features are organized the containment relation in \ref{ex:casetree}. The higher, more complex cases contain the smaller, less complex cases.\footnote{there has been discussion on the genitive. starke on two accusatives and two datives. this paper: only small accusative and dative and genitive is above both of them.}

 \ex. \label{ex:casetree}
 \begin{forest} boom
 [\tsc{comP}
     [\tsc{f6}]
     [\tsc{insP}
         [\tsc{f5}]
         [\tsc{genP}
             [\tsc{f4}]
             [\tsc{datP}
                 [\tsc{f3}]
                 [\tsc{accP}
                     [\tsc{f2}]
                     [\tsc{nomP}
                         [\tsc{f1}]
                         [DP
                             [...,roof]
                         ]
                     ]
                 ]
             ]
         ]
     ]
 ]
 \end{forest}

 Following the distinctions from \citet{cardinaletti1996}, \tit{'t} `it' is a weak pronoun. It is not a clitic, because it can occur in sentence initial position, shown in \ref{ex:tnoclitic}. It is not a strong pronoun, because it cannot be coordinated, as indicated in  \ref{ex:tcoordinated}. \ref{ex:datcoordinated} shows that \tit{'t} `it' needs to combine with \tit{da-}/\tit{di-} to be able to be coordinated.

 \ex.
 \ag. *Hij en 't staan in de hoek.\\
  he and it stand in the corner\\
  `He and it are standing in the corner.'\label{ex:tcoordinated}
 \bg. Hij en dit/dat staan in de hoek.\\
  he and this/that stand in the corner\\
  `He and it are standing in the corner.'\label{ex:datcoordinated}

I assume that the \tit{'t} contains the ontological category \tsc{thing} \citep{kayne2005}. The feature Σ indicates that the pronoun is a weak pronoun. I leave possible number and gender features out because they do not play a role in this paper. The morpheme \tit{'t} can act as nominative, accusative and dative, as I showed in \ref{ex:tsubobj}.\footnote{ Another possibility is to claim that \tit{'t} can only spell out \tsc{thing} and Σ and it combines with a zero suffix for the cases up to dative. This could be the same zero marker that full DPs combine with.

\ex.
\ag. De kast-∅ staat in de hal.\\
 the cabinet-\tsc{nom} stands in the hallway\\
 `The cabinet is standing in the hallway.'
\bg. Ik zie de kast-∅.\\
 I see the cabinet-\tsc{acc}\\
 `I see the cabinet.'
\bg. Ik heb de kast-∅ een klap gegeven.\\
 I have the cabinet-\tsc{dat} a hit given\\
 `I gave the cabinet a hit.'

The proposed account fares equally well with both alternatives. I work the proposal out with \tit{'t} realizing the cases up to the dative.} Taking this all together, \tit{'t} has the lexical entry given in \ref{ex:entryt}.

\ex. \begin{forest} boom
 [\tsc{datP}
     [\tsc{f3}]
     [\tsc{accP}
         [\tsc{f2}]
         [\tsc{nomP}
             [\tsc{f1}]
             [ΣP
                 [Σ]
                 [\tsc{thingP}
                     [\tsc{thing}, roof]
                 ]
             ]
         ]
     ]
 ]
 {\draw (.east) node[right]{⇔ \tit{'t}}; }
 \end{forest}\label{ex:entryt}

 This lexical entry can lexicalize the \tsc{datP}, but also the \tsc{accP} and \tsc{nomP}. This is due to the Superset Princple.

  \ex. The Superset Principle \citet{starke2009}: \\
  A lexically stored tree matches a syntactic node iff the lexically stored tree contains the syntactic node.

 In other words, a lexically stored structure does not have to be identical to the syntactic structure. It is enough if the syntactic structure is contained in the lexically stored tree. This has a as consequence that the lexical entry in \ref{ex:entryt} can also be inserted in \ref{ex:tacc} and \ref{ex:tnom}.

 \ex.
 \a. \begin{forest} boom
 [\tsc{accP}
     [\tsc{f2}]
     [\tsc{nomP}
         [\tsc{f1}]
         [ΣP
             [Σ]
             [\tsc{thingP}
                 [\tsc{thing}, roof]
             ]
         ]
     ]
 ]
 \end{forest}\label{ex:tacc}
 \b. \begin{forest} boom
 [\tsc{nomP}
     [\tsc{f1}]
     [ΣP
         [Σ]
         [\tsc{thingP}
             [\tsc{thing}, roof]
         ]
     ]
 ]
 \end{forest}\label{ex:tnom}

Let me move on to \tit{'r}. \tit{'r} `there' can be used as a locative.

 \exg. Ik ben er al geweest.\\
  I am there already been\\
  `I have already been there.'

I follow \cite{baunaz2018} in assuming that the ontological category \tsc{location} contains \tsc{thing}.\footnote{\citet{baunaz2018} place in addition \tsc{person} between \tsc{thing} and \tsc{location}, which I left out here.}

\ex. \begin{forest} boom
[\tsc{loc}P
    [\tsc{location}]
    [\tsc{thingP}
        [\tsc{thing}, roof]
    ]
]
{\draw (.east) node[right]{⇔ \tit{'r}}; }
\end{forest}\label{ex:entryr}

Notice already here that, via the superset principle, \tit{'r} can be used to realize the feature \tsc{thing} as well, as it is contained in \tsc{locP}. Moreover, in a syntactic structure like in \ref{ex:rthing} the lexical entry \ref{ex:entryr} will be inserted and not \ref{ex:entryt}.

\ex.
\begin{forest} boom
 [\tsc{thingP}
     [\tsc{thing}, roof]
 ]
{\draw (.east) node[right]{⇒ \tit{'r}}; }
\end{forest}\label{ex:rthing}

This is due to the Elsewhere Condition. The idea is that when two lexical entries are both candidates for spellout, the most specific is inserted.

\ex. The Elsewhere Condition (\citealt{kiparsky1973}, formulated as in \citealt{caha2020}):\\
When two entries can spell out a given node, the more specific entry wins. Under the Superset Principle governed insertion, the more specific entry is the one which has fewer unused features

The syntactic structure in \ref{ex:entryr} only has \tsc{loc} as an unused feature, whereas in \ref{ex:entryt} Σ up to \tsc{f3} remain unused.

What this means is that the base form of the neuter singular pronoun in Dutch is actually \tit{'r} and \tit{'t} should be analyzed as a suppletive nominative, accusative and dative. The base form only shows up in the higher cases, from instrumental on, see Table \ref{tbl:dutchcases}.

\begin{table}[ht]
	\center
	\caption {Fragment Dutch \tsc{n.sg}}
	\begin{minipage}{0.56\linewidth}
		\begin{tabularx}{\textwidth}{ccccccc}
		\toprule
              & \tsc{n.sg} \\
		\midrule
    \tsc{nom} & 't         \\
    \tsc{acc} & 't         \\
    \tsc{dat} & 't         \\
    \tsc{gen} & 'r-van     \\
    \tsc{ins} & -r -mee     \\
    \bottomrule
\end{tabularx}
\label{tbl:dutchcases}
\end{minipage}
\end{table}

A similar situation appears in Iron Ossetic, shown in \ref{tbl:ossetic}. In the first person singular of this language, it is only the nominative that is suppletive: \tit{æz}. The higher cases have the stem \tit{mæn} and they combine with the suffixes that nouns normally also combine with.

\begin{table}[ht]
	\center
	\caption {Fragment Iron Ossetic 1.\tsc{sg} and noun \citep{erschler2012}}
	\begin{minipage}{0.3\linewidth}
		\begin{tabularx}{\textwidth}{ccccccc}
		\toprule
              & 1.\tsc{sg}  & head    \\
		\midrule
    \tsc{nom} & æz          & sær-∅   \\
    \tsc{acc} & mæn-∅       & sær-∅   \\
    \tsc{gen} & mæn (??)    & sær-y   \\
    \tsc{ins} & mæn-æj      & sær-æj  \\
    \tsc{dat} & mæn-æn      & sær-æn  \\
    \bottomrule
\end{tabularx}
\label{tbl:ossetic}
\end{minipage}
\end{table}

\citet{caha2019competition} uses evidence from a phenomenon called suspended affixation to argue that \tit{mæn} is a caseless stem and and \tit{æz}. Consider the ordinary coordination in \ref{ex:horseoxboth}. Both conjuncts are marked by a plural marker and a case marker. Suspended affixation is shown in \ref{ex:horseoxone}. Here the case marker only appears on the second conjunct and not on the first one without changing the interpretation.
\tit{Bæx-tæ} `horse-\tsc{pl}' in \ref{ex:horseoxone} does not carry any case marking here.

\ex.
\ag. bæx-t-imæ æmæ gæl-t-imæ\\
horse-\tsc{pl-com} and ox-\tsc{pl-com}\\\label{ex:horseoxboth}
\bg. bæx-tæ æmæ gæl-t-imæ\\
horse-\tsc{pl} and ox-\tsc{pl-com}\\
`with horses and oxen' \hfill (Iron Ossetic, \citep[165]{erschler2012})\label{ex:horseoxone}

\ref{ex:suspendedme} gives examples of the first person singular in a suspended affixation contexts. It shows that it is \tit{mæn} that appears as a caseless first conjunct and that the use of \tit{æz} is ungrammatical. This means that \tit{mæn} is the bare stem that combines with case markers, and \tit{æz} the suppletive nominative. In Section \ref{sec:derivation} I show how a derivation with this type of elements works in nanosyntax.

\ex.\label{ex:suspendedme}
\ag. mæn æmæ Zauyr-æn\\
 1.\tsc{sg} and Zaur-\tsc{dat}\\\label{ex:izaurboth}
\bg. *æz æmæ Zauyr-æn\\
 1.\tbf{sg} and Zaur-\tsc{dat}\\
 `me and Zaur' \hfill (\citealt[39]{беляев2014} after \citealt{caha2019competition})\label{ex:izaurone}

The point of showing the Ossetic example is that Dutch is not unique in having suppletive forms that are less marked (in this case nominative, accusative and dative), and higher cases that are a combination of a suffix and a base form.



\subsubsection{\tit{-mee} vs. \tit{met}}

The last two forms to specify lexical entries for are \tit{-mee} `with' and \tit{met} `with'. An important distinction between these two is that \tit{-mee} appears after the element it combines with (\tsc{'r}), while \tit{met} appears before the element it combines with (\tit{'t}). I will analyze \tit{-mee} as a postposition and \tit{met} as a preposition.\footnote{A topic related to this paper is the different positioning of identical adpositions in Dutch (see \citet{caha2010} for an account of German and Dutch and \citet{pretorius2017} for Afrikaans). In \ref{ex:dutchin}, \tit{in} changes meaning dependening on whether it proceeds or follows the DP, it is respectively locational or directional.

\ex.\label{ex:dutchin}
\ag. Ik klim in de boom.\\
 I climb in the tree\\
 `I am climbing in the tree.'
\bg. Ik klim de boom in.\\
 I climb the tree in\\
 `I am climbing into the tree.'

In \ref{ex:dutchin}, the movement of the adposition is driven by movement, and it is meaningful. The movement I discuss in this paper with \tsc{r}-pronouns is driven by spellout, and it is meaningless.} In this section I discuss the relation between prepositions and postpositions, and how this is modeled with the case hierarchy in Nanosyntax \citep{caha2009}.

In the previous section I argued that \tit{'t} realizes case features up to \tsc{f3} (see \ref{ex:entryt}). However, case can also be expressed by prepositions (or prefixes) and postpositions (or suffixes). The division between which cases are expressed by prepositions and which are expressed by postpositions is not arbitrary.

\ex. The preposition/postposition hierarchy
\a. If the expression of a particular case in the Case sequence (below) involves a preposition, then all cases to its right do as well.
\b. The Case sequence: \tsc{nom – acc – dat – gen - ins – com} \hfill \citep{caha2009}

The result of that is that a PP can contain a preposition and a suffix, as in \ref{ex:loffel}. The dative suffix is used with a genitive preposition.

\exg. (die Farbe) von ein -em Löffel\\
the color of a -\tsc{dat.sg} spoon\\
`(the color) of a spoon' \hfill (German)\label{ex:loffel}

With the case hierarchy in nanosyntax this can be modeled by letting the DP move as high as above the \tsc{datP} in the syntactic structure. The features below the \tsc{datP} are realized as a suffix, and the features above \tsc{datP} are realized as a preposition.

There is variation with respect to how high a DP can move in the structure, both between languages and within languages. An example from the latter comes from Bulgarian. \ref{ex:bulpron} shows that pronouns can take the suffix \tit{-i} to realize dative, but full DPs need a preposition \tit{na} `to'.

\ex.\label{ex:bulgarian}
\ag. Tazi duma m -i e nepoznata.\\
that word I -\tsc{dat} is unfamiliar\\
`That word is unfamiliar to me.'\label{ex:bulpron}
\bg. Tazi duma e nepoznata na sina mi.\\
that word is unfamiliar to son my\\
`That word is unfamiliar to my son.'\label{ex:buldpto} \hfill \citep[39]{caha2009}

In Dutch the split is not between pronouns and full DPs but between inanimate pronouns on the one hand and animates and full DPs on the other hand. In Dutch, inanimate pronouns combine with the postposition \tit{-mee} (see \ref{ex:dutchina} and not with the preposition \tit{met} (see \ref{ex:dutchinmet}).

\ex.
\ag. Ik schilder 'r -mee.\\
 I paint there -with\\
 `I am painting with it.'\label{ex:dutchina}
\bg. *Ik schilder met 't\\
 I paint with it\\
 `I am painting with it.'\label{ex:dutchinmet}

Animate pronouns and full DPs, however, combine with the preposition met, as shown in \ref{ex:dutchan} and \ref{ex:dutchdp}. The use of the postposition \tit{-mee} is ungrammatical (see \ref{ex:dutchanmee} and \ref{ex:dutchdpmee}).

\ex.
\ag. Ik schilder samen met hem.\\
 I paint together with him\\
 `I am painting with him.'\label{ex:dutchan}
\bg. Ik schilder samen met de man.\\
 I paint together with the man\\
 `I am painting together with the man.'\label{ex:dutchdp}
\bg. *Ik schilder samen 'm-mee.\\
 I paint together him-with\\
 `I am painting together with him.'\label{ex:dutchanmee}
\bg. *Ik schilder samen de man-mee.\\
 I paint together the man-with\\
 `I am painting together with the man.'\label{ex:dutchdpmee}

 In other words, inanimates can move higher than animates and full DPs in Dutch. To be more precise, the inanimate \tit{'t} is replaced by \tit{'r}, and this element can move as high as above the dative to combine with \tit{-mee}. Later I return to what it is that prevents animates and full DPs from being combined with \tit{-mee}.

First I show to is what the lexical entry of \tit{-mee} looks like. This needs to capture three facts. \tit{-mee} combines with \tit{'r}, and it is a postposition. First, \tit{-mee} expresses instrumental (and comitative) case and it combines with \tit{'r}.\footnote{For reasons of space I leave \tsc{f6} out of the lexical entries and discussion, even though \tit{-mee} and \tit{met} can also express comitative.}
So far, \tit{'r} `there' only realizes the feature \tsc{thing}. This leaves Σ and \tsc{f1} to \tsc{f5} to be realized by \tit{-mee}. I give the lexical tree of \tit{-mee} in \ref{ex:entrymee}.

\ex. \begin{forest} boom
    [\tsc{insP}
        [\tsc{f6}]
        [\tsc{genP}
            [\tsc{f4}]
            [\tsc{datP}
                [\tsc{f3}]
                [\tsc{accP}
                    [\tsc{f2}]
                    [\tsc{nomP}
                        [\tsc{f1}]
                        [ΣP
                            [Σ]
                        ]
                    ]
                ]
            ]
        ]
    ]
{\draw (.east) node[right]{⇔ \tit{-mee}}; }
\end{forest}\label{ex:entrymee}

This leads us to the second point: \tit{-mee} is a postposition. Notice that the foot of the structure has a singleton feature. Nanosyntax distinguishes pre-elements from post-elements by the shape of their lexical entry \citep{starke2018}. As a result, the whether an element is pre or post is lexically stored as follows from the spellout procedure. I illustrate this in Section \ref{sec:derivation}. Post-elements have a unary bottom (i.e. the foot of the tree is a single feature), so they can only appear as the result of movement. Post-elements have a binary bottom (i.e. the foot of the tree consists of two features), so they cannot be a candidate as a result of movement.

Why does \tit{-mee} not combine with animates and full DPs? I claim that has to do with the bottom feature of the lexical entry of \tit{-mee}. Full DPs do not take features related to pronominal strength. According to the Superset Principle, a lexical tree can also match a syntactic tree with a subpart of the features, but a tree can only shrink from the top, so \tit{-mee} will always realize Σ. I have a less clear answer to why animates do not combine with \tit{-mee}. The crucial difference between animates and inanimates if gender features. For now I assume that gender features are situated between Σ and \tsc{f1}. The lexical entry of \tit{-mee} includes both these features, so any features are incompatible with \tit{-mee}. So far I do not have independent evidence for placing gender features between features of pronominal strength and case, and I leave this for future research.

So far I discussed \tit{-mee} is a postposition, which follows \tit{'r} and is stored with a unary bottom. \tit{Met}, on the other hand, is a preposition, it precedes \tit{'t}, so it should be stored with a binary bottom. The highest case feature \tit{'t} can realize is \tsc{f3}, so the preposition realizes all higher cases up to \tsc{f5}. I give the lexical entry for \tit{met} in \ref{ex:entrymet}.

\ex. \begin{forest} boom
[\tsc{insP}
    [\tsc{f5}]
    [\tsc{f4}]
]
{\draw (.east) node[right]{⇔ \tit{met}}; }
\end{forest}\label{ex:entrymet}

In the next section I put all features back together in a derivation and I show how \tit{waarmee} `with what' surfaces when all features form a constituent. \tit{Met wat} `with what' appears when the functional sequence is disrupted.




\section{In a derivation}\label{sec:derivation}

Before I show that \tit{waarmee} `with what' is used when all features form a proper constituent, I need to make some assumptions about the spellout process in Nanosyntax explicit. Spellout happens in a cyclic derivation, following a spellout algorithm \citep{starke2018}. After each instance of merge, spellout takes place. If no spellout exist for the phrase created by the newly added feature, evacuation movements specified in the spellout algorithm take place. The algorithm is given in \ref{ex:spellout}.

\ex. Merge F and \label{ex:spellout}
 \a. Spell out FP
 \b. If (a) fails, attempt movement of the spec of the complement of \tsc{f}, and retry (a)
 \b. If (b) fails, move the complement of \tsc{f}, and retry (a)

When a new match is found, it overrides previous spellouts.

\ex. Cyclic Override \citep{starke2018}:\\
Lexicalisation at a node XP overrides any previous match at a phrase contained in XP.

If the spellout procedure in \ref{ex:spellout} fails, backtracking takes place. This is a crucial operation to get from the suppletive nominative, accusative and dative \tit{'t} to the base form \tit{'r}.

\ex. Backtracking \citep{starke2018}:\\
When spellout fails, go back to the previous cycle, and try the next option for that cycle.\label{ex:backtracking}

If backtracking also does not help, a specifier is constructed. This is what happens when the preposition \tit{met} is inserted.

\ex. Spec Formation \citep{starke2018}:\\
If Merge F has failed to spell out (even after backtracking), try to spawn a new derivation providing the feature F and merge that with the current derivation, projecting the feature F at the top node.\label{ex:specformation}

I first show how \tit{'rmee} `with it' is constructed. I leave out \tit{w} and \tit{a}, because it unnecessarily complicates the story.\footnote{I assume that the WP and \tsc{deixP} appear lower in the structure than the case features, so the functional sequence is as given in \ref{ex:fseq}.

\ex. [ [ [ [ [ [ [ \tsc{thing} ] \tsc{deix} ] W ] \tsc{f1} ] \tsc{f2} ] \tsc{f3} ] \tsc{f4} ]\label{ex:fseq}
\z.

\phantom{x}

}

We start with \tsc{thing}. The two candidates here are \ref{ex:entryt} and \ref{ex:entryr}. Following the Elsewhere Condition, \ref{ex:entryr} wins the competition because it contains less unused material.

\ex.
\begin{forest} boom
 [\tsc{thingP}
     [\tsc{thing}, roof]
 ]
{\draw (.east) node[right]{⇒ \tit{'r}}; }
\end{forest}\label{ex:thingspellout}

In the next step, Σ is merged. \ref{ex:entryr} is no longer a candidate because it does not contain Σ. \ref{ex:entryt} still is a candidate, because it contains all features in \ref{ex:thingf1}. The spellout is overridden and the structure is realized as \tit{'t}.

\ex. \begin{forest} boom
[ΣP
   [Σ]
   [\tsc{thingP}
       [\tsc{thing}, roof]
   ]
]
{\draw (.east) node[right]{⇒ \tit{'t}}; }
\end{forest}\label{ex:thingf1}

Then \tsc{f1} is merged. This structure can still be realized by \tit{'t}.

\ex. \begin{forest} boom
[\tsc{nomP}
   [\tsc{f1}]
   [ΣP
       [Σ]
       [\tsc{thingP}
           [\tsc{thing}, roof]
       ]
   ]
]
{\draw (.east) node[right]{⇒ \tit{'t}}; }
\end{forest}

The same holds for the next two steps in which \tsc{f2} and \tsc{f3} are merged: the structure can still be spelled out as \tit{'t}.

\ex. \begin{forest} boom
[\tsc{datP}
    [\tsc{f3}]
    [\tsc{accP}
       [\tsc{f2}]
       [\tsc{nomP}
           [\tsc{f1}]
           [ΣP
               [Σ]
               [\tsc{thingP}
                   [\tsc{thing}, roof]
               ]
           ]
       ]
    ]
]
{\draw (.east) node[right]{⇒ \tit{'t}}; }
\end{forest}

Then \tsc{f4} is merged, as shown in \ref{ex:f4no}. \ref{ex:entryt} can no longer spell out the structure, because it does not contain \tsc{f4}. There is also no other candidate to spell out the structure as it is.

\ex. \begin{forest} boom
[\tsc{insP}
    [\tsc{f4}]
    [\tsc{datP}
        [\tsc{f3}]
        [\tsc{accP}
           [\tsc{f2}]
           [\tsc{nomP}
               [\tsc{f1}]
               [ΣP
                   [Σ]
                   [\tsc{thingP}
                       [\tsc{thing}, roof]
                   ]
               ]
           ]
        ]
    ]
]
{\draw (.east) node[right]{⇒ }; }
\end{forest}\label{ex:f4no}

According to the spellout algorithm in \ref{ex:spellout}, it should be attempted to move of the spec of the complement of \tsc{f4}. However, there is no specifier in \ref{ex:f4no}, so this does not apply. The second movement option is complement movement. The complement of \tsc{f4} moves to the specifier of \tsc{insP}, resulting in the structure in \ref{ex:f4comp}. The lexicon does not contain an entry with \tsc{insP} which contains only \tsc{f4}.\footnote{\tit{Met} `with' is not a candidate, because the syntactic structure has a unary bottom and the lexical structure has a binary bottom.}

\ex. \begin{forest} boom
[\tsc{datP}
    [\tsc{datP}
        [\tsc{f3}]
        [\tsc{accP}
           [\tsc{f2}]
           [\tsc{nomP}
               [\tsc{f1}]
               [ΣP
                   [Σ]
                   [\tsc{thingP}
                       [\tsc{thing}, roof]
                   ]
               ]
           ]
        ]
    ]
    {\draw (.east) node[right]{⇒ \tit{'t}}; }
    [\tsc{insP}
        [\tsc{f4}]
    ]
    {\draw (.east) node[right]{⇒ }; }
]
\end{forest}\label{ex:f4comp}

As I formulated in the introduction of this section, the operation called Backtracking is triggered (see \ref{ex:backtracking}). This means that the derivation goes back to the previous cycle, and the next option for that cycle is tried. In this case, the previous cycle is the one in which \tsc{f3} is merged. The next option for that cycle is spec-to-spec movement. As there is no specifier, this does not apply. The option after that is complement movement, shown in \ref{ex:f3comp}. However, there is no match in the lexicon for an \tsc{datP} that contains only \tsc{f3}.

\ex. \begin{forest} boom
[\tsc{accP}
    [\tsc{accP}
       [\tsc{f2}]
       [\tsc{nomP}
           [\tsc{f1}]
           [ΣP
               [Σ]
               [\tsc{thingP}
                   [\tsc{thing}, roof]
               ]
           ]
       ]
    ]
    {\draw (.east) node[right]{⇒ \tit{'t}}; }
    [\tsc{datP}
        [\tsc{f3}]
    ]
    {\draw (.east) node[right]{⇒ }; }
]
\end{forest}\label{ex:f3comp}

This means that backtracking proceeds further, into the cycle in which \tsc{f2} was merged. Again, spec-to-spec movement does not apply because there is no specifier, and complement movement can be tried, but there is no fitting lexical entry available. The same holds for the cycle in which \tsc{f1} is merged.

The situation changes when the derivation comes to the cycle in which Σ was merged. At this stage, \tsc{thing} was realized as \tit{'r}. Again there was no specifier, no spec-to-spec movement does not apply. However, complement movement provides a structure that is a match for the lexical entry in \ref{ex:entrymee}: \tit{-mee}.

\ex. \begin{forest} boom
[ΣP
   [\tsc{thingP}
       [\tsc{thing}, roof]
   ]
   {\draw (.east) node[right]{⇒ \tit{'r}}; }
   [ΣP
       [Σ]
   ]
   {\draw (.east) node[right]{⇒ \tit{-mee}}; }
]
\end{forest}

From this point on the previously unmerged features are merged again one by one. First, \tsc{f1} is merged again, shown in \ref{ex:f1again}. No match exists for this syntactic structure.

\ex. \begin{forest} boom
[\tsc{nomP}
    [\tsc{f1}]
    [ΣP
       [\tsc{thingP}
           [\tsc{thing}, roof]
       ]
       [ΣP
           [Σ]
       ]
    ]
]
{\draw (.east) node[right]{⇒}; }
\end{forest}\label{ex:f1again}

Following the spellout algorithm, the next step is spec-to-spec movement is tried. The result is shown in \ref{ex:f1spec}. In that configuration \tsc{f1} can be realized together with Σ as \tit{-mee}.

\ex. \begin{forest} boom
[\tsc{nomP}
   [\tsc{thingP}
       [\tsc{thing}, roof]
   ]
   {\draw (.east) node[right]{⇒ \tit{'r}}; }
   [\tsc{nomP}
       [\tsc{f1}]
       [ΣP
           [Σ]
       ]
   ]
   {\draw (.east) node[right]{⇒ \tit{-mee}}; }
 ]
\end{forest}\label{ex:f1spec}

The same happens for \tsc{f2}, \tsc{f3} and \tsc{f4}. The features are merged one at a time, there is no spellout after merging the feature, but there is a spellout after spec-to-spec movement. I show the situation after \tsc{f4} is realized as  \ref{ex:spellout'rmee}.

\ex. \begin{forest} boom
[\tsc{insP}
    [\tsc{thingP}
       [\tsc{thing}, roof]
    ]
    {\draw (.east) node[right]{⇒ \tit{'r}}; }
    [\tsc{insP}
       [\tsc{f4}]
       [\tsc{datP}
           [\tsc{f3}]
           [\tsc{accP}
               [\tsc{f2}]
               [\tsc{nomP}
                   [\tsc{f1}]
                   [ΣP
                       [Σ]
                   ]
               ]
           ]
       ]
    ]
    {\draw (.east) node[right]{⇒ \tit{-mee}}; }
]
\end{forest}\label{ex:spellout'rmee}

I skip over the details of how \tit{w} and \tit{a} end up in their positions.\footnote{I assume that WP and \tsc{deixP} are both complex specifiers that are created after \tsc{thing} is spelled out in \ref{ex:thingspellout}. After each instance of merge after that, backtracking always takes place, the complex specifier is detached from the structure and the case features are spelled out together with or as a postposition on \tsc{thing}.} The final result of the structure for \tit{waarmee} `with what' look as in \ref{ex:spelloutwaarmee}.

\ex. \begin{forest} boom
[\tsc{WP}
    [\tsc{WP}
        [\tsc{W}, roof]
    ]
    {\draw (.east) node[right]{⇒ \tit{w}}; }
    [\tsc{deixP}
        [\tsc{deixP}
            [\tsc{deix}, roof]
        ]
        {\draw (.east) node[right]{⇒ \tit{a}}; }
        [\tsc{insP}
            [\tsc{thingP}
               [\tsc{thing}, roof]
            ]
            {\draw (.east) node[right]{⇒ \tit{'r}}; }
            [\tsc{insP}
               [\tsc{f4}]
               [\tsc{datP}
                   [\tsc{f3}]
                   [\tsc{accP}
                       [\tsc{f2}]
                       [\tsc{nomP}
                           [\tsc{f1}]
                           [ΣP
                               [Σ]
                           ]
                       ]
                   ]
               ]
            ]
            {\draw (.east) node[right]{⇒ \tit{-mee}}; }
        ]
    ]
]
\end{forest}\label{ex:spelloutwaarmee}

A consequence of analyzing \tit{-mee} `with' as a postposition is that \tit{r} and \tit{-mee} always form a constituent to the exclusion of \tit{w} and \tit{a}. At first sight this seems problematic, because it is possible for \tit{waar} `where', stranding \tit{-mee} `with'. I repeat the relevant example in \ref{ex:meestranded}.

\exg. Ik heb gekocht waar jij mee schildert.\\
 I have bought waar you with paint\\
 `I bought what you are painting with.'\label{ex:meestranded}

There is no constituent in \ref{ex:spelloutwaarmee} that contains \tit{waar} but not \tit{-mee}. To resolve this situation I follow \citet{noonan2017dutch} in assuming that the phrase containing the adposition (here \tit{-mee}) syntactically moves to a position higher in the structure. The movement of the adposition has the typical distribution of that of verbal particles (cf. \citealt{riemsdijk1978,noonan2017dutch}, and it could possibly be triggered by the feature Σ, associated with weak pronouns.
With \tit{-mee} having moved out, the WP only contains features that are realized as \tit{waar}, and it moves to the left edge of the clause, resulting in the surface order in \ref{ex:meestranded}.

So far I showed how \tit{waarmee} `with what' is derived if all syntactic features form a constituent. Next I address how \tit{waarmee} is blocked and \tit{met wat} `with what' appears when the features do not form a proper constituent is derived. An example of a situation in which all features do not form a constituent is given in \ref{ex:noconst}.

\exg. Ik schilder met wat jij hebt gekocht.\\
 I paint with what you have bought\\
 `I paint with what you bought.'\label{ex:noconst}

I start at the point on which \tit{wat} `what' is part of a syntactic structure with the rest of a relative clause as a sister. Even though \tsc{f3} is not part of the embedded clause, I already added it to the structure. While it is unclear why, syncretic forms seem to behave differently in that they seem to resolve case conflicts in free relatives and related phenomena (cf. \citealt{groos1981,pullum1986,ingria1990}). I give the syntactic structure from which are start in \ref{ex:metwatst}.

\ex. \begin{forest} boom
[CP
    [\tsc{WP}
        [\tsc{WP}
            [\tsc{W}, roof]
        ]
        {\draw (.east) node[right]{⇒ \tit{w}}; }
        [\tsc{deixP}
            [\tsc{deixP}
                [\tsc{deix}, roof]
            ]
            {\draw (.east) node[right]{⇒ \tit{a}}; }
            [\tsc{datP}
                [\tsc{f3}]
                [\tsc{accP}
                   [\tsc{f2}]
                   [\tsc{nomP}
                       [\tsc{f1}]
                       [ΣP
                           [Σ]
                           [\tsc{thingP}
                               [\tsc{thing}, roof]
                           ]
                       ]
                   ]
                ]
            ]
            {\draw (.east) node[right]{⇒ \tit{'t}}; }
        ]
    ]
    [CP
        [...,roof]
    ]
]
\end{forest}\label{ex:metwatst}

At this point \tsc{f4} is merged, as shown in \ref{ex:f4metwat}. Because of the presence of the CP, there is no possibility for \tsc{f4} to be spelled out, even after the regular movements and backtracking.

\ex. \begin{forest} boom
[\tsc{genP}
    [\tsc{f4}]
    [CP
        [\tsc{WP}
            [\tsc{WP}
                [\tsc{W}, roof]
            ]
            {\draw (.east) node[right]{⇒ \tit{w}}; }
            [\tsc{deixP}
                [\tsc{deixP}
                    [\tsc{deix}, roof]
                ]
                {\draw (.east) node[right]{⇒ \tit{a}}; }
                [\tsc{accP}
                   [\tsc{f2}]
                   [\tsc{nomP}
                       [\tsc{f1}]
                       [ΣP
                           [Σ]
                           [\tsc{thingP}
                               [\tsc{thing}, roof]
                           ]
                       ]
                   ]
                ]
                {\draw (.east) node[right]{⇒ \tit{'t}}; }
            ]
        ]
        [CP
            [...,roof]
        ]
    ]
]
\end{forest}\label{ex:f4metwat}

The last resort possibility to spell out features is set in motion: a complex specifier is created, as described in \ref{ex:specformation}. This is illustrated in \ref{ex:metmetwat}.

\ex. \begin{forest} boom
[\tsc{insP}
    [\tsc{insP}
        [\tsc{f5}]
        [\tsc{f4}]
    ]
    {\draw (.east) node[right]{⇒ \tit{met}}; }
    [CP
        [\tsc{WP}
            [\tsc{WP}
                [\tsc{W}, roof]
            ]
            {\draw (.east) node[right]{⇒ \tit{w}}; }
            [\tsc{deixP}
                [\tsc{deixP}
                    [\tsc{deix}, roof]
                ]
                {\draw (.east) node[right]{⇒ \tit{a}}; }
                [\tsc{accP}
                   [\tsc{f2}]
                   [\tsc{nomP}
                       [\tsc{f1}]
                       [ΣP
                           [Σ]
                           [\tsc{thingP}
                               [\tsc{thing}, roof]
                           ]
                       ]
                   ]
                ]
                {\draw (.east) node[right]{⇒ \tit{'t}}; }
            ]
        ]
        [CP
            [...,roof]
        ]
    ]
]
\end{forest}\label{ex:metmetwat}

This section showed how the instrumental inanimate relative pronoun is realized as \tit{waarmee} `with what' when all syntactic features form a constituent. It also showed how \tit{met wat} `with what' appears when all features do not form a proper constituent, and other features intervene.



\section{Conclusion and discussion}

In this paper, I discussed the distribution of \tit{waarmee} `met wat' and \tit{met wat} `with what' is mismatching free relatives. I showed that \tit{waarmee} `with what' appears when all features form a proper constituent, and \tit{met wat} `with what' if they do not.

The described pattern follows from a core assumptions in Nanosyntax: phrasal spellout spells out constituents. Looking more into detail, \tit{waarmee} `with what' takes precedence over \tit{met wat} `with what' because \tit{-mee} has the structure of a postposition. Following the spellout algorithm, \tit{'t} `it' is replaced by \tit{'r} `there', so that it can combine with \tit{-mee} `with'. The preposition \tit{met} `with' is only used if there is no other option to spell out the features.

This proposal is in several aspects in accordance with earlier work.
Just like \citet{riemsdijk1978}, I claim that \tsc{r}-pronouns originate as the complement of P. He argued that due to some kind of suppletion \tit{'t} `it' changes form to \tsc{'r}, which is coincidentally also the locative in Dutch. In my proposal, this suppletion follows naturally from the regular spellout algorithm. Also, it is not a coincidence that the locative appears, as it is an item with little features that spells out \tsc{thing}.
The current proposal differs from \citet{riemsdijk1978} in that it is not the whole complement of P is moved. Instead, only part of the complement of P is extracted, which is generally allowed, also in non-preposition stranding languages \citep{abels2003diss}.




Giving \tit{met} `with' and \tit{-mee} `with' two distinct lexical entries has as a consequence that the phonological overlap between them seems like a coincidence. This can be questioned, because there is only one more preposition that changes form when it appears postpositionally. This preposition is \tit{tot} `to', and it changes into \tit{toe} as a postposition. It has in common with \tit{met} `with' that it is the only preposition in Dutch that has the phonological structure CVt. For now I take the phonological resemblance to be a relic from the past without having any influence on the synchronic data.

n all the other cases the preposition does not change form when it combines with an \tsc{r}-pronoun, e.g. \tit{in}. If this proposal is on the right track, elements as \tit{in} can be used as either a preposition and as a postposition. The lexical entry should then be usable as pre-element and as post-element, so it needs to have a binary bottom and a unary foot at the same time. The lexical entry in \ref{ex:presuf} would be a candidate for such an element. YP can be inserted as a post-element, and XP can be inserted via the Superset Princple as a pre-element.

\ex. \begin{forest} boom
[YP
    [XP
        [X,roof]
    ]
    [YP
        [Y]
    ]
]
{\draw (.east) node[right]{⇔ \tit{in}}; }
\end{forest}\label{ex:presuf}

I leave it to future research to determine whether this is a feasible solution.



\printbibliography

\end{document}
