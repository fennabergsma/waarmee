\documentclass[11pt,a4paper]{article}

\usepackage[margin=1in]{geometry}
%\usepackage{setspace}
%\doublespacing

\usepackage{fenna-files/packages}
\usepackage{fenna-files/commands}
\bibliography{fenna-files/references}{}

\title{The \tsc{r}-pronoun and postposition \tit{waar mee} in Dutch}
\author{Fenna Bergsma\\Goethe-Universität Frankfurt}
\date{\today}

\begin{document}

\maketitle


\section{Introduction}


Dutch has the preposition \tit{met} `with' that expresses the instrumental. In \ref{ex:metdp}, \tit{met} `with' is combined with a full noun phrase. The inanimate pronoun in Dutch is \tit{t} `it'.\footnote{The longer form that is mostly used in writing is \tit{het} `it'. I will use the spoken variant \tit{t} throughout this paper.} \ref{ex:tverb} illustrates that \tit{t} `it' can be used as the object of a verb.

\ex.
\ag. Ik schilder met een kwast.\\
 I paint with a brush\\
 `I am painting with a brush.'\label{ex:metdp}
 \bg. Ik zie t.\\
  I see it\\
  `I see it.'\label{ex:tverb}

\tit{Met} `with' and \tit{t} `it' do not appear together, as illustrated in \ref{ex:neemett}. Instead, Dutch uses the \tsc{r}-pronoun \tit{r} `there' and the postposition \tit{mee} `with', as shown in \ref{ex:jarmee}.\footnote{The \tsc{r}-pronoun \tit{r} in \ref{ex:jarmee} can be written as \tit{er}, \tit{der} and \tit{r} and pronounced as respectively /ɛr/, /dər/ or /ər/. There is no clear meaning difference between these forms that is relevant for the discussion in this paper
(see \citealt{wesseling2018} for discussion).
In my examples I use \tit{r}, but the other forms fit just as well.}$^,$\footnote{In this paper I do not make a distinction between prefixes and prepositions, or between suffixes and postpositions.}

\ex.\label{ex:r meeconst}
\ag. *Ik schilder met t.\\
 I paint with it\\
 `I am painting with it.'\label{ex:neemett}
\bg. Ik schilder r mee.\\
 I paint there with\\
 `I am painting with it.'\label{ex:jarmee}

\tsc{R}-pronouns are nominal elements that are syncretic with locative pronouns, which in Dutch means they contain the morpheme \tit{r} \citep[cf.][]{riemsdijk1978,koopman1994}. The adpositions they combine with obligatorily follow the \tsc{r}-pronoun (compare \ref{ex:jarmee} and \ref{ex:neemeer}).
Notice also that the preposition \tit{met} `with' differs phonologically from the postposition \tit{mee} `with` (see \ref{ex:metdp} and \ref{ex:jarmee}), although they express the same meaning.

\exg. *Ik schilder mee r.\\
 I paint with there\\
 `I am painting with it.'\label{ex:neemeer}

The main question I address in this paper is how to correctly rule out \tit{met t} `with it' in \ref{ex:neemett}, and let the \tsc{r}-pronoun plus postposition \tit{r mee} `with it' in \ref{ex:jarmee} appear. I argue that \tsc{r}-pronouns are not something special, but a consequence of regular spellout mechanisms. Just as \citet{riemsdijk1978}, I analyze the \tsc{r}-pronoun plus postposition as a type of allomorph of the preposition plus pronoun. The crucial difference in the current approach is that spellout - not the stipulation of a filter - rules out the ungrammaticality of the preposition plus inanimate pronoun.

This paper focuses on the instrumental \tsc{r}-pronoun plus postposition \tit{r mee} `with it' in Dutch. This instance is interesting for two reasons. First, just as for all \tsc{r}-pronouns, the \tsc{r}-pronoun is syncretic with the locative. In many of the \tsc{r}-pronouns and postpositions, the meaning component of the locative within the \tsc{r}-pronoun plus postposition is intuitive, as many adpositions express locations, directions, etc. However, an instrumental expresses an instrument, which does not have a meaning component associated with location. Ideally, a formal analysis would treat the syncretism as unaccidental but still allow the locative meaning to be absent.
The second reason for focusing on this particular \tsc{r}-pronoun plus postposition is the form of the adposition. The preposition \tit{met} does not only turn into a postposition, but it also changes into \tit{mee} when it is combined with an \tsc{r}-pronoun. This last observation has so far remained unexplained in the literature on \tsc{r}-pronouns (cf. \citealt{riemsdijk1978,koopman1994,abels2003diss,noonan2017dutch}).

The main generalization is that the instrumental \tsc{r}-pronoun plus postposition \tit{r mee} `with it' by default takes precedence over the instrumental preposition plus inanimate pronoun \tit{met t} `with it'. I show that a single requirement has to be satisfied: the instrumental object needs to form a proper constituent i.e. a constituent to the exclusion of other features. When this condition is not met, the preposition plus inanimate pronoun appear after all. This follows straightforwardly in a system in which spellout targets phrasal constituents: Nanosyntax \citep{starke2009}. I work out this idea capturing the following three observations. First, \tsc{r}-pronouns are syncretic with locatives. Second, inanimate pronouns appear with prepositions, and \tsc{r}-pronouns appear with postpositions. And third, the instrumental preposition and postposition differ in form (\tit{met} vs. \tit{mee}).

This paper is structured as follows. In Section \ref{sec:distribution} I show that \tsc{r}-pronouns plus postpositions and prepositions plus inanimate pronouns are in complementary distribution, depending on whether all features form a proper constituent or not. Besides discussing \tit{r mee} `with it' and \tit{met t} `with it', I introduce the \tsc{wh}-pronouns \tit{waarmee} `with what' and \tit{met wat} `with what'. Examples with these forms allow me illustrate that proper constituency is the critical condition.
In Section \ref{sec:takingapart} I decompose \tit{waar mee} and \tit{met wat}, and I link the morphemes they consist of to specific syntactic structures. Section \ref{sec:derivation} shows the derivation of \tit{waar mee} when all relevant features form a proper constituent and the derivation of \tit{met wat} when the features do not. Section \ref{sec:conclusion} concludes.

All examples in this paper are from spoken Dutch, unless indicated otherwise. Examples without a reference are constructed and have been verified by native speakers.


\section{\tsc{R}-pronouns are proper constituents}\label{sec:distribution}

The goal of this section is to show that \tsc{r}-pronouns and postpositions appear when all relevant features form a proper constituent, i.e. that all features form a constituent to the exclusion of any other features. I start the section by showing that \tit{r mee} `with it' and \tit{waar mee} `with what` are the default, which is to say that they normally appear (and \tit{met t} `with it' and \tit{met wat} `with what' do not). There is one exception, in which \tit{met wat} `with what` has to be used (and \tit{waar mee} `with what` cannot), which is when the relevant features do not form a proper constituent.


\subsection{\tsc{R}-pronouns as default}\label{sec:rdefault}

In what follows I discuss the distribution of \tsc{r}-pronouns and regular pronouns in general \citep{riemsdijk1978,koopman1994}. I start with the personal pronouns and then turn to the \tsc{wh}-pronouns.

Dutch has the accusative personal pronouns \tit{haar} `her', \tit{hem} `him' and \tit{t} `it' that can be used as animate and inanimate objects of verbs, as illustrated in \ref{ex:objverb}.

 \ex. \label{ex:objverb}
 \ag. Ik zie haar/hem.\\
  I see her/him\\
  `I see her/him.'\label{ex:aniobj}
 \bg. Ik zie t.\\
  I see it\\
  `I see it.'\label{ex:inaniobj}

Example \ref{ex:prepani} shows that for animate objects the same pronouns (\tit{haar} `her' and \tit{hem} `him') appear as objects of prepositions. Repeating from the introduction, the inanimate pronoun \tit{t} `it' cannot be used as an object of a preposition, shown in \ref{ex:prephet}. Instead, an \tsc{r}-pronoun appears, as is illustrated in \ref{ex:preper}. \ref{ex:erprep} shows that the \tsc{r}-pronoun must appear to the left of the adposition.

\ex. \label{ex:objprep}
\ag. Ik schilder samen met haar/hem.\\
 I paint together with her/him\\
 `I am painting together with her/him.'\label{ex:prepani}
\bg. *Ik schilder met t.\\
 I paint with it\\
 `I am painting with it.'\label{ex:prephet}
\bg. Ik schilder r mee.\\
 I paint there with\\
 `I am painting with it.'\label{ex:preper}
\bg. *Ik schilder mee r.\\
 I paint with there\\
 `I am painting with it.'\label{ex:erprep}

\tit{Met} is not the only preposition this occurs with. For example, \tit{op} `on' and \tit{in} `in' cannot combine with the inanimate pronoun \tit{t} `it' either, and the \tsc{r}-pronoun plus postposition are obligatorily used instead.

\ex.
\ag. Ik zit r op.\\
 I sit there on\\
 `I am sitting on it.
\bg. *Ik zit op t.\\
 I sit on it\\
 `I am sitting on it.

\ex.
 \ag. Ik zwem r in.\\
  I swim there in\\
  `I am swimming in it.'
 \bg. *Ik zwemt in t.\\
  I swim in it\\
  `I am swimming in it.'

The situation of the inanimate \tsc{wh}-pronouns resembles the one of the inanimate personal pronouns. \tit{Wat} `what' can function as the object of a verb (see \ref{ex:wat}), but not as the object of a preposition \ref{ex:metwat}. The \tsc{r}-pronoun plus postposition \tit{waar mee} `with what' appears instead, as shown in \ref{ex:waarmee}.\footnote{The sentence in \ref{ex:metwat} is unacceptable with neutral intonation. It becomes acceptable if \tit{wat} `what' is stressed, for example in a context in which the speaker is highly surprised about the choice of the object the hearer is painting with.}

\ex.
\ag. Wat zie jij?\\
 what see you\\
 `What do you see?'\label{ex:wat}
\bg. *Met wat schilder jij?\\
 with what paint you\\
 `What are you painting with?'\label{ex:metwat}
\bg. Waar mee schilder jij?\\
 where with paint you\\
 `What are you painting with?'\label{ex:waarmee}

\tit{Waar mee} `with what' and not \tit{met wat} `with what' does not only appear in \tsc{wh}-questions, but also in other contexts. \ref{ex:headed} gives an example of a headed relative, and \ref{ex:headless} shows a free relative in which both predicates combine with an instrumental object. The use of \tit{met wat} `with what' is ungrammatical in both contexts, and \tit{waar mee} `with what' is used instead.\footnote{
In this example, \tit{waar} and \tit{mee} are the two left-most elements of the relative clause. It is also possible for \tit{mee} to appear more to the right in the clause and for \tit{waar} to appear as the left-most element on its own.

\exg. Ik schilder waar jij ook mee schildert.\\
 I paint where you also with paint\\
 `I am painting with what you are painting with too.'\label{ex:meealong}

I return to this variant in Section \ref{sec:waarmee}.
}


\ex.\label{ex:headed}
\ag. Ik schilder met de kwast waar mee jij ook schildert.\\
 I paint with the brush where with you also paint\\
 `I am painting with the brush that you are painting with too.'
\bg. *Ik schilder met de kwast met wat jij ook schildert.\\
 I paint with the brush with what you also paint\\
 `I am painting with the brush that you are painting with too.'

 \ex.\label{ex:headless}
 \ag. Ik schilder waar mee jij ook schildert.\\
  I paint where with you also paint\\
  `I am painting with what you are painting with too.'
 \bg. *Ik schilder met wat jij ook schildert.\\
  I paint with what you also paint\\
  `I am painting with what you are painting with too.'

In sum, the inanimate pronouns \tit{t} `it' and \tit{wat} `what' do not combine with prepositions. They are substituted by the \tsc{r}-pronouns \tit{r} `there' and \tit{waar} `where' and combine with postpositions.


\subsection{When \tit{met wat} `with what' shows up}

This section starts by showing that \tit{met wat} `with what' has to be used instead of \tit{waar mee} `with what' under certain circumstances in mismatching free relatives. Later in this section I argue that under these circumstances the relevant features that need to be spelled out do not form a proper constituent.

A mismatching free relative is a free relative construction in which the two predicates (the one in the main clause and the one in the relative clause) require arguments with different cases. In other words, the case requirements of the two clauses do not match. First consider \ref{ex:gekochtwaarmee}. The predicate in the relative clause, \tit{schildert} `paint', combines with an instrumental object. The predicate in the main clause, \tit{gekocht} `bought' combines with an accusative object. The \tsc{r}-pronoun plus postposition \tit{waar mee} `with what' is used here, shown in \ref{ex:gekochtwaarmee}. The use of \tit{met wat} `with what' is ungrammatical in this context, illustrated in \ref{ex:gekochtmetwat}.

\ex.
\ag. Ik heb gekocht waar mee jij schildert.\\
 I have bought where with you paint\\
 `I bought what you are painting with.'\label{ex:gekochtwaarmee}
\bg. *Ik heb gekocht met wat jij schildert.\\
 I have bought with what you paint\\
 `I bought what you are painting with.'\label{ex:gekochtmetwat}

If the predicates are switched between the clauses, the \tsc{r}-pronoun plus postposition cannot appear anymore. In \ref{ex:schildermet}, \tit{schilder} `paint' combines with an instrumental object in the main clause and \tit{gekocht} `bought' combines with an accusative object in the relative clause.
The use of an \tsc{r}-pronoun plus postposition is ungrammatical, illustrated in \ref{ex:schilderwaarmee}. Instead, a combination of the instrumental preposition \tit{met} `with' and the \tsc{wh}-pronoun \tit{wat} `what' is used, shown in \ref{ex:schildermetwat}.

\ex.\label{ex:schildermet}
\ag. *Ik schilder waar mee jij hebt gekocht.\\
 I paint where with you have bought\\
 `I paint with what you bought.'\label{ex:schilderwaarmee}
\bg. Ik schilder met wat jij hebt gekocht.\\
 I paint with what you have bought\\
 `I paint with what you bought.'\label{ex:schildermetwat}

Table \ref{tbl:distribution} summarizes the pattern. When the main clause predicate combines with an accusative and the relative clause predicate with an instrumental, \tit{waar mee} is grammatical and \tit{met wat} is ungrammatical. When the main clause predicate combines with an instrumental and the relative clause predicate with an accusative, \tit{waar mee} is ungrammatical and \tit{met wat} appears instead.

\begin{table}[ht]
  \center
	\caption {Distribution between \tit{waar mee} and \tit{met wat}}
		\begin{tabular}{lll}
		\toprule
                              & \tit{waar mee} & \tit{met wat} \\
		\midrule
    m:\tsc{acc}, r:\tsc{ins}  & ✔             & *             \\
    m:\tsc{ins}, r:\tsc{acc}  & *             & ✔             \\
    \bottomrule
\end{tabular}
\label{tbl:distribution}
\end{table}

In the remainder of this section I argue that the crucial point of \ref{ex:schildermetwat} is that the instrumental object does not form a proper constituent, i.e. the instrumental object is not a constituent to the exclusion of any other elements. The other side of the coin is that constructions with \tsc{r}-pronouns and postpositions contain an instrumental object that does form a proper constituent.

Below I repeat the examples with instrumental objects discussed in this paper so far, except for the mismatching free relatives. I place square brackets around the relative clauses in \ref{ex:const3} and \ref{ex:const4}. In each of these examples the instrumental object forms a proper constituent at a certain point in the derivation.

\ex.
\ag. Ik schilder r mee.\\
 I paint there with\\
 `I am painting with it.'\label{ex:const1}
\bg. Waar mee schilder jij?\\
where with paint you with\\
 `What are you painting with?'\label{ex:const2}
\bg. Ik schilder met de kwast [waar mee jij ook schildert].\\
 I paint with the brush where with you also paint\\
 `I am painting with the brush that you are painting with too.'\label{ex:const3}
\bg. Ik schilder [waar mee jij ook schildert].\\
 I paint where with you also paint\\
 `I am painting with what you are painting with too.'\label{ex:const4}

Consider now the examples with the mismatching free relatives. The two predicates I used in the free relatives are \tit{kopen} `to buy' and \tit{schilderen} `to paint'. \tit{Kopen} `to buy' takes accusative objects, illustrated in \ref{ex:kopen}. \tit{Schilderen} `to paint' takes instrumental objects, shown in \ref{ex:schilderen}.\footnote{\tit{Schilderen} also optionally takes an (accusative) object, but I am focusing on the instrumental object here.}

\ex.
\ag. Ik koop het schilderij.\\
 I buy the painting\\
 `I am buying the painting.'\label{ex:kopen}
\bg. Ik schilder met een kwast.\\
 I paint with a brush\\
 `I am painting with a brush.'\label{ex:schilderen}

In \ref{ex:mismatchwaarmee}, I repeat the mismatching free relative in which \tit{waar mee} `with what' appears. The predicate \tit{schildert} `paints' combines in the relative clause with the instrumental object. I place square brackets around the relative clauses in \ref{ex:mismatchwaarmee}. The instrumental object forms a proper constituent within the relative clause, and it can be realized as the \tsc{r}-pronoun plus postposition \tit{waar mee} `with what'.

\exg. Ik heb gekocht [waar mee jij schildert].\\
 I have bought where with you paint\\
 `I bought what you are painting with.'\label{ex:mismatchwaarmee}

Consider now the other kind of mismatch. The relative clause predicate \tit{gekocht} `bought' combines with an accusative object. I place square brackets around the relative clauses in \ref{ex:mismatchmetwat}. The accusative \tsc{wh}-object of a verb is always \tit{wat} `what', as I have shown earlier in \ref{ex:wat}. The instrumental only comes into the picture in the main clause, when \tit{schilder} `paint' combines with an instrumental object. At no point in the derivation does the instrumental object form a proper constituent. \tit{Waar mee} `with what' cannot surface, and \tit{met wat} appears instead.

\exg. Ik schilder met [wat jij hebt gekocht].\\
 I paint with what you have bought\\
 `I paint with what you bought.'\label{ex:mismatchmetwat}

 The mismatching free relative in \ref{ex:mismatchmetwat} is not the only construction in which the string \tit{met wat} `with what' appears. I give examples of two more occurrences in \ref{ex:moremetwat}. In \ref{ex:watwasfur}, \tit{wat} `what' is the \tit{wat} in the so-called \tit{wat voor} `what for'-construction \citep[cf.][]{corver1991}.
 In \ref{ex:watindef}, \tit{wat} appears as a quantifier, and it means `some'. In both constructions \tit{wat} `what' takes a complement and \tit{met wat} `with what' is not a proper constituent. The brackets within the examples indicate the constituency.

 \ex.\label{ex:moremetwat}
 \ag. [Met [wat [voor [potloden]]] teken jij?\\
  with what for pencils draw you\\
  `What kind of pencils do you with?'\label{ex:watwasfur}
 \bg. Ik wil graag thee [met [wat [suiker]]].\\
  I want please tea with some sugar\\
  `I would like to have tea with some sugar.'\label{ex:watindef}

\ref{ex:summaryconst} summarizes the main results of this section. \tit{Met wat} can not surface when the features that correspond to \tit{met wat} form a proper constituent. It is always \tit{waar mee} that appears, as shown in \ref{ex:waar meefr}. \tit{Met wat} can appear when the instrumental object does not form a proper constituent.
This can be when \tit{wat} takes a complement before it combines with \tit{met}, as shown in \ref{ex:metwatx}, corresponding to the examples in \ref{ex:moremetwat}. The other option is that \tit{wat} is part of a clause that \tit{met} is not a part of, schematized as \ref{ex:metwatfr} and corresponding to \ref{ex:mismatchmetwat}.

\ex.\label{ex:summaryconst}
\a. [[met] [wat]] → [waar mee]\label{ex:waar meefr}
\b. [met [wat [X]]]\label{ex:metwatx}
\b. [met [[wat] [X]]]\label{ex:metwatfr}

In sum, \tit{waar mee} `with what' is the default, and \tit{met wat} `with what' only appears if the features that spell out an instrumental object do not form a proper constituent.

\section{Decomposing \tit{waar mee} and \tit{met wat}}\label{sec:takingapart}

In this section I decompose \tit{waar mee} `with what' and \tit{met wat} `with what' into morphemes. I connect these morphemes to parts of syntactic structure. \tit{Waar mee} and \tit{met wat} spell out the same set of features, but the distribution is different. I decompose \tit{waar mee} and \tit{met wat} as in \ref{ex:decompose}. Earlier works have decomposed these pronouns in similar ways (cf. \citealt{hachem2015,noonan2017dutch,wesseling2018}).

\ex.\label{ex:decompose}
\a. w-aa-r mee
\b. met w-a-t

In this section I first identify \tit{w} and \tit{a(a)} as morphemes that appear in both expressions. Putting these two aside, I then concentrate on \tit{r mee} and \tit{met t}.
I propose an account that makes the ungrammaticality of \tit{met t} and the appearance of \tit{r mee} follow from spellout. The analysis accounts for the following three observations, taking \tit{met t} as the point of departure. First, \tit{met} changes from being a preposition to being a postposition. This process is restricted to inanimate pronouns, and it does not apply to full noun phrases and animate pronouns. Second, \tit{met} changes form to \tit{mee}. Third, \tit{t} is replaced by \tit{r}, a morpheme that is associated with the locative in Dutch.
Along the way I introduce some necessary theoretical background on lexicalization in Nanosyntax.

\subsection{\tit{w} and \tit{a(a)}}

Let me start with the morphemes \tit{w} and \tit{a(a)} that appear in both \tit{waar mee} `with what' and \tit{met wat} `with what'. I assume that they correspond to the same syntactic structure in both expressions, As I am interested in the differences between the two expressions, I do not discuss the featural content of \tit{w} and \tit{a(a)} in depth.

For \tit{w}, I follow \posscitet{hachem2015} work on \tit{d} and \tit{w} elements in German and Dutch. In her work, \tit{d} establishes a definite reference and \tit{w} triggers the construction of a set of alternatives in the sense of \citet{rooth1992} (see \citealt{hachem2015} for discussion).\footnote{Throughout the paper, ⇔ indicates the pairing between a lexical tree and a phonological form in a lexical entry, and ⇒ indicates how a node in the syntactic structure is spelled out.}

\ex. \begin{forest}
[\tsc{wP}
    [W, roof]
]
{\draw (.east) node[right]{⇔ \tit{w}}; }
\end{forest}\label{ex:entryw}

I follow several authors (cf. \citealt{lander2016,noonan2017dutch,wesseling2018}) in the assumption that the morpheme \tit{a(a)} is related to deixis. Dutch distinguishes between proximal by using \tit{ie} (/i:/) and \tit{i} (/ɪ/) and distal by using \tit{aa} (/aː/) and \tit{a} (/ɑ/), illustrated in \ref{ex:deixis}.\footnote{
\tsc{Wh}-elements combine with the distal marker \tit{a(a)}, and they cannot with the proximal marker \tit{i/ie}. Conceptually, this can be understood if spatial deixis is connected to discourse deixis \citep[cf.][]{colasanti2019}. Proximal is spatially near the speaker and refers to knowledge that the speaker possesses. Distal is spatially away from the speaker and refers to knowledge that the speaker does not possess. In \tsc{wh}-pronouns, the speaker does not have the knowledge, so the distal is used and not the proximal.
}
I assume that the changes from from /ɪ/ into /i:/ and /ɑ/ into /aː/ is due to phonology, possibly as a result of the final /r/.\footnote{I leave the description of the exact phonological process for future research.}\footnote{
Note that in the proximal the initial \tit{d} is also replaced by an \tit{h}. This could indicate that (1) there must be some difference between \tit{hier} `here' and \tit{daar} `there' besides proximal and distal, or (2) there is a different distribution between \tit{h} plus \tit{ie} and \tit{d} plus \tit{i} and the difference in vowels is not a matter of phonology. As this is not the focus of this paper, I put this aside for now.
}

\ex.\label{ex:deixis}
\ag. h-ie-r\\
 here\\
\bg. d-aa-r\\
 there\\
\bg. d-i-t\\
 this\\
\bg. d-a-t\\
 that\\
 \z.
 \z.

For the purpose of this paper I let \tit{a(a)} correspond to \tsc{deixP}.

\ex. \begin{forest}
[\tsc{deixP}
    [\tsc{deix}, roof]
]
{\draw (.east) node[right]{⇔ \tit{a(a)}}; }
\end{forest}\label{ex:entrya}

I put \tit{w} and \tit{a(a)} aside for now, assuming they spell out the same syntactic structure in \tit{waar mee} `with what' and \tit{met wat} `with what'. This leaves \tit{r mee} `with it' and \tit{met t} `with it'.


\subsection{\tit{r mee} vs. \tit{met t}}

In this section I discuss the forms \tit{r mee} `with it' and \tit{met t} `with it`. I argue that both forms spell out the same features. Whether or not the features form a proper constituent determines which form appears.

\subsubsection{\tit{t} vs. \tit{r}}

In this section I give the lexical entries for \tit{t} and \tit{r}. I show that \tit{r} is the base form and \tit{t} a suppletive form that additionally encodes nominative, accusative or dative.

I start with the lexical entry for \tit{t}. The element \tit{t} `it' can be used as a subject (associated with nominative), direct object (associated with accusative) and indirect object (associated with dative), as shown in \ref{ex:tsubobj}.

\ex.\label{ex:tsubobj}
\ag. T staat in de hal.\\
 \tsc{3sg.n.nom} stands in the hallway\\
 `It is standing in the hallway.'\label{ex:tnoclitic}
\bg. Ik zie t.\\
 I see \tsc{3sg.n.acc}\\
 `I see it.'
\bg. Ik heb t een klap gegeven.\\
 I have \tsc{3sg.n.dat} a hit given\\
 `I gave it a hit.'

For case, I follow \citet{caha2009} in that case features are organized in a containment relation as in \ref{ex:casetree}. The higher, more complex cases contain the lower, less complex cases. For the purpose of this paper, I only show the case features relevant to my analysis.

 \ex. \label{ex:casetree}
 \begin{forest} boom
 [\tsc{insP}
     [\tsc{k4}]
     [\tsc{datP}
         [\tsc{k3}]
         [\tsc{accP}
             [\tsc{k2}]
             [\tsc{nomP}
                 [\tsc{k1}]
                 [DP
                     [\phantom{xxx},roof]
                 ]
             ]
         ]
     ]
 ]
 \end{forest}

 The morpheme \tit{t} can act as nominative, accusative and dative, so it spells out \tsc{k}1, \tsc{k}2 and \tsc{k}3.

 Following the distinctions from \citet{cardinaletti1996}, I assume \tit{t} `it' is a weak pronoun. It is not a clitic, because it can occur in sentence initial position, shown in \ref{ex:tnoclitic}. It is not a strong pronoun, because it cannot be coordinated, as indicated in  \ref{ex:tcoordinated}. \ref{ex:datcoordinated} shows that \tit{t} `it' needs to combine with \tit{da-}/\tit{di-} to be able to be coordinated.

 \ex.
 \ag. *Hij en t staan in de hoek.\\
  he and it stand in the corner\\
  `He and it are standing in the corner.'\label{ex:tcoordinated}
 \bg. Hij en dit/dat staan in de hoek.\\
  he and this/that stand in the corner\\
  `He and it are standing in the corner.'\label{ex:datcoordinated}

The feature Σ indicates that the pronoun is a weak pronoun.

In addition to the pronominal and case features, I assume that the \tit{t} contains the ontological category \tsc{thing} \citep{kayne2005}.  For reasons of space I will assume singular to be the absence of number and inanimate or neuter the absence of gender.

\ref{ex:entryt} shows the lexical entry for \tit{t} `it'.

\ex. \begin{forest} boom
 [\tsc{datP}
     [\tsc{k3}]
     [\tsc{accP}
         [\tsc{k2}]
         [\tsc{nomP}
             [\tsc{k1}]
             [ΣP
                 [Σ]
                 [\tsc{thingP}
                     [\tsc{thing}, roof]
                 ]
             ]
         ]
     ]
 ]
 {\draw (.east) node[right]{⇔ \tit{t}}; }
 \end{forest}\label{ex:entryt}

 This lexical entry can lexicalize the \tsc{datP}, but also the \tsc{accP} and \tsc{nomP}. This is due to the Superset Principle, given in \ref{ex:superset-principle}.

  \ex.\label{ex:superset-principle} The Superset Principle \citet{starke2009}: \\
  A lexically stored tree matches a syntactic node iff the lexically stored tree contains the syntactic node.

 In other words, a lexically stored tree does not have to be identical to the syntactic structure it spells out. The crucial requirement is that the syntactic structure is contained within the lexically stored tree. This has as a consequence that the lexical entry in \ref{ex:entryt} can also be inserted in \ref{ex:tacc} and \ref{ex:tnom}.

 \ex.
 \a. \begin{forest} boom
 [\tsc{accP}
     [\tsc{k2}]
     [\tsc{nomP}
         [\tsc{k1}]
         [ΣP
             [Σ]
             [\tsc{thingP}
                 [\tsc{thing}, roof]
             ]
         ]
     ]
 ]
 {\draw (.east) node[right]{⇒ \tit{t}}; }
 \end{forest}\label{ex:tacc}
 \b. \begin{forest} boom
 [\tsc{nomP}
     [\tsc{k1}]
     [ΣP
         [Σ]
         [\tsc{thingP}
             [\tsc{thing}, roof]
         ]
     ]
 ]
 {\draw (.east) node[right]{⇒ \tit{t}}; }
 \end{forest}\label{ex:tnom}

I continue with \tit{r}. The \tsc{r}-pronoun \tit{r} can be used as a locative.

 \exg. Ik ben r al geweest.\\
  I am there already been\\
  `I have already been there.'

I follow \citet{baunaz2018} in assuming that the ontological category \tsc{location} contains \tsc{thing}.\footnote{\citet{baunaz2018} place in addition \tsc{person} between \tsc{thing} and \tsc{location}, which I leave out here.} I give the lexical entry for \tit{r} `there' in \ref{ex:entryr}.

\ex. \begin{forest} boom
[\tsc{location}P
    [\tsc{location}]
    [\tsc{thingP}
        [\tsc{thing}, roof]
    ]
]
{\draw (.east) node[right]{⇔ \tit{r}}; }
\end{forest}\label{ex:entryr}

Notice here that, via the Superset Principle, \tit{r} can also realize the feature \tsc{thing}, as it is contained in \tsc{locationP}. Moreover, for a syntactic structure as \ref{ex:rthing} the lexical entry \ref{ex:entryr} will be inserted and not \ref{ex:entryt}.

\ex.
\begin{forest} boom
 [\tsc{thingP}
     [\tsc{thing}, roof]
 ]
{\draw (.east) node[right]{⇒ \tit{r}}; }
\end{forest}\label{ex:rthing}

The choice of \ref{ex:entryt} over \ref{ex:entryr} is due to the Elsewhere Condition, given in \ref{ex:elswhere-condition}.

\ex.\label{ex:elswhere-condition} The Elsewhere Condition (\citealt{kiparsky1973}, formulated as in \citealt{caha2020}):\\
When two entries can spell out a given node, the more specific entry wins. Under the Superset Principle governed insertion, the more specific entry is the one which has fewer unused features.

In other words, when two lexical entries are both candidates for spellout, the more specific one is inserted. The syntactic structure in \ref{ex:entryr} only has \tsc{location} as an unused feature, while in \ref{ex:entryt} all features from Σ up to \tsc{k3} remain unused.

Under this analysis, the caseless base form of the inanimate singular pronoun in Dutch is actually \tit{r}, and \tit{t} is a suppletive nominative, accusative and dative form. The base form only shows up in the higher cases, from the instrumental on. I show this in Table \ref{tbl:dutchcases}.

\begin{table}[ht]
	\center
	\caption {Fragment Dutch \tsc{n.sg}}
		\begin{tabular}{ll}
		\toprule
              & \tsc{n.sg} \\
		\midrule
    \tsc{nom} & t         \\
    \tsc{acc} & t         \\
    \tsc{dat} & t         \\
    \tsc{ins} & r-mee    \\
    \bottomrule
\end{tabular}
\label{tbl:dutchcases}
\end{table}

A similar pattern appears in Ossetic, shown in Table \ref{tbl:ossetic}. In the first person singular of this language, it is only the nominative that is suppletive: \tit{æz}. The higher cases have the stem \tit{mæn} and they combine with the suffixes that nouns normally combine with.

\begin{table}[ht]
	\center
	\caption {Partial paradigm of Ossetic 1\tsc{sg} and noun \citep{erschler2012}}
		\begin{tabular}{lll}
		\toprule
              & 1\tsc{sg}  & head    \\
		\midrule
    \tsc{nom} & æz          & sær-∅   \\
    \tsc{acc} & mæn-∅       & sær-∅   \\
    \tsc{dat} & mæn-æn      & sær-æn  \\
    \tsc{ins} & mæn-æj      & sær-æj  \\
    \bottomrule
\end{tabular}
\label{tbl:ossetic}
\end{table}

\citet{caha2019competition} uses evidence from a phenomenon called suspended affixation to argue that \tit{mæn} is a caseless stem and \tit{æz} is a suppletive nominative form. Consider the ordinary coordination of two full noun phrases in \ref{ex:horseoxboth}. Both conjuncts are marked by a plural marker and a case marker. Suspended affixation is shown in \ref{ex:horseoxone}. Here the case marker only appears on the second conjunct and not on the first one, without changing the interpretation.
\tit{Bæx-tæ} `horse-\tsc{pl}' in \ref{ex:horseoxone} does not carry any case marking here.

\ex.
\ag. bæx-t-imæ æmæ gæl-t-imæ\\
horse-\tsc{pl-com} and ox-\tsc{pl-com}\\\label{ex:horseoxboth}
\bg. bæx-tæ æmæ gæl-t-imæ\\
horse-\tsc{pl} and ox-\tsc{pl-com}\\
`with horses and oxen' \hfill (Ossetic, \citealt[165]{erschler2012} after \citealt{caha2019competition})\label{ex:horseoxone}

\ref{ex:suspendedme} gives examples of the first person singular in suspended affixation contexts. It shows that it is \tit{mæn} that appears as a caseless first conjunct and that the use of \tit{æz} is ungrammatical. This means that \tit{mæn} is the bare stem that combines with case markers, and \tit{æz} the suppletive nominative form.

\ex.\label{ex:suspendedme}
\ag. mæn æmæ Zauyr-æn\\
 1\tsc{sg} and Zaur-\tsc{dat}\\\label{ex:izaurboth}
\bg. *æz æmæ Zauyr-æn\\
 1\tsc{sg} and Zaur-\tsc{dat}\\
 `me and Zaur' \hfill (Ossetic, \citealt[39]{беляев2014} after \citealt{caha2019competition})\label{ex:izaurone}

The take-away message from the Ossetic example is that Dutch is not unique. Dutch and Ossetic both have a suppletive form that is less marked (in Dutch nominative, accusative and dative and in Ossetic nominative), and higher cases that are expressed by a combination a caseless base form and a suffix or postposition.


\subsubsection{\tit{mee} vs. \tit{met}}

The last two forms for which I specify lexical entries are \tit{mee} and \tit{met}. An important distinction between these two is that \tit{mee} appears after the element it combines with, while \tit{met} appears before the element it combines with. In other words, \tit{mee} is a postposition and \tit{met} is a preposition.\footnote{
A topic related to this paper is the different positioning of syncretic adpositions in Dutch (see \citealt{caha2010} for an account of German and Dutch and see \citealt{pretorius2017} for Afrikaans). In \ref{ex:dutchin}, \tit{in} `in' changes meaning depending on whether it proceeds or follows the noun phrase. In \ref{ex:dutch-in-loc}, \tit{in} `in' carries a locational meaning, and in \ref{ex:dutch-in-dir}, \tit{in} `in' carries a directional meaning.

\ex.\label{ex:dutchin}
\ag. Ik klim in de boom.\\
 I climb in the tree\\
 `I am climbing in the tree.'\label{ex:dutch-in-loc}
\bg. Ik klim de boom in.\\
 I climb the tree in\\
 `I am climbing into the tree.'\label{ex:dutch-in-dir}

\citet{caha2010} argues that the movement of the adposition is driven by movement, and it is meaningful. The movement of \tsc{r}-pronouns I discuss in this paper is driven by spellout, which is meaningless.}
In this section I discuss the difference between prepositions and postpositions, and how this is modeled with the case hierarchy in Nanosyntax \citep{caha2009}.

In the previous section I argued that \tit{t} realizes case features up to \tsc{k3} (see \ref{ex:entryt}). However, case can also be expressed by prepositions (or prefixes) and postpositions (or suffixes). The division between which cases are expressed by prepositions and which are expressed by postpositions is not arbitrary. If a particular case in the case sequence (see \ref{ex:casetree}) is expressed by a preposition, then all higher cases are too (the preposition/suffix hierarchy, \pgcitealt{caha2010}{36}). The lower cases are expressed by the noun phrase, either as a suffix or by the noun phrase itself.
The result of that is that a prepositional phrase can contain a preposition and a suffix. In the German example in \ref{ex:loffel}, the dative suffix is used with an instrumental preposition.

\exg. mit ein -em Löffel\\
with a \tsc{dat.m.sg} spoon\\
`with a spoon' \hfill (German)\label{ex:loffel}

The features up to the \tsc{datP} are realized as a suffix, and the features above \tsc{datP} are realized as a preposition (see \ref{ex:casetree}).

It varies between and within languages which cases are spelled out by prepositions and which ones are spelled out by suffixes. In other words, there is variation in how many case features a suffix can spell out, and how many it leaves for the preposition to spell out. An example of variation within a language comes from Bulgarian. \ref{ex:bulpron} shows that pronouns can take the suffix \tit{-i} to realize dative, and \ref{ex:buldpto} shows that full noun phrases (such as the proper name \tit{Kamen}) need a preposition \tit{na} `to'.

\ex.\label{ex:bulgarian}
\ag. Tazi duma m -i e nepoznata.\\
that word I -\tsc{acc/dat.cl} is unfamiliar\\
`That word is unfamiliar to me.'\label{ex:bulpron}
\bg. Tazi duma e nepoznata na Kamen.\\
that word is unfamiliar to Kamen\\
`That word is unfamiliar to Kamen.'\label{ex:buldpto} \hfill (Bulgarian, \citealt[39]{caha2009})

The example in \ref{ex:bulpron} indicates that the suffix \tit{-i} can spell out features up to the dative case. However, for some reason this suffix cannot combine with a full noun phrase to spell out these dative features. Instead, the preposition \tit{na} is used, as shown in \ref{ex:buldpto}.

In Dutch, the split between the use of a preposition and the use of a postposition is not between pronouns and full noun phrases but between inanimate pronouns and animate pronouns plus full noun phrases. In Dutch, inanimate pronouns combine with the postposition \tit{mee} and not with the preposition \tit{met} (compare \ref{ex:dutchina} and \ref{ex:dutchinmet}).

\ex.
\ag. Ik schilder r mee.\\
 I paint there with\\
 `I am painting with it.'\label{ex:dutchina}
\bg. *Ik schilder met t.\\
 I paint with it\\
 `I am painting with it.'\label{ex:dutchinmet}

Animate pronouns and full noun phrases, however, combine with the preposition \tit{met}, and the use of the postposition \tit{mee} is ungrammatical (compare \ref{ex:dutchan} and \ref{ex:dutchdp} to \ref{ex:dutchanmee} and \ref{ex:dutchdpmee}).

\ex.
\ag. Ik schilder samen met de kunstenares.\\
 I paint together with the artist\\
 `I am painting together with the artist.'\label{ex:dutchan}
\bg. Ik schilder samen met de kunstenares.\\
 I paint together with the artist\\
 `I am painting together with the artist.'\label{ex:dutchdp}
\bg. *Ik schilder samen de kunstenares mee.\\
 I paint together the artist with\\
 `I am painting together with the artist.'\label{ex:dutchanmee}
\bg. *Ik schilder samen de kunstenares mee.\\
 I paint together the artist with\\
 `I am painting together with the artist.'\label{ex:dutchdpmee}

The example in \ref{ex:dutchina} indicates that the postposition \tit{mee} can spell out features up to the instrumental case when there is an inanimate pronoun. However, for some reason this postposition cannot combine with an animate pronoun or a full noun phrase to spell out these dative features, as shown in \ref{ex:dutchanmee} and \ref{ex:dutchdpmee}. Instead, the preposition \tit{met} is used, as shown in \ref{ex:dutchan} and \ref{ex:dutchdp}.
In what follows, I give lexical entries that are in line with these observations.

I start by giving the lexical entry for \tit{mee}. Two facts need to be captured: \tit{mee} combines with \tit{r} and it is a postposition. The combination of \tit{r} and \tit{mee} expresses an inanimate instrumental.
This means that the combination of \tit{r} and \tit{mee} has to realize \tsc{thing}, Σ and all case features up to \tsc{k4}. So far, the \tsc{r}-pronoun \tit{r} only realizes the feature \tsc{thing}. This leaves Σ and \tsc{k1} to \tsc{k4} to be realized as \tit{mee}.

This brings me to the second point: \tit{mee} is a postposition. Nanosyntax distinguishes pre-elements (so prepositions and prefixes) from post-elements (so postpositions and suffixes) by the shape of their lexical entry \citep{starke2018}. Doing so, it follows from the spellout procedure whether an element appears before or after the previously inserted element. Post-elements have a unary bottom (i.e. the foot of the tree is a single feature), so they can only appear as the result of movement of their complement. Pre-elements have a binary bottom (i.e. the foot of the tree consists of two features), so they cannot be the inserted after complement movement. In Section \ref{sec:spellout} I come back to the complement movement I am referring to, and I illustrate it in a derivation. I give the lexical tree of \tit{mee} `with' in \ref{ex:entrymee}.

\ex. \begin{forest} boom
[\tsc{insP}
    [\tsc{k4}]
    [\tsc{datP}
        [\tsc{k3}]
        [\tsc{accP}
            [\tsc{k2}]
            [\tsc{nomP}
                [\tsc{k1}]
                [ΣP
                    [Σ]
                ]
            ]
        ]
    ]
]
{\draw (.east) node[right]{⇔ \tit{mee}}; }
\end{forest}\label{ex:entrymee}

Throughout the paper I have been mentioning that \tit{mee} cannot combine with animate pronouns and full noun phrases. This is not surprising with a lexical entry as the one in \ref{ex:entrymee}. Animate pronouns contain, in addition to the features of inanimate pronouns, gender (or animacy) features. I assume that gender features are situated between pronominal features (so Σ) and case features. The syntactic structure of an animate instrumental contains gender features that the lexical entry for \tit{mee} does not contain, so \tit{mee} cannot be inserted and a different lexical entries will be chosen (namely an alternative to \tit{r} that spells out gender features and the preposition \tit{met}).
Full nouns phrases do not combine with \tit{mee} because its lexical entry contains the pronominal feature Σ. Full noun phrases generally do not take features related to pronominal strength. According to the Superset Principle, a lexical tree can match a syntactic tree with a subpart of the features, but a tree can only shrink from the top, so \tit{mee} will always realize the feature Σ.

I just showed that \tit{mee} is a postposition that follows \tit{r}. This follows from the idea that its lexical tree has a unary bottom. \tit{Met} is a preposition, meaning it precedes \tit{t}, and it should be stored with a binary bottom. The highest case feature that \tit{t} can realize is \tsc{k3}, so the preposition must correspond to the feature that remains: \tsc{k4}. However, since the lexical tree should be stored with a binary bottom, the preposition must correspond to an additional feature. This cannot be just any feature, but it needs to be the feature that precedes it in the functional sequence, which is \tsc{f}3.\footnote{
There are different proposals regarding this overlap of features. \citet{caha2019competition} argues that there can only be a single feature overlap, \citet{starke2018} claims that the overlap can be more than one feature, and according to \citet{de2018}, there cannot be
any overlap at all. The proposal in this paper is compatible with the first two ideas.}
I give the lexical entry for \tit{met} `with' in \ref{ex:entrymet}.

\ex. \begin{forest} boom
[\tsc{insP}
    [\tsc{k4}]
    [\tsc{k3}]
]
{\draw (.east) node[right]{⇔ \tit{met}}; }
\end{forest}\label{ex:entrymet}

In the next section I put all features back together in a derivation and I show how \tit{waar mee} `with what' and not \tit{met wat} `with what' surfaces when all features form a constituent.


\section{In a derivation}\label{sec:derivation}

In this section I show derivations in which \tit{waar mee} appears when all features form a proper constituent and that \tit{met wat} appears when the features do not. Before I do that, I need to make some assumptions about the spellout procedure in Nanosyntax explicit. For a more complete  I do so in Section \ref{sec:spellout}. Section \ref{sec:waarmee} derives \tit{waarmee}, and Section \ref{sec:metwat} derives \tit{met wat}.

\subsection{The spellout procedure}\label{sec:spellout}

Spellout happens in a cyclic derivation, following a spellout algorithm \citep{starke2018}. After each instance of merge, spellout takes place. If no spellout exists for the phrase created by the newly added feature, evacuation movements specified in the spellout algorithm take place. The algorithm is given in \ref{ex:spellout}.

\ex. Merge F and \label{ex:spellout}
 \a. Spell out FP.
 \b. If (a) fails, attempt movement of the specifier of the complement of \tsc{f}, and retry (a).
 \b. If (b) fails, move the complement of \tsc{f}, and retry (a).

When a new match is found, it overrides previous spellouts, as described in \ref{ex:override}.

\ex. Cyclic Override \citep{starke2018}:\\
Lexicalisation at a node XP overrides any previous match at a phrase contained in XP.\label{ex:override}

If the spellout procedure in \ref{ex:spellout} fails, backtracking takes place. This is described in \ref{ex:backtracking}. In my analysis, backtracking is the operation that ultimately leads from the suppletive nominative, accusative and dative form \tit{t} to the base form \tit{r}.

\ex. Backtracking \citep{starke2018}:\\
When spellout fails, go back tothe previous cycle, and try the next option for that cycle.\label{ex:backtracking}

If backtracking also does not help, a specifier is constructed, as described in \ref{ex:specformation}. This is what happens when the preposition \tit{met} is inserted.

\ex. Spec Formation \citep{starke2018}:\\
If Merge F has failed to spell out (even after backtracking), try to spawn a new derivation providing the feature F and merge that with the current derivation, projecting the feature F at the top node.\label{ex:specformation}

With this theoretical background in place, I can turn to the derivation.


\subsection{Deriving \tit{waarmee}}\label{sec:waarmee}

I first show how \tit{r mee} is constructed. I leave out \tit{w} and \tit{a(a)} in order to not unnecessarily complicate the story.\footnote{I assume that the \tsc{wP} and \tsc{deixP} appear lower in the structure than the case features, so the functional sequence is as given in \ref{ex:fseq}.

\ex. [ [ [ [ [ [ [ \tsc{thing} ] \tsc{deix} ] W ] \tsc{k1} ] \tsc{k2} ] \tsc{k3} ] \tsc{k4} ]\label{ex:fseq}
\z.

\phantom{x}

}

I start with \tsc{thing}. The two candidates here are \ref{ex:entryt} and \ref{ex:entryr}. Following the Elsewhere Condition, \ref{ex:entryr} wins the competition because it contains less unused material.

\ex.
\begin{forest} boom
 [\tsc{thingP}
     [\tsc{thing}, roof]
 ]
{\draw (.east) node[right]{⇒ \tit{r}}; }
\end{forest}\label{ex:thingspellout}

In the next step, Σ is merged. \ref{ex:entryr} is no longer a candidate because it does not contain Σ. \ref{ex:entryt} still is a candidate, because it contains all features in \ref{ex:thingf1}. The spellout is overridden and the structure is realized as \tit{t}.

\ex. \begin{forest} boom
[ΣP
   [Σ]
   [\tsc{thingP}
       [\tsc{thing}, roof]
   ]
]
{\draw (.east) node[right]{⇒ \tit{t}}; }
\end{forest}\label{ex:thingf1}

Then \tsc{k1} is merged. This structure can still be realized as \tit{t}.

\ex. \begin{forest} boom
[\tsc{nomP}
   [\tsc{k1}]
   [ΣP
       [Σ]
       [\tsc{thingP}
           [\tsc{thing}, roof]
       ]
   ]
]
{\draw (.east) node[right]{⇒ \tit{t}}; }
\end{forest}

The same holds for the next two steps in which \tsc{k2} and \tsc{k3} are merged: the structure can still be spelled out as \tit{t} due to the Superset Principle.

\ex. \begin{forest} boom
[\tsc{datP}
    [\tsc{k3}]
    [\tsc{accP}
       [\tsc{k2}]
       [\tsc{nomP}
           [\tsc{k1}]
           [ΣP
               [Σ]
               [\tsc{thingP}
                   [\tsc{thing}, roof]
               ]
           ]
       ]
    ]
]
{\draw (.east) node[right]{⇒ \tit{t}}; }
\end{forest}

Then \tsc{k4} is merged, as shown in \ref{ex:f4no}. \ref{ex:entryt} can no longer spell out the structure, because it does not contain \tsc{k4}. There is no other candidate to spell out the structure as it is.

\ex. \begin{forest} boom
[\tsc{insP}
    [\tsc{k4}]
    [\tsc{datP}
        [\tsc{k3}]
        [\tsc{accP}
           [\tsc{k2}]
           [\tsc{nomP}
               [\tsc{k1}]
               [ΣP
                   [Σ]
                   [\tsc{thingP}
                       [\tsc{thing}, roof]
                   ]
               ]
           ]
        ]
    ]
]
{\draw (.east) node[right]{⇒ }; }
\end{forest}\label{ex:f4no}

According to the spellout algorithm in \ref{ex:spellout}, the next step is to move the specifier of the complement of \tsc{k4}. However, there is no specifier in \ref{ex:f4no}, so this does not apply. The second movement option is complement movement. The complement of \tsc{k4} moves to the specifier of \tsc{insP}, resulting in the structure in \ref{ex:f4comp}. The lexicon does not contain an entry with \tsc{insP} which contains only \tsc{k4}.\footnote{\tit{Met} is not a candidate, because the syntactic structure has a unary bottom whereas the lexical structure of \tit{met} has a binary bottom.}

\ex. \begin{forest} boom
[\tsc{datP}
    [\tsc{datP}
        [\tsc{k3}]
        [\tsc{accP}
           [\tsc{k2}]
           [\tsc{nomP}
               [\tsc{k1}]
               [ΣP
                   [Σ]
                   [\tsc{thingP}
                       [\tsc{thing}, roof]
                   ]
               ]
           ]
        ]
    ]
    {\draw (.east) node[right]{⇒ \tit{t}}; }
    [\tsc{insP}
        [\tsc{k4}]
    ]
    {\draw (.east) node[right]{⇒ }; }
]
\end{forest}\label{ex:f4comp}

As I mentioned in Section \ref{sec:spellout}, the following operation that is triggered is backtracking (see \ref{ex:backtracking}). This means that the derivation goes back to the previous cycle, and the next option for that cycle is tried. In this case, the previous cycle is the one in which \tsc{k3} is merged. The next option for that cycle is spec-to-spec movement. As there is no specifier, this does not apply. The option after that is complement movement, shown in \ref{ex:f3comp}. However, there is no match in the lexicon for a \tsc{datP} that contains only \tsc{k3}.

\ex. \begin{forest} boom
[\tsc{accP}
    [\tsc{accP}
       [\tsc{k2}]
       [\tsc{nomP}
           [\tsc{k1}]
           [ΣP
               [Σ]
               [\tsc{thingP}
                   [\tsc{thing}, roof]
               ]
           ]
       ]
    ]
    {\draw (.east) node[right]{⇒ \tit{t}}; }
    [\tsc{datP}
        [\tsc{k3}]
    ]
    {\draw (.east) node[right]{⇒ }; }
]
\end{forest}\label{ex:f3comp}

This means that backtracking proceeds further, into the cycle in which \tsc{k2} is merged. Again, spec-to-spec movement does not apply because there is no specifier, and complement movement can be tried, but there is no appropriate lexical entry available for the remaining structure. The same holds for the cycle in which \tsc{k1} is merged.

The situation changes when the derivation comes to the cycle in which Σ is merged. At this stage, \tsc{thing} is realized as \tit{r}. Again, there is no specifier, so spec-to-spec movement does not apply. However, complement movement provides a structure that is a match for the lexical entry in \ref{ex:entrymee}: \tit{mee}.\footnote{This picture resembles \posscitet{abels2003diss} analysis of \tsc{r}-pronouns in that not the whole complement of the preposition is moved but only a part is subextracted. The current analysis differs in that the movement is not syntactically driven but spellout-driven.
}

\ex. \begin{forest} boom
[ΣP
   [\tsc{thingP}
       [\tsc{thing}, roof]
   ]
   {\draw (.east) node[right]{⇒ \tit{r}}; }
   [ΣP
       [Σ]
   ]
   {\draw (.east) node[right]{⇒ \tit{mee}}; }
]
\end{forest}

From this point on the previously unmerged features are merged again one by one. First, \tsc{k1} is merged, shown in \ref{ex:f1again}. No match exists for this syntactic structure.

\ex. \begin{forest} boom
[\tsc{nomP}
    [\tsc{k1}]
    [ΣP
       [\tsc{thingP}
           [\tsc{thing}, roof]
       ]
       [ΣP
           [Σ]
       ]
    ]
]
{\draw (.east) node[right]{⇒}; }
\end{forest}\label{ex:f1again}

Following the spellout algorithm, the next step is spec-to-spec movement. The result is shown in \ref{ex:f1spec}. \tsc{k1} can be spelled out together with Σ as \tit{mee}.

\ex. \begin{forest} boom
[\tsc{nomP}
   [\tsc{thingP}
       [\tsc{thing}, roof]
   ]
   {\draw (.east) node[right]{⇒ \tit{r}}; }
   [\tsc{nomP}
       [\tsc{k1}]
       [ΣP
           [Σ]
       ]
   ]
   {\draw (.east) node[right]{⇒ \tit{mee}}; }
 ]
\end{forest}\label{ex:f1spec}

The same happens for \tsc{k2}, \tsc{k3} and \tsc{k4}. The features are merged one at a time, there is no spellout immediately after merging the feature, but there is a spellout after spec-to-spec movement. I show the situation after \tsc{k4} in \ref{ex:spelloutrmee}.

\ex. \begin{forest} boom
[\tsc{insP}
    [\tsc{thingP}
       [\tsc{thing}, roof]
    ]
    {\draw (.east) node[right]{⇒ \tit{r}}; }
    [\tsc{insP}
       [\tsc{k4}]
       [\tsc{datP}
           [\tsc{k3}]
           [\tsc{accP}
               [\tsc{k2}]
               [\tsc{nomP}
                   [\tsc{k1}]
                   [ΣP
                       [Σ]
                   ]
               ]
           ]
       ]
    ]
    {\draw (.east) node[right]{⇒ \tit{mee}}; }
]
\end{forest}\label{ex:spelloutrmee}

I skip over the details of how \tit{w} and \tit{a(a)} end up in their positions.\footnote{
I assume that \tsc{wP} and \tsc{deixP} are both complex specifiers that are created after \tsc{thing} is spelled out in \ref{ex:thingspellout}. After each instance of merge after that, backtracking takes place, the complex specifiers are detached from the structure, and the new feature is merged in all workspaces (see Multiple Merge, \pgcitealt{caha2019competition}{227}). The new feature is successfully spelled out in the workspace that contains the feature \tsc{thing}, and the workspaces are merged back together.
} The final result of the structure for \tit{waar mee} appears in \ref{ex:spelloutwaarmee}.

\ex. \begin{forest} boom
[\tsc{\tsc{wP}}
    [\tsc{\tsc{wP}}
        [\tsc{w}, roof]
    ]
    {\draw (.east) node[right]{⇒ \tit{w}}; }
    [\tsc{deixP}
        [\tsc{deixP}
            [\tsc{deix}, roof]
        ]
        {\draw (.east) node[right]{⇒ \tit{a(a)}}; }
        [\tsc{insP}
            [\tsc{thingP}
               [\tsc{thing}, roof]
            ]
            {\draw (.east) node[right]{⇒ \tit{r}}; }
            [\tsc{insP}
               [\tsc{k4}]
               [\tsc{datP}
                   [\tsc{k3}]
                   [\tsc{accP}
                       [\tsc{k2}]
                       [\tsc{nomP}
                           [\tsc{k1}]
                           [ΣP
                               [Σ]
                           ]
                       ]
                   ]
               ]
            ]
            {\draw (.east) node[right]{⇒ \tit{mee}}; }
        ]
    ]
]
\end{forest}\label{ex:spelloutwaarmee}

A consequence of analyzing \tit{mee} as a postposition is that \tit{r} and \tit{mee} always form a constituent to the exclusion of \tit{w} and \tit{a(a)}. At first sight this seems problematic, because it is possible for \tit{waar} `where' to not directly precede \tit{mee}. I repeat the relevant example in \ref{ex:meestranded}.

\exg. Ik heb gekocht waar jij mee schildert.\\
 I have bought waar you with paint\\
 `I bought what you are painting with.'\label{ex:meestranded}

There is no constituent in \ref{ex:spelloutwaarmee} that contains \tit{waar} but not \tit{mee}, so it is not possible to move \tit{waar} from \ref{ex:spelloutwaarmee} and strand \tit{mee} in the process. To resolve this situation I follow \citet{noonan2017dutch} in assuming that the phrase containing the adposition (\tit{mee}) can syntactically move to a position higher in the structure.

The movement of the adposition has a distribution typical of verbal particles (cf. \citealt{riemsdijk1978,noonan2017dutch}). The trigger for the movement of \tit{mee} could be the feature Σ (which is contained in \tit{mee}), associated with weak pronouns. Allowing \tit{mee} to subextract from the \tsc{w}P resolves the issue of the surface order in \ref{ex:meestranded}. With \tit{mee} having moved out, the remnant \tsc{wP} only contains features that are realized as \tit{waar}, and it can move to the left edge of the clause, resulting in the surface order in \ref{ex:meestranded}.

In this section I showed how \tit{waar mee} `with what' is derived if all syntactic features form a proper constituent. In the following section I address how \tit{waar mee} `with what' is blocked and \tit{met wat} `with what' appears when the features do not.


\subsection{Deriving \tit{met wat}}\label{sec:metwat}

An example of a situation in which the relevant features do not form a constituent and \tit{met wat} appears is given in \ref{ex:noconst}.

\exg. Ik schilder met wat jij hebt gekocht.\\
 I paint with what you have bought\\
 `I paint with what you bought.'\label{ex:noconst}

\ref{ex:metwatst} shows the syntactic structure of the relative clause \tit{wat jij hebt gekocht} `what you bought' in \ref{ex:noconst}. \tit{Wat} `what' is the element in the specifier of the CP.\footnote{
I added the feature \tsc{k3} that creates a \tsc{dat}P in the relative clause. It is unclear why, but syncretic forms seem to resolve case conflicts in free relatives and related phenomena (cf. \citealt{groos1981,pullum1986,ingria1990}). It seems that the syntax only sees the phonological form (in this case \tit{t}) and it is irrelevant whether the feature \tsc{k3} is present in the relative clause or not. Only considering the relative clause in \ref{ex:noconst}, we do not expect \tsc{k}3 to be present in the relative clause. However, also taking the instrumental case requirement from the main clause into account, we expect the feature to be present because the feature \tsc{k4} can only be merged if the feature \tsc{k3} has been.
}

\ex. \begin{forest} boom
[CP
    [\tsc{\tsc{wP}}
        [\tsc{\tsc{wP}}
            [\tsc{w}, roof]
        ]
        {\draw (.east) node[right]{⇒ \tit{w}}; }
        [\tsc{deixP}
            [\tsc{deixP}
                [\tsc{deix}, roof]
            ]
            {\draw (.east) node[right]{⇒ \tit{a(a)}}; }
            [\tsc{dat}P
                [\tsc{k3}]
                [\tsc{accP}
                   [\tsc{k2}]
                   [\tsc{nomP}
                       [\tsc{k1}]
                       [ΣP
                           [Σ]
                           [\tsc{thingP}
                               [\tsc{thing}, roof]
                           ]
                       ]
                    ]
                ]
            ]
            {\draw (.east) node[right]{⇒ \tit{t}}; }
        ]
    ]
    [CP
        [\tit{jij hebt gekocht},roof]
    ]
]
\end{forest}\label{ex:metwatst}

At this point \tsc{k4} is merged, as shown in \ref{ex:f4metwat}. Because of the presence of the CP, backtracking cannot take place and the additional features cannot be spelled out as the postposition \tit{mee}.\footnote{
Here I assume that backtracking is prohibited once syntactic movement has taken place (in this case \tsc{wh}-movement of the \tsc{wh}-element to the specCP position). I am not aware of any research about the interaction between backtracking and syntactic movement. Future research on the interaction between the two will show whether my assumption holds.
}

\ex. \begin{forest} boom
[\tsc{insP}
    [\tsc{k4}]
    [CP
        [\tsc{\tsc{wP}}
            [\tsc{\tsc{wP}}
                [\tsc{w}, roof]
            ]
            {\draw (.east) node[right]{⇒ \tit{w}}; }
            [\tsc{deixP}
                [\tsc{deixP}
                    [\tsc{deix}, roof]
                ]
                {\draw (.east) node[right]{⇒ \tit{a(a)}}; }
                [\tsc{accP}
                   [\tsc{k2}]
                   [\tsc{nomP}
                       [\tsc{k1}]
                       [ΣP
                           [Σ]
                           [\tsc{thingP}
                               [\tsc{thing}, roof]
                           ]
                       ]
                   ]
                ]
                {\draw (.east) node[right]{⇒ \tit{t}}; }
            ]
        ]
        [CP
            [\tit{jij hebt gekocht},roof]
        ]
    ]
]
\end{forest}\label{ex:f4metwat}

Instead, the last resort possibility to spell out features is set in motion: a complex specifier is created, as described in \ref{ex:specformation}. The newly merged feature merges with the previous feature in the functional sequence (\tsc{k}3), and together they form a complex specifier. This is illustrated in \ref{ex:metmetwat}.

\ex. \begin{forest} boom
[\tsc{insP}
    [\tsc{insP}
        [\tsc{k4}]
        [\tsc{k3}]
    ]
    {\draw (.east) node[right]{⇒ \tit{met}}; }
    [CP
        [\tsc{\tsc{wP}}
            [\tsc{\tsc{wP}}
                [\tsc{w}, roof]
            ]
            {\draw (.east) node[right]{⇒ \tit{w}}; }
            [\tsc{deixP}
                [\tsc{deixP}
                    [\tsc{deix}, roof]
                ]
                {\draw (.east) node[right]{⇒ \tit{a(a)}}; }
                [\tsc{dat}P
                    [\tsc{k3}]
                    [\tsc{accP}
                       [\tsc{k2}]
                       [\tsc{nomP}
                           [\tsc{k1}]
                           [ΣP
                               [Σ]
                               [\tsc{thingP}
                                   [\tsc{thing}, roof]
                               ]
                           ]
                        ]
                    ]
                ]
                {\draw (.east) node[right]{⇒ \tit{t}}; }
            ]
        ]
        [CP
            [\tit{jij hebt gekocht},roof]
        ]
    ]
]
\end{forest}\label{ex:metmetwat}

In sum, \tit{met wat} appears when the relevant features that form an instrumental inanimate relative pronoun do not form a proper constituent.



\section{Conclusion and discussion}\label{sec:conclusion}

In this paper, I discussed the instrumental \tsc{r}-pronoun plus postposition \tit{waar mee} `with what' in Dutch. The main intuition is that this form appears when features of an instrumental inanimate object form a proper constituent. The form cannot appear, and \tit{met wat} `with what` surfaces instead, when the relevant features do not form a proper constituent. The described pattern follows from a core assumption in Nanosyntax: only constituents can be targeted for spellout \citep{starke2009}.

Under this analysis, there is no need for a filter that rules out a combination of a preposition plus an inanimate pronoun. As long as the relevant features form a proper constituent, the spellout algorithm ensures that the \tsc{r}-pronoun plus postposition takes precedence over the preposition plus inanimate pronoun. The fact that locatives are syncretic with \tsc{r}-pronouns is not a coincidence either. The lexicon contains a single entry that can spell out the features that refer to a locative and features that refer to a thing.
The change in placement of the adposition is a consequence of spellout too. \tit{Mee} is a postposition and is stored besides and shaped differently from the preposition \tit{met}, which leads them to either appear before or after the previously spelled out element. The fact that \tit{met} and \tit{mee} differ phonologically can also be captured by storing them as two separate lexical entries.

Giving \tit{met} and \tit{mee} two distinct lexical entries has as the consequence that the phonological overlap between them seems like a coincidence. This can be questioned, because there is only one other preposition that changes form when it appears as a postposition. This preposition is \tit{tot} `to', which changes to \tit{toe} as a postposition. It has in common with \tit{met} that it is the only preposition in Dutch with the phonological structure CVt. For now I take the phonological resemblance to be a relic from the past without having any influence on the synchronic data. Other Dutch adpositions do not change form when they combine with an \tsc{r}-pronoun, e.g. \tit{in} `in'. This raises the question of whether the preposition \tit{in} `in' and the postposition \tit{in} `in' are stored as separate lexical entries in the lexicon or whether they refer to a single entry. Connected to this are the verbal particles that appear in Dutch. They have the form of the postposition and not of the preposition: it is \tit{mee-lopen} `walk along' and not *\tit{met-lopen} `walk along'. Ideally, an analysis derives all these observations.
For now, I leave these matters for future research.

\printbibliography

\section*{Acknowledgements}

I am grateful for the feedback of audiences at On the place of case in grammar, Universität Leizpig, Georg-August-Universität Göttingen and Goethe-Universität Frankfurt, Pavel Caha, Hagen Blix, Karen De Clercq, Francesco Pinzin, Michal Starke and Guido Vanden Wyngaerd. I also wish to thank two anonymous reviewers for their constructive and helpful comments.

\end{document}
