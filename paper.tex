\documentclass{article}

\usepackage{fenna-files/packages}
\usepackage{fenna-files/commands}
\bibliography{fenna-files/references}{}

\title{Dutch, PPs, ʀ-pronouns}
\author{Fenna Bergsma}
\date{\today}

\begin{document}


\maketitle






\section{Introduction}

The term \tsc{r}-pronoun \citep{riemsdijk1978} refers to a set of nominal elements that can strand prepositions in Dutch (and German). It is special because Dutch is a normally non-preposition stranding language. \tsc{R}-pronouns morphologically consist of a combination of a preposition and locative pronoun. In the example \ref{ex:klimerop} \tit{mee} `with' is the preposition and \tit{er} `there' is the locative pronoun.

\exg. Ik klim er-op.\\
 I climb there-on\\
 `I am climbing on it.'\label{ex:klimerop}

This paper focuses on a single \tsc{r}-pronoun in a specific type of free relative construction. The \tsc{r}-pronoun I focus on is the relative pronoun \tit{waarmee} `with what', which is interesting for two reasons. First, just like for all \tsc{r}-pronouns, the wh-element is the locative, but there is no meaning component related to location in `with what'. Second, the preposition \tit{met} `with' changes into \tit{mee} `with when it is combined with an \tsc{r}-pronoun.

The construction I focus on is a mismatching free relative. This is a free relative construction in which the two predicates (the one in the main clause and the one in the embedded clause) combine with two different cases (i.e. the case requirements do not match). I illustrate this in \ref{ex:gekochtwaarmee}. The predicate in the embedded clause, \tit{schildert} `paint', combines with an instrumental PP. The predicate in the main clause clause, \tit{gekocht} `bought' combines with an accusative DP. The \tsc{r}-pronoun \tit{waarmee} `with what' is used here.
\footnote{In this example \tit{mee} `with' is stranded and \tit{waar} `where' is moved to the left edge of the embedded clause. It is also possible for \tit{waar} `where' to bring \tit{mee} `with' along, as in \ref{ex:meealong} but it is regarded as slightly less natural.

\exg. Ik heb gekocht waar jij mee schildert.\\
 I have bought where you with paint\\
 `I bought what you are painting with.'\label{ex:meealong}
\z.

\phantom{x}
}

\exg. Ik heb gekocht waar jij mee schildert.\\
 I have bought where you with paint\\
 `I bought what you are painting with.'\label{ex:gekochtwaarmee}

If one were to switch around the predicates between the clauses, the \tsc{r}-pronoun does not appear anymore. In \ref{ex:schildermet}, \tit{schilder} `paint' combines with an instrumental PP in the main clause and \tit{gekocht} `bought' combines with an accusative DP in the embedded clause. The use of an \tsc{r}-pronoun is ungrammatical, as indicated by the ungrammaticality of \ref{ex:schilderwaarmee}. Instead, a combination of the regular instrumental preposition \tit{met} `with' and the regular wh-pronoun \tit{was} `what' in used.

\ex.\label{ex:schildermet}
\ag. *Ik schilder waarmee jij hebt gekocht.\\
 I paint {where with} you have bought\\
 `I paint with what you bought.'\label{ex:schilderwaarmee}
\bg. Ik schilder met wat jij hebt gekocht.\\
 I paint with what you have bought\\
 `I paint with what you bought.'\label{ex:schildermetwat}

The use of \tit{met wat} `with what' is ungrammatical in the context in which \tit{waarmee} `with what' appeared in \ref{ex:gekochtwaarmee}. This is illustrated in \ref{ex:gekochtmetwat}.

\exg. *Ik heb gekocht met wat jij schildert.\\
 I have bought with what you paint\\
 `I bought what you are painting with.'\label{ex:gekochtmetwat}

In this paper I show that distribution of \tit{waarmee} `with what' and \tit{met wat} `with what' in these free relative constructions gives us a unique insight into the internal structure of \tsc{r}-pronouns. In what follows I show that \tsc{r}-pronouns and regular preposition compete to spell out the same syntactic features. If all features form a constituent, the \tsc{r}-pronoun surfaces. If the constituent is interrupted, the preposition-pronoun combination shows up. This straightforwardly follows in a system in which spellout targets phrasal constituents: Nanosyntax \citep{starke2009}.

This paper is structured as follows. First, I show that it really is constituency. Then I decompose \tsc{r}-pronouns, prepositions and regular pronouns. Last, I show in derivations that constituency connects to the choice for \tsc{r}-pronoun or preposition and regular pronoun. Unmarked examples are constructed and have been verified by native speakers.


\section{The distribution between \tit{waarmee} and \tit{met wat}}

In the introduction I discussed the distribution between \tit{waarmee} `with what' and \tit{met wat} `with what' in free relatives with predicates that combine with different cases. Table \ref{tbl:distribution} repeats the generalization. When the main clause predicate combines with an accusative and the embedded clause predicate with an instrumental, \tit{waarmee} is grammatical and \tit{met wat} is ungrammatical. When the main clause predicate combines with an instrumental and the embedded clause predicate with an accusative, \tit{waarmee} is ungrammatical and \tit{met wat} is used.

\begin{table}[ht]\label{tbl:distribution}
	\center
	\caption {Distribution between \tit{waarmee} and \tit{met wat}}
	\begin{minipage}{0.45\linewidth}
		\begin{tabularx}{\textwidth}{ccc}
		\toprule
                              & \tit{waarmee} & \tit{met wat} \\
		\midrule
    m:\tsc{acc}, e:\tsc{ins}  & ✔             & *             \\
    m:\tsc{ins}, e:\tsc{acc}  & *             & ✔             \\
    \bottomrule
\end{tabularx}
\end{minipage}
\end{table}

In this section I first show that \tsc{r}-pronouns are the default complement of a preposition. Next, I illustrate that a necessary requirement for an \tsc{r}-pronoun is that is forms a proper constituent.


\subsection{\tsc{R}-pronouns as default}\label{sec:rdefault}

The goal of this section is to show that \tit{waarmee} `with what' is the default as instrumental relative pronoun. This generalization is not new, it has already been made \citet{riemsdijk1978,koopman2000}. In order to show that \tit{waarmee} `with what' is the default, I discuss the distribution of \tsc{r}-pronouns and regular pronouns in more general. I start with the personal pronouns and then return to the wh-pronouns.

Dutch has the personal pronouns \tit{haar} `her', \tit{hem} `him' and \tit{het} `it' that can be used as animate and inanimate objects of verbs, as illustrated in \ref{ex:objverb}.

 \ex. Objects of verbs \label{ex:objverb}
 \ag. Ik zie haar/hem.\\
  I see her/him\\
  `I see her/him.'\label{ex:aniobj}
 \bg. Ik zie 't.\\
  I see it\\
  `I see it.'\label{ex:inaniobj}

The example in \ref{ex:prepani} shows that for animate objects the same pronouns (\tit{haar} `her' and \tit{hem} `him') appear as objects of prepositions. However, the inanimate personal pronoun \tit{het} `it' cannot be used as an object of a preposition, shown in \ref{ex:prephet}. Instead, an ʀ-pronoun appears. This is illustrated in \ref{ex:preper}. \ref{ex:erprep} shows that the \tsc{r}-pronoun obligatorily moves to the left of the pronoun.

\ex. Objects of prepositions \label{ex:objprep}
\ag. Ik schilder samen met haar/hem.\\
 I paint together with her/him\\
 `I am painting together with her/him.'\label{ex:prepani}
\bg. *Ik schilder met 't.\\
 I paint with it\\
 `I am painting with it.'\label{ex:prephet}
\bg. Ik schilder er-mee.\\
 I paint there-with\\
 `I am painting with it.'\label{ex:preper}
\bg. *Ik schilder mee er.\\
 I paint with-there\\
 `I am painting with it.'\label{ex:erprep}

\tit{Met} is not the only preposition with which this happens. \tit{Op} `on' and \tit{in} `in' do not combine with the inanimate personal pronoun \tit{'t}, but the \tsc{r}-pronoun is used obligatorily.

\ex.
\ag. Ik zit er-op.\\
 I sit it-on\\
 `I am sitting on it.
\bg. *Ik zit op 't.\\
 I sit on it\\
 `I am sitting on it.

\ex.
 \ag. Hij zwemt er-in.\\
  He swims it-in\\
  `He is swimming in it.'
 \bg. *Hij zwemt in 't.\\
  He swims in it\\
  `He is swimming in it.'

The situation of the inanimate wh-pronouns resembles the inanimate personal pronouns. \tit{Wat} `what' can function as an object of a verb (see \ref{ex:wat}), but not as an object of a preposition \ref{ex:metwat}. In that case, the \tsc{r}-pronoun \tit{waarmee} `with what' is appears, as shown in \ref{ex:waarmee}.
\footnote{The sentence in \ref{ex:metwat} is unacceptable with neutral intonation. It becomes is only acceptable if \tit{wat} `what' is stressed, for example in a context in which the speaker questions the choice for the object hearer is painting with.}

\ex.
\ag. Wat zie jij?\\
 what see you\\
 `What do you see?'\label{ex:wat}
\bg. *Met wat schilder jij?\\
 With what paint you\\
 `What are you painting with?'\label{ex:metwat}
\bg. Waar schilder jij mee?\\
 Where paint you with\\
 `What are you painting with?'\label{ex:waarmee}

\tit{Waarmee} `with what' and not \tit{met wat} `with what' does not only appear in wh-questions, but also in other contexts. \ref{ex:headed} gives an example of a headed relative, and \ref{ex:headless} shows a free relative in which both predicates combine with an instrumental PP. The use of \tit{met wat} `with what' is ungrammatical in both contexts, and \tit{waarmee} `with what' is used.

\ex.\label{ex:headed}
\ag. Ik schilder met de kwast waar jij ook mee schildert.\\
 I paint with the brush where you also with paint\\
 `I am painting with the brush that you are painting with too.'
\bg. *Ik schilder met de kwast met wat jij ook schildert.\\
 I paint with the brush with what you also paint\\
 `I am painting with the brush that you are painting with too.'

 \ex.\label{ex:headless}
 \ag. Ik schilder waar jij ook mee schildert.\\
  I paint where you also with paint\\
  `I am painting with what you are painting with too.'
 \bg. *Ik schilder met wat jij ook schildert.\\
  I paint with what you also paint\\
  `I am painting with what you are painting with too.'

Table \ref{tbl:inanimates} summarizes the distribution of inanimates in Dutch. Inanimate (personal and wh-)pronouns in Dutch can function as objects verbs, but they are ungrammatical as objects of prepositions. In these contexts, \tsc{r}-pronouns appear.

\begin{table}[ht]\label{tbl:inanimates}
	\center
	\caption {Inanimates in Dutch}
	\begin{minipage}{0.7\linewidth}
		\begin{tabularx}{\textwidth}{ccc}
		\toprule
                        & pers. pronouns & wh-pronouns \\
  	\midrule
objects of verbs        & ət             & wat         \\
objects of prepositions & er             & waar        \\
\bottomrule
\end{tabularx}
\end{minipage}
\end{table}

The next section discusses the role of constituency in \tsc{r}-pronouns.


\subsection{\tit{Waarmee} is a constituent, \tit{met wat} is not}

Let me now return to the mismatching free relatives. I repeat the relevant grammatical examples in \ref{ex:grammatical}.

\ex.\label{ex:grammatical}
\ag. Ik heb gekocht waar jij mee schildert.\\
 I have bought where you with paint\\
 `I bought what you are painting with.'\label{ex:grammaticalwaar}
\bg. Ik schilder met wat jij hebt gekocht.\\
 I paint with what you have bought\\
 `I paint with what you bought.'\label{ex:grammaticalwat}

In this section I showed that ʀ-pronouns are expected in combinations with prepositions. This means that the use of \tit{waarmee} `with what' \ref{ex:grammaticalwaar} is not surprising. Something that is surprising is the use of \tit{met wat} `with what' in \ref{ex:grammaticalwat}, and this is the example something more needs to be said about. In the remainder of this section I argue that this `something more' is that the instrumental PP in \ref{ex:grammaticalwat} does not form a proper constituent, i.e. it is not a constituent to the exclusion of any other elements. The other side of the coin is that constructions with \tsc{r}-pronouns contain a PP that does form a proper constituent.

Below I repeat the examples with instrumental PPs I discuss so far in this paper.

\ex.
\ag. Ik schilder er-mee.\\
 I paint there-with\\
 `I am painting with it.'\label{ex:const1}
\bg. Waar schilder jij mee?\\
 Where paint you with\\
 `What are you painting with?'\label{ex:const2}
\bg. Ik schilder met de kwast [waar jij ook mee schildert].\\
 I paint with the brush where you also with paint\\
 `I am painting with the brush that you are painting with too.'\label{ex:const3}
\bg. Ik schilder [waar jij ook mee schildert].\\
 I paint where you also with paint\\
 `I am painting with what you are painting with too.'\label{ex:const4}

In each of these examples the instrumental PP forms a constituent at a certain point in the derivation. In \ref{ex:const1}, the PP forms a proper constituent in the surface order, as shown in \ref{ex:const1stage}. In \ref{ex:const2}, the PP forms a proper constituent before wh- and V2- movement, shown in \ref{ex:const2stage}.
The structure in \ref{ex:const3stage} represents a stage in the derivation of the embedded clauses in \ref{ex:const3} and \ref{ex:const4}.
\footnote{I assume that the antecedent in a free relative is a phonologically empty element, in line with cf. \cite{bresnan1978a,groos1981,himmelreich2017}. Under that view, \ref{ex:const3} and \ref{ex:const4} are identical, except for that in \ref{ex:const4} modifies phonologically empty element.}
Again, in the stage, which comes before relative movement of the pronoun to the left periphery of the relative clause, the PP forms a proper constituent.

\ex.
\a. [[ik] [[schilder] [er-mee]]]\label{ex:const1stage}
\b. [[jij] [[schilder] [[waar] [mee]]]]\label{ex:const2stage}
\b. [[jij] [[ook] [[schilder] [[waar] [mee]]]]]\label{ex:const3stage}

There is additional evidence for the fact that \tit{waarmee} `with what' forms a constituent in the constructions above. In the example in \ref{ex:const2}-\ref{ex:const4} only \tit{waar} `where' was fronted, but \ref{ex:constwaarmee} shows that also the phrase containing \tit{mee} `with' can be moved. As wh-movement can only target constituents, it follows that \tit{waar} `where' and \tit{mee} `with' have to form a constituent.

\ex.\label{ex:constwaarmee}
\ag. Waarmee schilder jij?\\
 {where with} paint you\\
 `What are you painting with?'
\bg. Ik schilder [waarmee jij ook schildert].\\
 I paint {where with} you also paint\\
 `I am painting with what you are painting with too.'
\bg. Ik schilder met de kwast [waarmee jij ook schildert].\\
 I paint with the brush {where with} you also paint\\
 `I am painting with the brush that you are painting with too.'

The mismatching free relative in \ref{ex:grammaticalwat} is not the only construction in which the string \tit{met wat} `with what' appears. I give examples of two more occurrences in \ref{ex:moremetwat}. In \ref{ex:watindef}, \tit{wat} `what' is the \tit{wat} `what' in the so-called \tit{wat voor} `what for'-construction \citep[cf.][]{corver1991}.
In \ref{ex:watindef}, \tit{wat} appears as a quantifier, and it means `some'. In both construction \tit{wat} `what' takes a complement and \tit{met wat} `with what' do not form a proper constituent. The brackets within the examples indicate the constituency.

\ex.\label{ex:moremetwat}
\ag. [Met [wat [voor [potloden]]] teken jij?\\
 with what for pencils draw you\\
 `What kind of pencils do you with?'\label{ex:watwasfur}
\bg. Ik wil graag thee [met [wat [suiker]]].\\
 I want please tea with some sugar\\
 `I would like to have tea with some sugar.'\label{ex:watindef}

Let me now show how this applies to the examples with the mismatching free relatives. The two predicates I used in the free relatives are \tit{kopen} `to buy' and \tit{schilderen}  `to paint'. \tit{Kopen} `to buy' takes an accusative DP as its object, illustrated in \ref{ex:kopen}. \tit{Schilderen} `to paint' can take an instrumental PP as its object., shown in \ref{ex:schilderen}.\footnote{Of course, \tit{schilderen} also optionally takes an accusative DP, but I am focussing on the instrumental here.}

\ex.
\ag. Ik koop het schilderij.\\
 I buy the painting\\
 `I am buying the painting.'\label{ex:kopen}
\bg. Ik schilder met een kwast.\\
 I paint with a brush\\
 `I am painting with a brush.'\label{ex:schilderen}

I repeat the mismatching free relative in which \tit{waarmee} `met wat' appears in \ref{ex:mismatchwaarmee}. The predicate \tit{schildert} `paints' combines in the embedded clause with the instrumental PP. The PP forms a proper constituent within the embedded clause, and the PP can be realized as the \tsc{r}-pronoun \tit{waarmee} `with what'.\footnote{I assume that the embedded clause modifies a phonologically empty noun here which is the DP argument of \tit{gekocht} `bought'.}

\exg. Ik heb gekocht [waar jij mee schildert].\\
 I have bought where you with paint\\
 `I bought what you are painting with.'\label{ex:mismatchwaarmee}

Next, we arrive at the mismatching free relative in which \tit{waarmee} `with what' cannot be used, but \tit{met wat} `with what' appears. The embedded clause predicate \tit{gekocht} `bought' combines with an accusative DP. The accusative object of a verb is always \tit{wat} `what', as I showed in \ref{ex:wat}. The instrumental PP only comes into the picture in the main clause, when \tit{schilder} `paint' combines with an instrumental PP. At no point in the derivation does the PP form a proper constituent, and \tit{waarmee} `with what' does not surface.

\exg. Ik schilder met [wat jij hebt gekocht].\\
 I paint with what you have bought\\
 `I paint with what you bought.'\label{ex:mismatchmetwat}

\ref{ex:summaryconst} summarizes what I showed in this section. \tit{Met wat} `with what' can never surface when \tit{met} `with' and \tit{wat} `what' form a proper constituent. It always becomes \tit{waarmee} `with what'. This is schematically shown in \ref{ex:waarmeefr}.
There are other contexts in which \tit{met wat} `with what' appears. This can be either when \tit{wat} `what' takes a complement, or when \tit{wat} is part of the a clause that \tit{met} `with' is not a part of.This last option is schematically showed in \ref{ex:metwatfr}, and it represents the mismatching free relative in \ref{ex:mismatchmetwat}.

\ex.\label{ex:summaryconst}
\a. *[[met] [wat]] → [waarmee]\label{ex:waarmeefr}
\b. [met [wat [X]]]
\b. [met [[wat] [X]]]\label{ex:metwatfr}

This section showed that being a proper constituent is a necessary requirement for \tit{waarmee} `with what' to surface. The next section elaborates on how proper constituency and spellout relate to each other.


\section{\tit{Waarmee} as suppletion}

\tit{Met} `with' is a preposition that combines with full DPs and animate pronouns. \tit{Wat} `what' is a wh-element that appears as subject or object. There is no principled reason why \tit{met wat} `with what' should be ruled out. Previous work on \tsc{r}-pronoun has not given many insights, other than \citet{riemsdijk1978} postulating a filter that prohibits the existence of constituents consisting of prepositions and wh-elements or \citet{koopman2000} who makes reference to a different paradigm.

In this section I start with showing a similar instance in which proper constituency and spellout go hand in hand. Then I show how these facts can be straightforwardly follow from the spellout algorithm in Nanosyntax. The section ends with showing how this logic can be applied to \tit{waarmee} `with what' and \tit{met wat} `with what'.

In what follows I show another instance in which proper constituency influences spellout, suppletive negative forms in Korean as discussed in \citet{chung2007}. The point is that suppletion only goes through when the relevant features form a proper constituent, and not when an additional feature intervenes.

Sentences in Korean can be negated by the two negative prefixes \tit{ani} or \tit{mos}, as shown in \ref{ex:koreannegation}.

\ex.\label{ex:koreannegation}
\ag. ca -n -ta\\
 sleep -\tsc{pres} -\tsc{decl}\\
 `is sleeping'
\bg. mos/an(i) ca -n -ta\\
 \tsc{neg} sleep -\tsc{pres} -\tsc{decl}\\
 `cannot sleep / is not sleeping' \hfill \citep{chung2007}

The verb \tit{al-} `know' does not combine with any of the negative markers, but it it uses the suppletive stem \tit{molu.\tsc{neg}}. This is illustrated in \ref{ex:koreanknow}.

\ex.\label{ex:koreanknow}
\ag. al -n -ta\\
know \tsc{-pres} \tsc{-decl}\\
`know(s)'
\bg. *mos/*an(i) al -n -ta\\
\tsc{neg} know -\tsc{pres} -decl\tsc{}\\
\bg. molu -n -ta\\
\tsc{neg}.know -\tsc{pres} -\tsc{decl}\\
`do(es) not / cannot know'\label{ex:molu} \hfill \citep{chung2007}

When \tit{al-} `know' is causativized before it is negated, the regular stem is used again, shown in \ref{ex:knowcausneg}. \ref{ex:molucaus} illustrates that the suppletive stem \tit{molu-} is no longer grammatical.

\ex.
\ag. al -li-\\
 know -\tsc{caus}\\
 `let know, inform'
\bg. ani/ mos al -li -ess -ta\\
 \tsc{neg} \tsc{neg} know -\tsc{caus} -\tsc{past} -\tsc{decl}\\
 `did not /could not inform'\label{ex:knowcausneg}
\bg. *molu -li -ess -ta\\
 \tsc{neg}.know -\tsc{caus} -\tsc{past} -\tsc{decl}\\\label{ex:molucaus} \hfill \citep{chung2007}

\citet{chung2007} points out that it cannot be phonological adjacency of the negation and \tit{al-} `know' that causes the suppletion in \ref{ex:molu}. The reason for that is that in \ref{ex:knowcausneg} \tit{al-} also follows the negation, but there is no suppletion. \citet{chung2007} suggests crucial difference has to do with in proper constituency.
The structure of \ref{ex:molu} is given in \ref{ex:structuremolu}: Neg and \tit{al-} `know' form a proper constituent. \ref{ex:structurenegmolu} represents the structure of \ref{ex:knowcausneg}. Neg and \tit{al-} `know' no longer form a proper constituent, because \tsc{caus} is also part of the structure as the sister of \tit{al-} `know'.

\ex.
%%%
\a. \begin{forest} boom
[
    [Neg]
    [know]
]
\end{forest}\label{ex:structuremolu}
\b. \begin{forest} boom
[
    [Neg]
    [
        [know]
        [\tsc{caus}]
    ]
]
\end{forest}\label{ex:structurenegmolu}

\citet{caha2009} uses this pattern to argue in favor of a model with phrasal spellout rather than inserting phonology into terminal nodes, specifically Nanosyntax \citep{starke2009}. The idea is that the lexicon contains syntactic structures that are linked to a particular piece of phonology. If Korean has a lexical entry as in \ref{ex:entrymolu}, this can be inserted into the syntactic tree in \ref{ex:structuremolu}.

\ex. \begin{forest} boom
[
    [Neg]
    [know]
]
  {\draw (.east) node[right]{⇔ \tit{molu}}; }
\end{forest}\label{ex:entrymolu}

Spellout targets only constituent \citep{starke2009}. This ensures \ref{ex:entrymolu} is not inserted into the structure in \ref{ex:structurenegmolu}, as \tsc{caus} is contained in the syntactic structure but not in the structure in the lexicon in \ref{ex:entrymolu}.

The same logic can be applied to \tit{waarmee} `with what' and \tit{met wat} `with what'. The preposition and wh-element together form a proper constituent, as in \ref{ex:structurewaarmee}, or the embedded CP of the relative clause is a sister of the wh-element, as in \ref{ex:structuremetwat}.

\ex.
\a. \begin{forest} boom
[
    [P]
    [Wh]
]
\end{forest}\label{ex:structurewaarmee}
%%
\b. \begin{forest} boom
[
    [P]
    [
        [Wh]
        [CP
            [...,roof]
        ]
    ]
]
\end{forest}\label{ex:structuremetwat}

Assume we have a lexical entry for \tit{waarmee} `met wat' that look as \ref{ex:entrywaarmee}. This entry can only be inserted into \ref{ex:structurewaarmee}, because the lexical structure matches the syntactic structure. It cannot be into \ref{ex:structuremetwat}, because the lexical structure does not contain the CP.

\ex. \begin{forest} boom
[
    [InsP]
    [WhP]
]
  {\draw (.east) node[right]{⇔ \tit{waarmee}}; }
\end{forest}\label{ex:entrywaarmee}

What I have shown in this section is how proper constituency relates to a change in phonology. However, this analysis missed numerous phonological properties of \tit{waarmee} `with what' and \tit{met wat} `with what', e.g. that they both contain the morpheme \tit{wa}. Under the suppletion analysis above, one can the phonological form of \tit{waarmee} could just as well be completely phonologically unrelated. The point of presenting the problem like this is to illustrate that the distribution between \tit{waarmee} `with what' and \tit{met wat} `with what' can be captured as a matter of spellout.

In the next section I decompose \tit{waarmee} `with what' and \tit{met wat} `with what'. I show that both spell out the same set of features, but the distribution is different.


\section{A finer decomposition}

In this section I investigate the internal structure of \tit{waarmee} `with what' and \tit{met wat} `with what' to capture the phonological similarities between the two forms. First, I identify \tit{w-} and \tit{-a-} as morphemes that appear in both expressions. Putting these two aside, I concentrate on \tit{ermee} `with it' and \tit{met 't} `with it'. I decompose these two expressions further, and end up arguing for a picture as shown in Table \ref{tbl:nofeatures}.
The elements \tit{w-} and \tit{-a-} express the same syntactic structure in both expressions. The elements \tit{met} and \tit{'t} together also express the same features as \tit{mee} and \tit{er} together, but the distribution differs. \tit{Met} expresses less structure than \tit{mee} and \tit{'t} expresses more structure than \tit{er}. It becomes clear that the \tit{er} in fact corresponds to the locative \tit{r} in Dutch.

\begin{table}[ht]
	\center
	\caption {\tit{met wat} and \tit{waarmee}}
	\begin{minipage}{0.56\linewidth}
		\begin{tabularx}{\textwidth}{ccccccc}
		\toprule
    \phantom{\tsc{wh}}  & \phantom{\tsc{deix}}                    & \phantom{\tsc{f}4}  & \phantom{\tsc{f}3} & \phantom{\tsc{f}2}  & \phantom{\tsc{f}1}  & \phantom{\tsc{thing}} \\
		%\midrule
    \tit{w}   & \multicolumn{1}{|c|}{\tit{a}}  & \tit{met} & \multicolumn{4}{|c}{\tit{(ə)t}}                \\\hline
    \tit{w}   & \multicolumn{1}{|c|}{\tit{a}}  & \multicolumn{4}{c|}{\tit{mee}}               & \tit{(ə)r}  \\
    \bottomrule
\end{tabularx}
\end{minipage}
\end{table}\label{tbl:nofeatures}

Finally, I put everything together and show how \tit{waarmee} `with what' is inserted if all features form a constituent, and \tit{met wat} `with what' if this is not the case.


\subsection{Taking everything apart}

In this section I decompose \tit{waarmee} `with what' and \tit{met wat} `with what'. The idea is that both expessions realize the same features, and \tit{waarmee} `with what' takes precedence when all features form a constituent. The morphemes I distinguish in \tit{waarmee} `with what' are \tit{w-}, \tit{-aa-}, \tit{-r} and \tit{-mee}. Within \tit{met wat} `with what' I distinguish \tit{met}, \tit{w-}, \tit{-a-} and \tit{-t}. \ref{ex:decompose} shows this as well.

\ex.\label{ex:decompose}
\a. w -aa -r -mee
\b. met w -a -t

Let me start with the morphemes \tit{w-} and \tit{-a-} that appear in both expressions. Exactly because they appear in both expression, I do not pay much attention to them. For \tit{w-} I follow \citet{hachem2015} in that \tit{w-} realizes a WP.

\ex. \begin{forest}
[WP
    [W, roof]
]
{\draw (.east) node[right]{⇔ \tit{w}}; }
\end{forest}\label{ex:entryw}

The morpheme \tit{-a-} expresses deixis. Dutch distinguishes between proximal by using \tit{-ie-/-i-} and distal by using \tit{-aa-/-a-}, illustrated in \ref{ex:deixis}. I assume that lengthening or shortening of the vowel is a result of the final \tit{r}.

\ex.\label{ex:deixis}
\ag. h-ie-r\\
 here\\
\bg. d-aa-r\\
 there\\
\bg. d-i-t\\
 this\\
\bg. d-a-t\\
 that\\
 \z.

In combination with wh-elements, the distal marker \tit{-a-} is used. For the purpose of this paper I simply let \tit{-a-} correspond to \tsc{deixP}.

\ex. \begin{forest}
[deixP
    [deix, roof]
]
{\draw (.east) node[right]{⇔ \tit{a}}; }
\end{forest}\label{ex:entrya}

I put \tit{w-} and \tit{-a-} aside for now, assuming they spell out the same syntactic structure in \tit{waarmee} `with what' and \tit{met wat} `with what'. This leaves \tit{er-mee} `with it' and \tit{met 't} `with it'.\footnote{\tit{'rmee} `with it' is often precdeded by an /ε/, which I assume to be for phonological reasons.}

\ex.
\ag. (e)r -mee\\
there with\\
\b. met 't\\
with it\\

I repeat the examples from Section \ref{sec:rdefault} below that show that \tit{ermee} `with it' is used when it forms a proper constituent and \tit{met 't} `with it' is ungrammatical.

\ex.
\ag. Ik schilder er-mee.\\
 I paint there-with\\
 `I am painting with it.'\label{ex:jarmee}
\bg. *Ik schilder met 't.\\
 I paint with it\\
 `I am painting with it.'\label{ex:neemett}

In other words, both expessions realize the same features, and \tit{ermee} `with it' takes precedence when all features form a constituent.

Let me continue with specifying the lexical entries for \tit{met 't} `with it'. \tit{'t} `it' can be used as subject, direct object and indirect object, as shown in \ref{ex:tsubobj}.

\ex.\label{ex:tsubobj}
\ag. 't Staat in de hal.\\
 \tsc{3sg.n.nom} stands in the hallway\\
 `It is standing in the hallway.'
\bg. Ik zie 't.\\
 I see \tsc{3sg.n.acc}\\
 `I see it.'
\bg. Ik heb 't een klap gegeven.\\
 I have \tsc{3sg.n.dat} a hit given\\
 `I hit it.'

Pronouns in other genders alternate between nominative (non-oblique) and accusative/dative (oblique) in these contexts, illustrated in \ref{ex:hemsubobj}.

\ex.\label{ex:hemsubobj}
\ag. Hij staat in de hal.\\
 \tsc{3sg.m.nom} stands in the hallway\\
 `He is standing in the hallway.'
\bg. Ik zie hem.\\
 I see \tsc{3sg.m.acc}\\
 `I see it.'
\bg. Ik heb hem een klap gegeven.\\
 I have \tsc{3sg.m.acc} a hit given\\
 `I hit him.'

I assume that the inanimate personal pronoun minimally contains the ontological category \tsc{thing} \citep{kayne2005}. I leave possible number and gender features out because they do not play a role in this paper. For case, I follow \citet{caha2009} that case features case features are organized the containment relation in \ref{ex:casetree}. The higher, more complex cases contain the smaller, less complex cases.

\ex. \label{ex:casetree}
\begin{forest} boom
[\tsc{comP}
    [\tsc{f5}]
    [\tsc{insP}
        [\tsc{f4}]
        [\tsc{datP}
            [\tsc{f3}]
            [\tsc{accP}
                [\tsc{f2}]
                [\tsc{nomP}
                    [\tsc{f1}]
                    [DP]
                ]
            ]
        ]
    ]
]
\end{forest}

The morpheme \tit{'t} can act as nominative, accusative and dative, as I showed in \ref{ex:tsubobj}. Taking this all together, \tit{'t} should have the lexical entry given in \ref{ex:entryt}.

\ex. \begin{forest} boom
 [\tsc{dat}P
     [\tsc{f}3]
     [\tsc{acc}P
         [\tsc{f}2]
         [\tsc{nom}P
             [\tsc{f}1]
             [\tsc{thingP}
                 [\tsc{thing}, roof]
             ]
         ]
     ]
 ]
 {\draw (.east) node[right]{⇔ \tit{(ə)t}}; }
 \end{forest}\label{ex:entryt}

This lexical entry can lexicalize the \tsc{datP}, but also the \tsc{accP} and \tsc{nomP}. This is due to the Superset Princple.

 \ex. The Superset Principle \citet{starke2009}: \\
 A lexically stored tree matches a syntactic node iff the lexically stored tree contains the syntactic node.

In other words, a lexically stored structure does not have to be identical to the syntactic structure. It is enough if the syntactic structure is contained in the lexically stored tree.

\tit{Met} `with' can be used for as instrumental or comitative, as shown in \ref{ex:metinscom}.

\exg. Ik dans met hem.\\
 I dance with him\\
 `I am dancing with him.'\label{ex:metinscom}

Again, due to the superset principle, \tit{met} `with' can be used for comitative and as instrumental case. It combines with the dative, so the lowest feature is \tsc{f4}.

 \ex. \begin{forest} boom
 [\tsc{com}P
     [\tsc{f}5]
     [\tsc{ins}P
         [\tsc{f}4, roof]
     ]
 ]
 {\draw (.east) node[right]{⇔ \tit{met}}; }
 \end{forest}\label{ex:entrymet}

-------
now we get to \tit{er-mee}. they have to be able to lexicalize the same features as in \tit{met 't}.
-----

 \tit{Er} `there' can be used as a locative.

 \exg. Ik ben er al geweest.\\
 I am there already been\\
 `I have already been there.'

I follow \cite{baunaz2018} in assuming that the ontological category \tsc{location} contains \tsc{thing}.\footnote{\citet{baunaz2018} place in addition \tsc{person} between \tsc{thing} and \tsc{location}, which I left out here.}

 \ex. \begin{forest} boom
 [\tsc{loc}P
 [\tsc{location}]
     [\tsc{thngP}
         [\tsc{thing}, roof]
     ]
 ]
 {\draw (.east) node[right]{⇔ \tit{(ə)r}}; }
 \end{forest}\label{ex:entryr}

So \tit{met} expresses \tsc{f}4 and \tsc{f5}, and \tit{(ə)r} expresses \tsc{thing} and \tsc{f}1 to \tsc{f}3. \tit{(ə)r} only expresses \tsc{thing}. \tit{Ermee} and \tit{met 't} express the same features, which leaves \tsc{f}1 to \tsc{f}5 for \tit{mee}.

And! Nanosyntax encodes the difference between prefixes and suffixes by the difference in the shape of the lexical tree. \tit{-mee} follows \tit{er}, so I am going to call it a suffix. So:

\ex. \begin{forest} boom
[\tsc{com}P
    [\tsc{f}5]
    [\tsc{insP}
        [\tsc{f}4]
        [\tsc{dat}P
            [\tsc{f}3]
            [\tsc{acc}P
                [\tsc{f}2]
                [\tsc{nom}P
                    [\tsc{f}1]
                    [\phantom{x}]
                ]
            ]
        ]
    ]
]
{\draw (.east) node[right]{⇔ \tit{mee}}; }
\end{forest}\label{ex:entrymee1}



\begin{table}[ht]
	\center
	\caption {\tit{met wat} and \tit{waarmee}}
	\begin{minipage}{0.56\linewidth}
		\begin{tabularx}{\textwidth}{ccccccc}
		\toprule
    \phantom{\tsc{wh}}  & \phantom{\tsc{deix}}                    & \phantom{\tsc{f}4}  & \phantom{\tsc{f}3} & \phantom{\tsc{f}2}  & \phantom{\tsc{f}1}  & \phantom{\tsc{thing}} \\
		%\midrule
    \tit{w}   & \multicolumn{1}{|c|}{\tit{a}}  & \tit{met} & \multicolumn{4}{|c}{\tit{(ə)t}}                \\\hline
    \tit{w}   & \multicolumn{1}{|c|}{\tit{a}}  & \multicolumn{4}{c|}{\tit{mee}}               & \tit{(ə)r}  \\
    \bottomrule
\end{tabularx}
\end{minipage}
\end{table}\label{tbl:nofeatures}




% I do not care whether \tit{w-} and \tit{-a-} are merged before or after.
%
% WHY DOES MEE MOVE? I DO NOT KNOW.
% BUT IT DOES.
% AND SO DOES WAAR.
% waar maybe because it is remnant movement?
%
%
%
% Either way, it combines with accusative/dative (oblique), which can be seen on the pronouns. But only for the comitative, because for the instrumental we are getting an ʀ-pronoun. Full DPs do not show any marking, leading me to postulate the zero marker up to the dative.
%
% \ex.
% \ag. Ik dans met de man.\\
%  I dance with the man\\
%  `I am dancing with the man.'
% \bg. Ik schilder met een kwast.\\
%  I paint with een brush\\
%  `I am painting with een brush.'
%
% \ex. \begin{forest} boom
% [\tsc{datP}
%    [\tsc{f}3]
%    [\tsc{acc}P
%        [\tsc{f}2]
%        [\tsc{nom}P
%            [\tsc{f}1]
%            [\phantom{x}]
%        ]
%    ]
% ]
% {\draw (.east) node[right]{⇔ \tit{∅}}; }
% \end{forest}»




\subsection{Putting it back together}



Spellout

\ex. The Elsewhere Condition (\citealt{kiparsky1973}, formulated as in \citealt{caha2020}):\\
When two entries can spell out a given node, the more specific entry wins. Under the Superset Principle governed insertion, the more specific entry is the one which has fewer unused features

Spellout algorithm \citep{starke2018}

\ex. Merge F and
 \a. Spell out FP
 \b. If (a) fails, attempt movement of the spec of the complement of \tsc{f}, and retry (a)
 \b. If (b) fails, move the complement of \tsc{f}, and retry (a)

\ex. Cyclic Override \citep{starke2018}:\\
Lexicalisation at a node XP overrides any previous match at a phrase contained in XP.

\ex. Backtracking \citep{starke2018}:
When spellout fails, go back to the previous cycle, and try the next option for that cycle.

\ex. Spec Formation \citep{starke2018}:\\
If Merge F has failed to spell out (even after backtracking), try to spawn a new derivation providing the feature F and merge that with the current derivation, projecting the feature F at the top node.




The functional sequence of \tit{ermee}.

\ex. \tsc{f}4 > \tsc{f}3 > \tsc{f}2 > \tsc{f}1 > \tsc{thing}



We start with \tsc{thing}. This is realized as \tit{(ə)r}, because it has less junk than \tit{(ə)t}.

\ex.
\begin{forest} boom
 [\tsc{thingP}
     [\tsc{thing}, roof]
 ]
{\draw (.east) node[right]{⇔ \tit{(ə)r}}; }
\end{forest}

\tsc{f}1 is merged. \tit{(ə)r} is no longer a candidate, but \tit{(ə)t} is. So the spellout is overridden.

\ex. \begin{forest} boom
[\tsc{nom}P
   [\tsc{f}1]
   [\tsc{thingP}
       [\tsc{thing}, roof]
   ]
]
{\draw (.east) node[right]{⇔ \tit{(ə)t}}; }
\end{forest}

\tsc{f}2 is merged, which can still be realizd by \tit{(ə)t}.

\ex. \begin{forest} boom
[\tsc{accP}
   [\tsc{f}2]
   [\tsc{nom}P
       [\tsc{f}1]
       [\tsc{thingP}
           [\tsc{thing}, roof]
       ]
   ]
]
{\draw (.east) node[right]{⇔ \tit{(ə)t}}; }
\end{forest}

Idem for \tsc{f}3.

\ex. \begin{forest} boom
[\tsc{datP}
    [\tsc{f}3]
    [\tsc{accP}
       [\tsc{f}2]
       [\tsc{nom}P
           [\tsc{f}1]
           [\tsc{thingP}
               [\tsc{thing}, roof]
           ]
       ]
    ]
]
{\draw (.east) node[right]{⇔ \tit{(ə)t}}; }
\end{forest}

Now \tsc{f}4 is merged. No candidate for phrasal spellout, no candidate after complement movement.

\ex.
\a. \begin{forest} boom
[\tsc{insP}
    [\tsc{f}4]
    [\tsc{datP}
        [\tsc{f}3]
        [\tsc{accP}
           [\tsc{f}2]
           [\tsc{nom}P
               [\tsc{f}1]
               [\tsc{thingP}
                   [\tsc{thing}, roof]
               ]
           ]
        ]
    ]
]
{\draw (.east) node[right]{⇔ }; }
\end{forest}
\b. \begin{forest} boom
[\phantom{x}
    [\tsc{datP}
        [\tsc{f}3]
        [\tsc{accP}
           [\tsc{f}2]
           [\tsc{nom}P
               [\tsc{f}1]
               [\tsc{thingP}
                   [\tsc{thing}, roof]
               ]
           ]
        ]
    ]
    {\draw (.east) node[right]{⇔ \tit{(ə)t}}; }
    [\tsc{insP}
        [\tsc{f}4]
    ]
    {\draw (.east) node[right]{⇔ }; }
]
\end{forest}

The next step in the algorithm is backtracking. We try the next option for the cycle in which \tsc{f}3 was merged. This option is complement movement. However, there is no lexical entry that fits, so we try the next option.

show here

The exact same happens for \tsc{f}2.

show here

Now for \tsc{f}1 it is different. There is a candidate when we try complement movemement, namely \tsc{thing} is realized as \tit{(ə)r} and \tsc{f}1 is spelled out as \tit{∅}

\ex. \begin{forest} boom
[\phantom{x}
   [\tsc{thingP}
       [\tsc{thing}, roof]
   ]
   {\draw (.east) node[right]{⇔ \tit{(ə)r}}; }
   [\tsc{nom}P
       [\tsc{f}1]
   ]
   {\draw (.east) node[right]{⇔ \tit{∅}}; }
]
\end{forest}

Merge \tsc{f}2, there is no candidate for phrasal spellout, we try spec to spec movement and get a match:

\ex. \begin{forest} boom
[\phantom{x}
   [\tsc{thingP}
       [\tsc{thing}, roof]
   ]
   {\draw (.east) node[right]{⇔ \tit{(ə)r}}; }
   [\tsc{acc}P
       [\tsc{f}2]
       [\tsc{nom}P
           [\tsc{f}1]
       ]
   ]
   {\draw (.east) node[right]{⇔ \tit{∅}}; }
 ]
\end{forest}

Idem for \tsc{f}3.

\ex. \begin{forest} boom
[\phantom{x}
   [\tsc{thingP}
       [\tsc{thing}, roof]
   ]
   {\draw (.east) node[right]{⇔ \tit{(ə)r}}; }
   [\tsc{dat}P
       [\tsc{f}3]
       [\tsc{acc}P
           [\tsc{f}2]
           [\tsc{nom}P
               [\tsc{f}1]
           ]
       ]
   ]
   {\draw (.east) node[right]{⇔ \tit{∅}}; }
]
\end{forest}

Now \tsc{f}4 is merged. No phrasal spellout, but there is a match after spec-to-spec movement. However, is is not \tit{∅} anymore, but the inserted morpheme is \tit{mee}.

\ex. \begin{forest} boom
[\phantom{x}
    [\tsc{thingP}
       [\tsc{thing}, roof]
    ]
    {\draw (.east) node[right]{⇔ \tit{(ə)r}}; }
    [\tsc{insP}
       [\tsc{f}4]
       [\tsc{dat}P
           [\tsc{f}3]
           [\tsc{acc}P
               [\tsc{f}2]
               [\tsc{nom}P
                   [\tsc{f}1]
               ]
           ]
       ]
    ]
    {\draw (.east) node[right]{⇔ \tit{mee}}; }
]
\end{forest}

Then show here that this does not happen if not all features form a proper constituent. Spellout can just not target a constituent.




I did not address the ordering difference of the wh-element and the adposition. A topic related to (but not relevant for) this paper is the different positioning of identical adpositions in Dutch. In \ref{ex:dutchin}, \tit{in} changes meaning dependent on whether it procedes or follows the DP, it is respectively locational or directional.

\ex.\label{ex:dutchin}
\ag. Ik klim in de boom.\\
 I climb in the tree\\
 `I am climbing in the tree.'
\bg. Ik klim de boom in.\\
 I climb the tree in\\
 `I am climbing into the tree.'

The different positioning of the adpositions in \tsc{r}-pronouns is driven by spellout and is, therefore, meaningless.






\section{Conclusion and discussion}

So, the answer of how constituency connects to change in phonological form lies in one of the core assumptions of nanosyntax: phrasal spellout spells out constituents.

Just like Abels said: no preposition stranding, but extraction from the PP is ok!


Van Riemsdijk said that R-pronouns are pronouns and they orginate in the complement of P. Due to \tsc{r}-movement they move to the spec of PP, and due to suppletion they change form, coincidentally to something identical to the locative. Full DPs cannot move to the spec of PP because they do not have access to that spec.

I claim that this movement is not a special r-movement, but a movement driven by spellout. The suppletion also follows from the regular spellout algorithm, and it is the locative that is inserted.


Koopman's account differ from Van Riemsdijk's in that she assumes \tsc{r}-pronouns obligatorily move to specPlace. The fact that locatives and \tsc{r}-pronouns are syncretic is not a surprise. They occur in the same structural configuration.

Place does not make a lot of sense in \tit{waarmee}, because we do not have a place here.

Abels says that this movement from the complement to the spec is not allowed. He argues that, instead, that R-words are base generated as the specifiers of distinguished class of zero-place prepositions. \tsc{r} is the complement of P and is subextracted from the complement of P (which is allowed, because it is subextraction and not extraction of it as a whole.)

I agree with Abels in that \tit{r} is substracted from the complement of the preposition. However, I say that \tit{r} is actually the pronoun. What he refers to as an \tsc{r}-word is in Dutch \tit{w-aa-}, which I also assume is merged later.


Now the difference in distribution between \tit{met wat} and \tit{waarmee} is going to follow from spellout. If we get higher than dative case, we need prepositions. Making prepositions is a very 'costly' operation in the langauge, it is the last resort option in the spellout procedure. Dutch also has forms that are not only prepositions, but can also function as postpositions. They are less 'costly'. To avoid building a complex specifier, different spellout options are used. That is why we get the \tsc{r}-pronoun in the picture. \tsc{r}-pronouns can combine with these postpositions.



Open ends

In the analysis laid out above it is a coincidence that \tit{mee} `with' and \tit{met} `with' look so much alike.

Another topic I do not discuss is that in almost all cases the preposition does not change form when it combines with an \tsc{r}-pronoun, e.g. \tit{in}. If this proposal is on the right track, such elements can be used as either a prefix and as a suffix. A lexical entry as in \ref{ex:presuf} would be a candidate for such an element.

\ex. \begin{forest} boom
[XP
    [XP
        [X,roof]
    ]
    [YP
        [Y]
    ]
]
{\draw (.east) node[right]{⇔ \tit{in}}; }
\end{forest}\label{ex:presuf}

I leave it to future research to determine whether this is a feasible solution.


still need to incorporate \tit{a} and \tit{w}. They can either come before case or after it.


\printbibliography

\end{document}
