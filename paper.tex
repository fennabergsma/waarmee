\documentclass{article}

\usepackage{fenna-files/packages}
\usepackage{fenna-files/commands}
\bibliography{fenna-files/references}{}

\title{Dutch, PPs, ʀ-pronouns}
\author{Fenna Bergsma}
\date{\today}

\begin{document}


\maketitle




\section{Introduction}

hoi

The term \tsc{r}-pronoun \citep{riemsdijk1978} refers to a set of nominal elements that can strand prepositions in Dutch (and German). It is special because Dutch is a normally non-preposition stranding language. \tsc{R}-pronouns morphologically consist of a combination of a preposition and locative pronoun. In the example \ref{ex:klimerop} \tit{mee} `with' is the preposition and \tit{er} `there' is the locative pronoun.

\exg. Ik klim er-op.\\
 I climb there-on\\
 `I am climbing on it.'\label{ex:klimerop}

This paper focuses on a single \tsc{r}-pronoun in a specific type of free relative construction, namely on the relative pronoun \tit{waarmee} `with what' in a free relative construction in which the predicates require different cases. The \tsc{r}-pronoun \tit{waarmee} `with what' is interesting for two reasons. First, just like for all \tsc{r}-pronouns, the wh-element is the locative, but there is no meaning component related to location in `with what'. Second, the preposition \tit{met} `with' changes into \tit{mee} `with' when it is combined withn \tsc{r}-pronoun.

The construction I focus on is a free relative construction, in which the two predicates combine with different cases. I illustrate this in \ref{ex:gekochtwaarmee}. The predicate in the embedded clause, \tit{schildert} `paint', combines with an instrumental PP. The predicate in the main clause clause, \tit{gekocht} `bought' combines with an accusative DP. The \tsc{r}-pronoun \tit{waarmee} `with what' is used here.


%%%%%%
Footnote
In this example \tit{mee} `with' is stranded and \tit{waar} `where' is moved to the left edge of the embedded clause. It is also possible for \tit{waar} `where' to bring \tit{mee} `with' along, as in \ref{ex:meealong} but it is regarded as less natural.

\exg. Ik heb gekocht waar jij mee schildert.\\
 I have bought where you with paint\\
 `I bought what you are painting with.'\label{ex:meealong}
Footnote
%%%%%%


\exg. Ik heb gekocht waar jij mee schildert.\\
 I have bought where you with paint\\
 `I bought what you are painting with.'\label{ex:gekochtwaarmee}

If one were to switch around the predicates between the clauses, the \tsc{r}-pronoun does not appear anymore. In \ref{ex:schildermet}, \tit{schilder} `paint' combines with an instrumental PP in the main clause and \tit{gekocht} `bought' combines with an accusative DP in the embedded clause. The use of an \tsc{r}-pronoun is ungrammatical, as indicated by the ungrammaticality of \ref{ex:schilderwaarmee}. Instead, a combination of the regular instrumental preposition \tit{met} `with' and the regular wh-pronoun \tit{was} `what' in used.

\ex.\label{ex:schildermet}
\ag. *Ik schilder waarmee jij hebt gekocht.\\
 I paint {where with} you have bought\\
 `I paint with what you bought.'\label{ex:schilderwaarmee}
\bg. Ik schilder met wat jij hebt gekocht.\\
 I paint with what you have bought\\
 `I paint with what you bought.'\label{ex:schildermetwat}

The use of \tit{met wat} `with what' is ungrammatical in the context in which \tit{waarmee} `with what' appeared in \ref{ex:gekochtwaarmee}. This is illustrated in \ref{ex:gekochtmetwat}.

\exg. *Ik heb gekocht met wat jij schildert.\\
 I have bought with what you paint\\
 `I bought what you are painting with.'\label{ex:gekochtmetwat}

In this paper I show that distribution of \tit{waarmee} `with what' and \tit{met wat} `with what' in these free relative constructions gives us a unique insight into the internal structure of \tsc{r}-pronouns. In what follows I show that \tsc{r}-pronouns and regular preposition compete to spell out the same syntactic features. If all features form a constituent, the \tsc{r}-pronoun surfaces. If the constituent is interrupted, the preposition-pronoun combination shows up. This straightforwardly follows in a system in which spellout targets phrasal constituents: Nanosyntax \citep{starke2006}.

This paper is structured as follows. First, I show that it really is constituency. Then I decompose \tsc{r}-pronouns, prepositions and regular pronouns. Last, I show in derivations that constituency connects to the choice for \tsc{r}-pronoun or preposition and regular pronoun. Unmarked examples are constructed and have been verified by native speakers.



\section{\tsc{R}-pronouns as default}

In the introduction I discussed the distribution between \tit{waarmee} `with what' and \tit{met wat} `with what' in free relatives with predicates that combine with different cases. Table \ref{tbl:distribution} repeats the generalization. When the main clause predicate combines with an accusative and the embedded clause predicate with an instrumental, \tit{waarmee} is grammatical and \tit{met wat} is ungrammatical. When the main clause predicate combines with an instrumental and the embedded clause predicate with an accusative, \tit{waarmee} is ungrammatical and \tit{met wat} is used.

\begin{table}[ht]
	\center
	\caption {Distribution between \tit{waarmee} and \tit{met wat}}
	\begin{minipage}{0.27\linewidth}
		\begin{tabularx}{\textwidth}{ccc}
		\toprule
                              & \tit{waarmee} & \tit{met wat} \\
		\midrule
    m:\tsc{acc}, e:\tsc{ins}  & ✓             & *             \\
    m:\tsc{ins}, e:\tsc{acc}  & *             & ✓             \\
    \bottomrule
\end{tabularx}
\end{minipage}
\end{table}\label{tbl:distribution}

The goal of this section is to show that \tit{waarmee} is the default as instrumental relative pronoun. In order to do that, I discuss the distribution of \tsc{r}-pronouns and regular pronouns in more general. Dutch has the personal pronouns \tit{haar} `her', \tit{hem} `him' and \tit{het} `it' that can be used as animate and inanimate objects of verbs, as illustrated in \ref{ex:objverb}.

 \ex. Objects of verbs \label{ex:objverb}
 \ag. Ik zie haar/hem.\\
  I see her/him\\
  `I see her/him.'\label{ex:aniobj}
 \bg. Ik zie 't.\\
  I see it\\
  `I see it.'\label{ex:inaniobj}

The example in \ref{ex:prepani} shows that for animate objects the same pronouns (\tit{haar} `her' and \tit{hem} `him') appear as objects of prepositions. However, the inanimate personal pronoun \tit{het} `it' cannot be used as an object of a preposition, shown in \ref{ex:prephet}. Instead, an ʀ-pronoun appears. This is illustrated in \ref{ex:preper}. \ref{erprep} shows that the \tsc{r}-pronoun obligatorily moves to the left of the pronoun.

\ex. Objects of prepositions \label{ex:objprep}
\ag. Ik schilder samen met haar/hem.\\
 I paint together with her/him\\
 `I am painting together with her/him.'\label{ex:prepani}
\bg. *Ik schilder met het.\\
 I paint with a brush\\
 `I am painting with it.'\label{ex:prephet}
\bg. Ik schilder er-mee.\\
 I paint there-with\\
 `I am painting with it.'\label{ex:preper}
\bg. *Ik schilder mee er.\\
 I paint with-there\\
 `I am painting with it.'\label{ex:erprep}

\tit{Met} is not the only preposition with which this happens. \tit{Op} `on' and \tit{in} `in' do not combine with the inanimate personal pronoun \tit{'t}, but the \tsc{r}-pronoun is used obligatorily.

\ex.
\ag. Ik zit er-op.\\
 I sit it-on\\
 `I am sitting on it.
\bg. *Ik zit op 't.\\
 I sit on it\\
 `I am sitting on it.

\ex.
 \ag. Hij zwemt er-in.\\
  He swims it-in\\
  `He is swimming in it.'
 \bg. *Hij zwemt in 't.\\
  He swims in it\\
  `He is swimming in it.'

A picture similar to the one for inanimate personal pronouns appears with wh-inanimates.

object of verbs \tit{wat}.
object of preposition with \tit{wat}


then show they appear in the contexts below.

\tit{Waarmee} and not \tit{met wat} also appears in other contexts. \ref{ex:headless} shows matching headless free relatives, \ref{ex:headed} shows headed relatives, \ref{ex:wh} gives wh-questions. The use of \tit{met wat} is ungrammatical here, and \tit{waarmee} should be chosen.

\ex. Matching headless relatives\label{ex:headless}
\ag. Ik schilder waar jij ook mee schildert.\\
 I paint \tit{w-aa-r} you also with paint\\
 `I am painting with what you are painting with too.'
\bg. *Ik schilder met wat jij ook schildert.\\
 I paint with what you also paint\\
 `I am painting with what you are painting with too.'

\ex. Headed relatives\label{ex:headed}
\ag. Ik schilder met de kwast waar-mee jij ook schildert.\\
 I paint with the brush \tit{w-aa-r}-with you also paint\\
 `I am painting with the brush that you are painting with too.'
\bg. *Ik schilder met de kwast met wat jij ook schildert.\\
 I paint with the brush with what you also paint\\
 `I am painting with the brush that you are painting with too.'

\ex. Wh-question\label{ex:wh}
\ag.  Waar-mee schilder jij?\\
 \tit{W-aa-r}-with paint you\\
 `What are you painting with?'
\bg. *Met wat schilder jij?\\
 With what paint you\\
 `What are you painting with?'

In sum, no inanimate pronominal elements in Dutch is able to act as the object of a preposition. It is always the \tsc{r}-pronoun that appears. What this shows is that ʀ-pronouns are not exceptions, but they are the default. There is something special about the examples in which \tit{met wat} is used. I suggest that the crucial difference lies in constituency. \tit{Waarmee} can only be used if the instrumental PP forms a proper constituent, i.e. a constituent to the exclusion of any other elements. --explain here how they all form a constituent. In the examples from xx to xx above this is the case.

There are actually more examples of \tit{wat} following \tit{met} in which \tit{met} and \tit{wat} do not form a constituent without containing any other elements. First, \tit{wat} is an indefinite that takes an NP complement, and only then it is combined with the \tit{met}. Second, \tit{wat} combines with \tit{voor} and then combines with an NP complement en then with \tit{met}. \tit{Met} and \tit{wat} are never a constituent to the exclusion of everything else.

\ex.
\ag. Ik wil graag thee met wat suiker.\\
 I want please tea with some sugar\\
 `I would like to have tea with some sugar.'
\bg. [Met [wat [voor [potloden]]] teken jij?\\
 with what for pencils draw you\\
 `What kind of pencils do you with?'

Let me now show how this applies to the examples with the mismatching relatives.


I repeat the example in which \tit{schildert} is in the embedded clause.

\ex.
\ag. Ik heb gekocht waar jij mee schildert.\\
 I have bought where you with paint\\
 `I bought what you are painting with.'
\bg. Ik schilder met wat jij hebt gekocht.\\
 I paint with what you have bought\\
 `I paint with what you bought.'

Now let us return to the mismatching headed relatives. The two predicates I used in these are \tit{kopen} `to buy' and \tit{schilderen}  `to paint'. \tit{Kopen} takes an accusative DP as its object. \tit{Schilderen} can take an instrumental PP as its object.\footnote{Of course, \tit{schilderen} also possibly takes an accusative DP, but I am focussing on the instrumental here.}

\ex.
\ag. Ik koop het schilderij.\\
 I buy the painting\\
 `I am buying the painting.'
\bg. Ik schilder met een kwast.\\
 I paint with a brush\\
 `I am painting with a brush.'


\tit{Schildert} wants the instrumental PP, so the embedded clause looks like this:

\ex. \begin{forest} boom
[
    [
        [gekocht, roof]
    ]
    [
        [waar jij mee schildert, roof]
    ]
]
\end{forest}


Here \tit{waar} and \tit{mee} are both part of the embedded clause. At a certain point in the derivation they formed a proper constituent, i.e. they formed a constituent without any other elements.


\ex. \begin{forest} boom
[
    [
        [met, roof]
    ]
    [
        [wat jij hebt gekocht, roof]
    ]
]
\end{forest}



Then in the main clause \tit{gekocht} is merged, which selects for an accusative object. Somehow \tit{waarmee} can satisfy that requirement.



\tit{Schildert} wants the instrumental PP, so the embedded clause looks like this:

\ex. [waar jij mee schildert]

Here \tit{wat} is part of the embedded clause but \tit{met} is not. It belongs to the main clause. An embedded clause is finished before the main clause is built. At no point in the derivation can \tit{met} \tit{wat} form a proper constituent.



In conclusion:

\ex.
\a. *[[met] [wat]] $\rightarrow$ [waarmee]
\b. [met [wat X]]
\b. [met [wat [X]]]



How do I connect constituency and different spellouts of the instrumental? Nano: only constituents can spell out.









\section{internal structure of \tit{met wat} and \tit{waarmee}}





Van Riemsdijk said that R-pronouns are pronouns and they orginate in the complement of P. Due to \tsc{r}-movement they move to the spec of PP, and due to suppletion they change form, coincidentally to something identical to the locative. Full DPs cannot move to the spec of PP because they do not have access to that spec.

I claim that this movement is not a special r-movement, but a movement driven by spellout. The suppletion also follows from the regular spellout algorithm, and it is the locative that is inserted.


Koopman's account differ from Van Riemsdijk's in that she assumes \tsc{r}-pronouns obligatorily move to specPlace. The fact that locatives and \tsc{r}-pronouns are syncretic is not a surprise. They occur in the same structural configuration.

Place does not make a lot of sense in \tit{waarmee}, because we do not have a place here.

Abels says that this movement from the complement to the spec is not allowed. He argues that, instead, that R-words are base generated as the specifiers of distinguished class of zero-place prepositions. \tsc{r} is the complement of P and is subextracted from the complement of P (which is allowed, because it is subextraction and not extraction of it as a whole.)

I agree with Abels in that \tit{r} is substracted from the complement of the preposition. However, I say that \tit{r} is actually the pronoun. What he refers to as an \tsc{r}-word is in Dutch \tit{w-aa-}, which I also assume is merged later.




I do not care whether \tit{w-} and \tit{-a-} are merged before or after.

WHY DOES MEE MOVE? I DO NOT KNOW.
BUT IT DOES.
AND SO DOES WAAR.
waar maybe because it is remnant movement?




start with \tit{met wat}.


--here taking xxx apart--

--but first introduce a bit of theory--
















\section{Connecting \tit{wat} to \tit{waar} and \tit{met} to \tit{mee}}

So, \tit{waarmee} and \tit{met wat} realize the same features but their spellouts differ.

\ex.
\ag. Ik heb gekocht [waar jij mee schildert].\\
 I have bought where you with paint\\
 `I bought what you are painting with.'
\bg. Ik schilder met [wat jij hebt gekocht].\\
 I paint with what you have bought\\
 `I paint with what you bought.'

A few things should be noted here. First, position of the elements: the regular pronoun follows the adposition and the ʀ-pronoun precedes the preposition. Second, the shape of the wh-pronoun: the ʀ-pronoun differs from the non-ʀ-pronoun in the lengthening of the vowel that the final consonant is \tit{r}. Third, the shape of the adposition: \tit{met} and \tit{tot} change in \tit{mee} and \tit{toe} when they combine with ʀ-pronouns.

Now let's look at it like this: \tit{waarmee} realizes particular features. Say \tit{waar} expresses A and B and \tit{mee} expresses C and D. Say, A B C and D. \tit{met wat} realizes the same features. Saying that \tit{wat} also realizes A and B and \tit{met} also C and D misses the point of having different realizations. So instead, say \tit{wat} only realizes A and \tit{met} realizes B, C and D. Or \tit{wat} realizes A, B and C and \tit{met} only realizes D. In both cases all features are expresses, but they are distributed differently, allowing for different realizations.\footnote{I did not address the ordering difference of the elements. Yes, this can lead to a difference in meaning.

\ex.
\ag. Ik klim in de boom.\\
 I climb in the tree\\
 `I am climbing in the tree.'
\bg. Ik klim de boom in\\
 I climb the tree in\\
 `I am climbing into the tree.'

Here, however, I will argue that the movement is driven by spellout and is meaningless.}

\begin{table}[ht]
	\center
	\caption {Distribution options}
	\begin{minipage}{0.27\linewidth}
		\begin{tabularx}{\textwidth}{cccc}
		\toprule
    \tsc{a}   & \tsc{b}            & \tsc{c}   & \tsc{d}                           \\
		\midrule
    \multicolumn{2}{c}{\tit{met}}  & \multicolumn{2}{c}{\cellcolor{Gray}\tit{wat}} \\
    \tit{mee} & \multicolumn{3}{c}{\cellcolor{Gray}\tit{waar}}                     \\
    \multicolumn{3}{c}{\tit{mee}}               & \cellcolor{Gray}{\tit{waar}}     \\
    \bottomrule
\end{tabularx}
\end{minipage}
\end{table}

I will argue for the last option, that \tit{met} spells out fewer features than \tit{mee}. The main argument for that comes from the fact that Dutch is a non-preposition stranding language, so prepositions cannot be moved out of the PP. Also, you cannot move the complement of a P to its spec, but you can subextract from them (Abels). And I show that ʀ-pronouns behave like clitic in how high they move.

In a nutshell, if all features are merged at once, we will always get an ʀ-pronoun. Due to spellout driven movement the ʀ-pronoun strands the features to be realizes by a preposition and a little more. The ʀ-pronoun is a clitic and moves high. The preposition + a little more realizes as \tit{mee} in case of \tit{waarmee}. Now if the features are not merged all at once, the regular spellout driven movements cannot take place, and we see \tit{met wat}. So, the answer of how constituency connects to change in phonological form lies in one of the core assumptions of nanosyntax: phrasal spellout spells out constituents.

\ex.
\a. \tit{waar} = w + a + r
\b. \tit{wat} = w + a + t

So the difference betwween \tit{t} and \tit{r} was actually the only difference we need to account for.

Dutch is a non P-stranding language.

\ex.
\ag. Met wie heb jij gedanst?\\
 with who have you danced\\
 `Whom did you dance with?'
\bg. *Wie heb jij gedanst met?\\
 who have you danced with\\
 `Whom did you dance with?'

Abels shows that extraction from a PP is not categorically excluded. It happens in Slavic.

--examples here--

What if ʀ-pronouns are actually clitic? And they pattern with Slavic in that they clitic can never follow Ps.

Do we have more reasons to believe that ʀ-pronouns are clitics? The non-wh ʀ-pronouns in Dutch are able to move very high, just like clitics and unlike regular pronouns in Dutch.\footnote{\ref{ex:hemaan} can also be excluded by saying that \tit{aan} and \tit{hem} cannot split.}

\ex.
\ag. Ik heb veel geld aan de badkamer uitgegeven.\\
 I have much money on the bathroom spent\\
 `I spent a lot of money on the bathroom.'
\bg. Ik heb er veel geld aan uitgegeven.\\
 I have there much money on spent\\
 `I spent a lof of money on it.'

\ex.
\ag. Ik heb veel geld aan mijn zoon uitgegeven.\\
 I have much money on my son spent\\
 `I spent a lof of money on my son'
\bg. Ik heb veel geld aan hem uitgegeven.\\
 I have much money on him spent\\
 `I spent a lot of money on him.'
\bg. *Ik heb \tbf{hem} veel geld aan uitgegeven.\\
 I have him much money on spent\\
 `I spent a lof of money on it.'\label{ex:hemaan}
\z.
\z.

This has shown that \tit{er} is small, smaller than regular pronouns. So we go for this option:

\begin{table}[ht]
	\center
	\caption {\tit{er} < \tit{'t}}
	\begin{minipage}{0.27\linewidth}
		\begin{tabularx}{\textwidth}{cccc}
		\toprule
    \tsc{a}   & \tsc{b}            & \tsc{c}   & \tsc{d}                           \\
		\midrule
    \multicolumn{2}{c}{\tit{met}}  & \multicolumn{2}{c}{\cellcolor{Gray}\tit{wat}} \\
    \multicolumn{3}{c}{\tit{mee}}               & \cellcolor{Gray}{\tit{waar}}     \\
    \bottomrule
\end{tabularx}
\end{minipage}
\end{table}

As a result of \tit{waar} being small, \tit{mee} becomes big. And \tit{mee} spells out more features than \tit{met}.

Now the difference in distribution between \tit{met wat} and \tit{waarmee} is going to follow from spellout. If we get higher than dative case, we need prepositions. Making prepositions is a very 'costly' operation in the langauge, it is the last resort option in the spellout procedure. Dutch also has forms that are not only prepositions, but can also function as postpositions. They are less 'costly'. To avoid building a complex specifier, different spellout options are used. That is why we get the \tsc{r}-pronoun in the picture. \tsc{r}-pronouns can combine with these postpositions.


\section{Taking \tit{waarmee} and \tit{met wat} apart}

\tit{'t} `it' from \ref{ex:entryt} can be used as subject, direct object and indirect object.

\ex.
\ag. 't Staat in de hal.\\
 \tsc{3sg.n.nom} stands in the hallway\\
 `It is standing in the hallway.'
\bg. Ik zie 't.\\
 I see \tsc{3sg.n.acc}\\
 `I see it.'
\bg. Ik heb 't een klap gegeven.\\
 I have \tsc{3sg.n.dat} a hit given\\
 `I hit it.'

 Pronouns in other genders alternate between nominative (non-oblique) and accusative/dative (oblique) in these contexts.

 \ex.
 \ag. Hij staat in de hal.\\
  \tsc{3sg.m.nom} stands in the hallway\\
  `He is standing in the hallway.'
 \bg. Ik zie hem.\\
  I see \tsc{3sg.m.acc}\\
  `I see it.'
 \bg. Ik heb hem een klap gegeven.\\
  I have \tsc{3sg.m.acc} a hit given\\
  `I hit him.'

\ex. \begin{forest} boom
[\tsc{dat}P
    [\tsc{f}3]
    [\tsc{acc}P
        [\tsc{f}2]
        [\tsc{nom}P
            [\tsc{f}1]
            [\tsc{thingP}
                [\tsc{thing}, roof]
            ]
        ]
    ]
]
{\draw (.east) node[right]{⇔ \tit{(ə)t}}; }
\end{forest}\label{ex:entryt}

\tit{Er} can be used as a locative.

\exg. Ik ben er al geweest.\\
I am there already been\\
`I have already been there.'

\ex. \begin{forest} boom
[\tsc{loc}P
[\tsc{location}]
    [\tsc{thngP}
        [\tsc{thing}, roof]
    ]
]
{\draw (.east) node[right]{⇔ \tit{(ə)r}}; }
\end{forest}\label{ex:entryr}

\tit{Met} can be used for as instrumental or comitative.

\exg. Ik dans met hem.\\
 I dance with him\\
 `I am dancing with him.'

\ex. \begin{forest} boom
[\tsc{com}P
    [\tsc{f}5]
    [\tsc{ins}P
        [\tsc{f}4, roof]
    ]
]
{\draw (.east) node[right]{⇔ \tit{met}}; }
\end{forest}\label{ex:entrymet}

Either way, it combines with accusative/dative (oblique), which can be seen on the pronouns. But only for the comitative, because for the instrumental we are getting an ʀ-pronoun. Full DPs do not show any marking, leading me to postulate the zero marker up to the dative.

\ex.
\ag. Ik dans met de man.\\
 I dance with the man\\
 `I am dancing with the man.'
\bg. Ik schilder met een kwast.\\
 I paint with een brush\\
 `I am painting with een brush.'

\ex. \begin{forest} boom
[\tsc{datP}
   [\tsc{f}3]
   [\tsc{acc}P
       [\tsc{f}2]
       [\tsc{nom}P
           [\tsc{f}1]
           [\phantom{x}]
       ]
   ]
]
{\draw (.east) node[right]{⇔ \tit{∅}}; }
\end{forest}

So \tit{met} expresses \tsc{f}4, and \tit{(ə)r} expresses \tsc{thing} and \tsc{f}1 to \tsc{f}1. \tit{(ə)r} only expresses \tsc{thing}. \tit{Ermee} and \tit{met 't} express the same features, which leaves \tsc{f}1 to \tsc{f}4 for \tit{mee}.

\begin{table}[ht]
	\center
	\caption {\tit{met 't} and \tit{ermee}}
	\begin{minipage}{0.38\linewidth}
		\begin{tabularx}{\textwidth}{ccccc}
		\toprule
    \tsc{f}4   & \tsc{f}3 & \tsc{f}2  & \tsc{f}1                 & \tsc{thing}                 \\
		\midrule
    \tit{met}  & \multicolumn{4}{c}{\cellcolor{Gray}\tit{(ə)t}}                                \\
    \multicolumn{4}{c}{\tit{mee}}                                & \cellcolor{Gray}\tit{(ə)r}  \\
    \bottomrule
\end{tabularx}
\end{minipage}
\end{table}

\ex. \begin{forest} boom
[\tsc{com}P
    [\tsc{f}5]
    [\tsc{insP}
        [\tsc{f}4]
        [\tsc{dat}P
            [\tsc{f}3]
            [\tsc{acc}P
                [\tsc{f}2]
                [\tsc{nom}P
                    [\tsc{f}1]
                    [\phantom{x}]
                ]
            ]
        ]
    ]
]
{\draw (.east) node[right]{⇔ \tit{mee}}; }
\end{forest}\label{ex:entrymee1}

\tit{a}/\tit{aa} is distal, \tit{ie}/\tit{i} is proximal. Medial?

\ex.
\ag. h-ie-r\\
 here\\
 `here'
\bg. d-aa-r\\
 there\\
 `there'
\bg. d-i-t\\
 this\\
 `this'
\bg. d-a-t\\
 that\\
 `that'
\z.
\z.

\ex. \begin{forest}
[WP
    [W, roof]
]
{\draw (.east) node[right]{⇔ \tit{w}}; }
\end{forest}\label{ex:entryw}

WP is the interrogative/wh, whatever.

\ex. \begin{forest}
[\tsc{distP}
    [\tsc{deix3}]
    [\tsc{medP}
        [\tsc{deix2}]
        [\tsc{deix1}]
    ]
]
{\draw (.east) node[right]{⇔ \tit{a}}; }
\end{forest}\label{ex:entrya}
\z.









\section{Derivation}

Spellout

\ex. Merge F and
 \a. Spell out FP
 \b. If (a) fails, attempt movement of the spec of the complement of \tsc{f}, and retry (a)
 \b. If (b) fails, move the complement of \tsc{f}, and retry (a)

\ex. Cyclic Override\\
Lexicalisation at a node XP overrides any previous match at a phrase contained in XP.

\ex. Backtracking\\
When spellout fails, go back to the previous cycle, and try the next option for that cycle.

The functional sequence of \tit{ermee}.

\ex. \tsc{f}4 > \tsc{f}3 > \tsc{f}2 > \tsc{f}1 > \tsc{thing}

We start with \tsc{thing}. This is realized as \tit{(ə)r}, because it has less junk than \tit{(ə)t}.

\ex.
\begin{forest} boom
 [\tsc{thingP}
     [\tsc{thing}, roof]
 ]
{\draw (.east) node[right]{⇔ \tit{(ə)r}}; }
\end{forest}

\tsc{f}1 is merged. \tit{(ə)r} is no longer a candidate, but \tit{(ə)t} is. So the spellout is overridden.

\ex. \begin{forest} boom
[\tsc{nom}P
   [\tsc{f}1]
   [\tsc{thingP}
       [\tsc{thing}, roof]
   ]
]
{\draw (.east) node[right]{⇔ \tit{(ə)t}}; }
\end{forest}

\tsc{f}2 is merged, which can still be realizd by \tit{(ə)t}.

\ex. \begin{forest} boom
[\tsc{accP}
   [\tsc{f}2]
   [\tsc{nom}P
       [\tsc{f}1]
       [\tsc{thingP}
           [\tsc{thing}, roof]
       ]
   ]
]
{\draw (.east) node[right]{⇔ \tit{(ə)t}}; }
\end{forest}

Idem for \tsc{f}3.

\ex. \begin{forest} boom
[\tsc{datP}
    [\tsc{f}3]
    [\tsc{accP}
       [\tsc{f}2]
       [\tsc{nom}P
           [\tsc{f}1]
           [\tsc{thingP}
               [\tsc{thing}, roof]
           ]
       ]
    ]
]
{\draw (.east) node[right]{⇔ \tit{(ə)t}}; }
\end{forest}

Now \tsc{f}4 is merged. No candidate for phrasal spellout, no candidate after complement movement.

\ex.
\a. \begin{forest} boom
[\tsc{insP}
    [\tsc{f}4]
    [\tsc{datP}
        [\tsc{f}3]
        [\tsc{accP}
           [\tsc{f}2]
           [\tsc{nom}P
               [\tsc{f}1]
               [\tsc{thingP}
                   [\tsc{thing}, roof]
               ]
           ]
        ]
    ]
]
{\draw (.east) node[right]{⇔ }; }
\end{forest}
\b. \begin{forest} boom
[\phantom{x}
    [\tsc{datP}
        [\tsc{f}3]
        [\tsc{accP}
           [\tsc{f}2]
           [\tsc{nom}P
               [\tsc{f}1]
               [\tsc{thingP}
                   [\tsc{thing}, roof]
               ]
           ]
        ]
    ]
    {\draw (.east) node[right]{⇔ \tit{(ə)t}}; }
    [\tsc{insP}
        [\tsc{f}4]
    ]
    {\draw (.east) node[right]{⇔ }; }
]
\end{forest}

The next step in the algorithm is backtracking. We try the next option for the cycle in which \tsc{f}3 was merged. This option is complement movement. However, there is no lexical entry that fits, so we try the next option.

show here

The exact same happens for \tsc{f}2.

show here

Now for \tsc{f}1 it is different. There is a candidate when we try complement movemement, namely \tsc{thing} is realized as \tit{(ə)r} and \tsc{f}1 is spelled out as \tit{∅}

\ex. \begin{forest} boom
[\phantom{x}
   [\tsc{thingP}
       [\tsc{thing}, roof]
   ]
   {\draw (.east) node[right]{⇔ \tit{(ə)r}}; }
   [\tsc{nom}P
       [\tsc{f}1]
   ]
   {\draw (.east) node[right]{⇔ \tit{∅}}; }
]
\end{forest}

Merge \tsc{f}2, there is no candidate for phrasal spellout, we try spec to spec movement and get a match:

\ex. \begin{forest} boom
[\phantom{x}
   [\tsc{thingP}
       [\tsc{thing}, roof]
   ]
   {\draw (.east) node[right]{⇔ \tit{(ə)r}}; }
   [\tsc{acc}P
       [\tsc{f}2]
       [\tsc{nom}P
           [\tsc{f}1]
       ]
   ]
   {\draw (.east) node[right]{⇔ \tit{∅}}; }
 ]
\end{forest}

Idem for \tsc{f}3.

\ex. \begin{forest} boom
[\phantom{x}
   [\tsc{thingP}
       [\tsc{thing}, roof]
   ]
   {\draw (.east) node[right]{⇔ \tit{(ə)r}}; }
   [\tsc{dat}P
       [\tsc{f}3]
       [\tsc{acc}P
           [\tsc{f}2]
           [\tsc{nom}P
               [\tsc{f}1]
           ]
       ]
   ]
   {\draw (.east) node[right]{⇔ \tit{∅}}; }
]
\end{forest}

Now \tsc{f}4 is merged. No phrasal spellout, but there is a match after spec-to-spec movement. However, is is not \tit{∅} anymore, but the inserted morpheme is \tit{mee}.

\ex. \begin{forest} boom
[\phantom{x}
    [\tsc{thingP}
       [\tsc{thing}, roof]
    ]
    {\draw (.east) node[right]{⇔ \tit{(ə)r}}; }
    [\tsc{insP}
       [\tsc{f}4]
       [\tsc{dat}P
           [\tsc{f}3]
           [\tsc{acc}P
               [\tsc{f}2]
               [\tsc{nom}P
                   [\tsc{f}1]
               ]
           ]
       ]
    ]
    {\draw (.east) node[right]{⇔ \tit{mee}}; }
]
\end{forest}

Then show here that this does not happen if not all features form a proper constituent. Spellout can just not target a constituent.



Open issues:

\begin{itemize}
  \item I still need to incorporate \tit{a} and \tit{w}. They can either come before case or after it.
  \item Right now I do not capture the formal similarity between \tit{met} and \tit{mee}. I have ideas about that:
  \item Some adpositions do not change shape. They should be usable both as prepositions and postpositions. This should be reflected in their lexical entry. A prediction is that there exist adpositions that cannot be used as postpositions. Examples of that are: \tit{zonder}, \tit{midden}, \tit{sinds}, \tit{rond}, \tit{beneden}.
\end{itemize}

\section{Conclusion}

\printbibliography

\end{document}
