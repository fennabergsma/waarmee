\documentclass[12pt]{article}

\usepackage[margin=1in]{geometry}
%\usepackage{setspace}
%\doublespacing

\usepackage{fenna-files/packages}
\usepackage{fenna-files/commands}
\bibliography{fenna-files/references}{}

\title{How [preposition inanimate] becomes [\tsc{r}-pronoun postposition]\\
On \tit{waarmee} vs. \tit{met wat} in Dutch}
\author{Fenna Bergsma}
\date{\today}

\begin{document}

\maketitle



\section{Introduction}

English has a sentence. This is ungrammatical in Dutch. We need an \tsc{r}-pronoun. The term \tsc{r}-pronoun \citep{riemsdijk1978} refers to a set of nominal elements that can strand prepositions in Dutch (and German). This is noteworthy because Dutch is a normally non-preposition stranding language.
In the example \ref{ex:klimerop} \tit{op} `on' is the preposition and \tit{'r} `there' is the locative pronoun.\footnote{The \tsc{r}-pronoun \tit{'r} in \ref{ex:klimerop} can be written as \tit{er}, \tit{der} and \tit{'r} and pronounced as respectively /ɛr/, /dər/ or /ər/. As far as I am aware, there is no clear meaning difference between these forms. See \citet{wesseling2018} for discussion. In my examples I use \tit{'r}, but the other two forms fit just as well.}

\ex. I climb on it.

\ex.
\ag. *Ik klim op 't.\\
 I climb on it\\
 `I climb on it.'\label{ex:klimopt}
\bg. Ik klim 'r op.\\
 I climb there on\\
 `I climb on it.'\label{ex:klimerop}

In Dutch it is impossible to have [P inanimate]. But why? \citet{riemsdijk1978} postulating a filter that prohibits the existence of constituents consisting of prepositions and inanimates + some kind of allomoprhy. \citet{koopman2000} who makes reference to a different paradigm. Abels: base-generated somewhere else.

This chapter is an in-depth study of the \tsc{r}-pronoun-postposition combination \tit{waarmee} `with what' in Dutch. This instance is interesting for two reasons. First, just like for all \tsc{r}-pronouns, the wh-element is the locative, but there is no meaning component related to location in `with what'. Second, the preposition \tit{met} `with' does not only turn into a postposition, but it also changes into \tit{mee} `with' when it is combined with an \tsc{r}-pronoun.\footnote{I do not have anything to say about the distinction between prefixes and prepositions, or suffixes and postpositions.} This last observervation has so far remained unexplained.

In answering the question why [P inanimate] I look at an exception to the rule that *P inanimate pronoun. That exception is a mismatching free relative. Here the string \tit{met wat} does appear.

what I will show that r-pronouns are not a different construction to express the same meaning, but they realize the same features as regular P + pronoun combinations. It is a kind of allomorphy, just like Van Riemsdijk said. And not a different way to express the same function.

Evidence from that comes from the complementary distribution of \tit{waarmee} and \tit{met wat} in mismatching free relatives.

This is a free relative construction in which the two predicates (the one in the main clause and the one in the embedded clause) combine with two different cases (i.e. the case requirements do not match). I illustrate this in \ref{ex:gekochtwaarmee}. The predicate in the embedded clause, \tit{schildert} `paint', combines with an instrumental object. The predicate in the main clause clause, \tit{gekocht} `bought' combines with an accusative DP. The \tsc{r}-pronoun \tit{waarmee} `with what' is used here.\footnote{In this example, \tit{waar} `where' takes \tit{mee} `with' to the left edge of the embedded clause. It is also possible for \tit{mee} `with' to be stranded, and \tit{waar} `where' to be moved to the left edge of the embedded clause on its own.

\exg. Ik heb gekocht waar jij mee schildert.\\
 I have bought waar you with paint\\
 `I bought what you are painting with.'\label{ex:meealong}
\z.

\phantom{x}
}

\exg. Ik heb gekocht waarmee jij schildert.\\
 I have bought {where with} you paint\\
 `I bought what you are painting with.'\label{ex:gekochtwaarmee}

If the predicates are switched around between the clauses, the \tsc{r}-pronoun does not appear anymore. In \ref{ex:schildermet}, \tit{schilder} `paint' combines with an instrumental object in the main clause and \tit{gekocht} `bought' combines with an accusative object in the embedded clause. The use of an \tsc{r}-pronoun is ungrammatical, as indicated by the ungrammaticality of \ref{ex:schilderwaarmee}. Instead, a combination of the regular instrumental preposition \tit{met} `with' and the regular wh-pronoun \tit{was} `what' in used.

\ex.\label{ex:schildermet}
\ag. *Ik schilder waarmee jij hebt gekocht.\\
 I paint {where with} you have bought\\
 `I paint with what you bought.'\label{ex:schilderwaarmee}
\bg. Ik schilder met wat jij hebt gekocht.\\
 I paint with what you have bought\\
 `I paint with what you bought.'\label{ex:schildermetwat}

The use of \tit{met wat} `with what' is ungrammatical in the context in which \tit{waarmee} `with what' appeared in \ref{ex:gekochtwaarmee}. This is illustrated in \ref{ex:gekochtmetwat}.

\exg. *Ik heb gekocht met wat jij schildert.\\
 I have bought with what you paint\\
 `I bought what you are painting with.'\label{ex:gekochtmetwat}

In this paper I show that distribution of \tit{waarmee} `with what' and \tit{met wat} `with what' in these free relative constructions gives us a good insight into the internal structure of \tsc{r}-pronouns. In what follows I show that \tsc{r}-pronouns and regular preposition compete to spell out the same syntactic features. If all features form a proper constituent (i.e. a constituent to the exclusion of other features), the \tsc{r}-pronoun surfaces. If it is not a proper constituent, the preposition-pronoun combination shows up. This straightforwardly follows in a system in which spellout targets phrasal constituents: Nanosyntax \citep{starke2009}. --here more about the internal structure of r-pronouns, and the postposition preposition distinction--

This paper is structured as follows. In Section \ref{sec:distribution} I discuss the distribution of \tit{waarmee} `waarmee' and \tit{met wat} `with what'. I show that the \tsc{r}-pronoun surfaces when it forms a proper constituent, and \tit{met wat} `with what' when it does not. I decompose \tit{waarmee} `with what' and \tit{met wat} `with what' to make sense of the phonological similarities, and I provide a more detailed analysis. Unmarked examples are constructed and have been verified by native speakers.


\section{The distribution between \tit{waarmee} and \tit{met wat}}\label{sec:distribution}

In the introduction I discussed the distribution between \tit{waarmee} `with what' and \tit{met wat} `with what' in free relatives with predicates that combine with different cases. Table \ref{tbl:distribution} repeats the generalization. When the main clause predicate combines with an accusative and the embedded clause predicate with an instrumental, \tit{waarmee} `with what' is grammatical and \tit{met wat} `with what' is ungrammatical. When the main clause predicate combines with an instrumental and the embedded clause predicate with an accusative, \tit{waarmee} `with what' is ungrammatical and \tit{met wat} `with what' is used.

\begin{table}[ht]
	\center
	\caption {Distribution between \tit{waarmee} and \tit{met wat}}
	\begin{minipage}{0.45\linewidth}
		\begin{tabularx}{\textwidth}{ccc}
		\toprule
                              & \tit{waarmee} & \tit{met wat} \\
		\midrule
    m:\tsc{acc}, e:\tsc{ins}  & ✔             & *             \\
    m:\tsc{ins}, e:\tsc{acc}  & *             & ✔             \\
    \bottomrule
\end{tabularx}
\label{tbl:distribution}
\end{minipage}
\end{table}

In this section I first show that \tsc{r}-pronouns are the default complement of a preposition. Next, I illustrate that a necessary requirement for an \tsc{r}-pronoun is that is forms a proper constituent.


\subsection{\tsc{R}-pronouns as default}\label{sec:rdefault}

The goal of this section is to show that \tit{waarmee} `with what' is the default as instrumental relative pronoun. This generalization is not new, it has already been made \citet{riemsdijk1978,koopman2000}. In order to show that \tit{waarmee} `with what' is the default, I discuss the distribution of \tsc{r}-pronouns and regular pronouns in more general. I start with the personal pronouns and then return to the wh-pronouns.

Dutch has the personal pronouns \tit{haar} `her', \tit{hem} `him' and \tit{het} `it' that can be used as animate and inanimate objects of verbs, as illustrated in \ref{ex:objverb}.

 \ex. \label{ex:objverb}
 \ag. Ik zie haar/hem.\\
  I see her/him\\
  `I see her/him.'\label{ex:aniobj}
 \bg. Ik zie 't.\\
  I see it\\
  `I see it.'\label{ex:inaniobj}

The example in \ref{ex:prepani} shows that for animate objects the same pronouns (\tit{haar} `her' and \tit{hem} `him') appear as objects of prepositions. However, the inanimate personal pronoun \tit{het} `it' cannot be used as an object of a preposition, shown in \ref{ex:prephet}. Instead, an ʀ-pronoun appears. This is illustrated in \ref{ex:preper}. \ref{ex:erprep} shows that the \tsc{r}-pronoun obligatorily moves to the left of the pronoun.

\ex. \label{ex:objprep}
\ag. Ik schilder samen met haar/hem.\\
 I paint together with her/him\\
 `I am painting together with her/him.'\label{ex:prepani}
\bg. *Ik schilder met 't.\\
 I paint with it\\
 `I am painting with it.'\label{ex:prephet}
\bg. Ik schilder 'r mee.\\
 I paint there-with\\
 `I am painting with it.'\label{ex:preper}
\bg. *Ik schilder mee 'r.\\
 I paint with there\\
 `I am painting with it.'\label{ex:erprep}

\tit{Met} is not the only preposition with which this happens. \tit{Op} `on' and \tit{in} `in' do not combine with the inanimate personal pronoun \tit{'t}, but the \tsc{r}-pronoun is used obligatorily.

\ex.
\ag. Ik zit 'r op.\\
 I sit there on\\
 `I am sitting on it.
\bg. *Ik zit op 't.\\
 I sit on it\\
 `I am sitting on it.

\ex.
 \ag. Hij zwemt 'r in.\\
  he swims it-in\\
  `He is swimming in it.'
 \bg. *Hij zwemt in 't.\\
  he swims in it\\
  `He is swimming in it.'

The situation of the inanimate wh-pronouns resembles the inanimate personal pronouns. \tit{Wat} `what' can function as an object of a verb (see \ref{ex:wat}), but not as an object of a preposition \ref{ex:metwat}. In that case, the \tsc{r}-pronoun \tit{waarmee} `with what' is appears, as shown in \ref{ex:waarmee}.\footnote{The sentence in \ref{ex:metwat} is unacceptable with neutral intonation. It becomes is only acceptable if \tit{wat} `what' is stressed, for example in a context in which the speaker is highly surprised about the choice for the object hearer is painting with.}

\ex.
\ag. Wat zie jij?\\
 what see you\\
 `What do you see?'\label{ex:wat}
\bg. *Met wat schilder jij?\\
 with what paint you\\
 `What are you painting with?'\label{ex:metwat}
\bg. Waarmee schilder jij?\\
 {where with} paint you with\\
 `What are you painting with?'\label{ex:waarmee}

\tit{Waarmee} `with what' and not \tit{met wat} `with what' does not only appear in wh-questions, but also in other contexts. \ref{ex:headed} gives an example of a headed relative, and \ref{ex:headless} shows a free relative in which both predicates combine with an instrumental object. The use of \tit{met wat} `with what' is ungrammatical in both contexts, and \tit{waarmee} `with what' is used.

\ex.\label{ex:headed}
\ag. Ik schilder met de kwast waarmee jij ook schildert.\\
 I paint with the brush {where with} you also paint\\
 `I am painting with the brush that you are painting with too.'
\bg. *Ik schilder met de kwast met wat jij ook schildert.\\
 I paint with the brush with what you also paint\\
 `I am painting with the brush that you are painting with too.'

 \ex.\label{ex:headless}
 \ag. Ik schilder waarmee jij ook schildert.\\
  I paint {where with} you also paint\\
  `I am painting with what you are painting with too.'
 \bg. *Ik schilder met wat jij ook schildert.\\
  I paint with what you also paint\\
  `I am painting with what you are painting with too.'

\tit{'t} `it' and \tit{wat} `what' do not combine with prepositions. They are substituted by respectively \tit{'r} `there' and \tit{waar} `where'.

The next section discusses the role of constituency in \tsc{r}-pronouns.


\subsection{\tit{Waarmee} is a constituent, \tit{met wat} is not}

Let me now return to the mismatching free relatives. I repeat the relevant grammatical examples in \ref{ex:grammatical}.

\ex.\label{ex:grammatical}
\ag. Ik heb gekocht waarmee jij schildert.\\
 I have bought {where with} you paint\\
 `I bought what you are painting with.'\label{ex:grammaticalwaar}
\bg. Ik schilder met wat jij hebt gekocht.\\
 I paint with what you have bought\\
 `I paint with what you bought.'\label{ex:grammaticalwat}

In this section I showed that ʀ-pronouns are expected in combinations with prepositions. This means that the use of \tit{waarmee} `with what' \ref{ex:grammaticalwaar} is not surprising. Something that is surprising is the use of \tit{met wat} `with what' in \ref{ex:grammaticalwat}, and this is the example something more needs to be said about. In the remainder of this section I argue that this `something more' is that the instrumental object in \ref{ex:grammaticalwat} does not form a proper constituent, i.e. it is not a constituent to the exclusion of any other elements. The other side of the coin is that constructions with \tsc{r}-pronouns contain a object that does form a proper constituent.

Below I repeat the examples with instrumentals I discussed so far in this paper.

\ex.
\ag. Ik schilder 'r mee.\\
 I paint there with\\
 `I am painting with it.'\label{ex:const1}
\bg. Waarmee schilder jij?\\
{where with} paint you with\\
 `What are you painting with?'\label{ex:const2}
\bg. Ik schilder met de kwast [waarmee jij ook schildert].\\
 I paint with the brush {where with} you also paint\\
 `I am painting with the brush that you are painting with too.'\label{ex:const3}
\bg. Ik schilder [waarmee jij ook schildert].\\
 I paint {where with} you also paint\\
 `I am painting with what you are painting with too.'\label{ex:const4}

In each of these examples the instrumental object forms a constituent at a certain point in the derivation. In \ref{ex:const1}, the instrumental object forms a proper constituent in the surface order, as shown in \ref{ex:const1stage}. In \ref{ex:const2}, the instrumental object forms a proper constituent before wh- and V2- movement, shown in \ref{ex:const2stage}.
The structure in \ref{ex:const3stage} represents a stage in the derivation of the embedded clauses in \ref{ex:const3} and \ref{ex:const4}.\footnote{I assume that the antecedent in a free relative is a phonologically empty element, in line with cf. \cite{bresnan1978a,groos1981,himmelreich2017}. Under that view, \ref{ex:const3} and \ref{ex:const4} are identical, except for that in \ref{ex:const4} modifies phonologically empty element.}
Again, in the stage, which comes before relative movement of the pronoun to the left periphery of the relative clause, the instrumental object forms a proper constituent.

\ex.
\a. [[ik] [[schilder] ['r mee]]]\label{ex:const1stage}
\b. [[jij] [[schilder] [waarmee]]]\label{ex:const2stage}
\b. [[jij] [[ook] [[schilder] [waarmee]]]]\label{ex:const3stage}

%There is additional evidence for the fact that \tit{waarmee} `with what' forms a constituent in the constructions above. In the example in \ref{ex:const2}-\ref{ex:const4} only \tit{waar} `where' was fronted, but \ref{ex:constwaarmee} shows that also the phrase containing \tit{mee} with' can be moved. As wh-movement can only target constituents, it follows that \tit{waar} `where' and \tit{mee} `with' have to form a constituent.

%\ex.\label{ex:constwaarmee}
%\ag. Waar-mee schilder jij?\\
 %{where-with} paint you\\
 %`What are you painting with?'
%\bg. Ik schilder [waar-mee jij ook schildert].\\
 %I paint {where-with} you also paint\\
 %`I am painting with what you are painting with too.'
%\bg. Ik schilder met de kwast [waar-mee jij ook schildert].\\
 %I paint with the brush {where-with} you also paint\\
 %`I am painting with the brush that you are painting with too.'

The mismatching free relative in \ref{ex:grammaticalwat} is not the only construction in which the string \tit{met wat} `with what' appears. I give examples of two more occurrences in \ref{ex:moremetwat}. In \ref{ex:watindef}, \tit{wat} `what' is the \tit{wat} `what' in the so-called \tit{wat voor} `what for'-construction \citep[cf.][]{corver1991}.
In \ref{ex:watindef}, \tit{wat} appears as a quantifier, and it means `some'. In both construction \tit{wat} `what' takes a complement and \tit{met wat} `with what' do not form a proper constituent. The brackets within the examples indicate the constituency.

\ex.\label{ex:moremetwat}
\ag. [Met [wat [voor [potloden]]] teken jij?\\
 with what for pencils draw you\\
 `What kind of pencils do you with?'\label{ex:watwasfur}
\bg. Ik wil graag thee [met [wat [suiker]]].\\
 I want please tea with some sugar\\
 `I would like to have tea with some sugar.'\label{ex:watindef}

Let me now show how this applies to the examples with the mismatching free relatives. The two predicates I used in the free relatives are \tit{kopen} `to buy' and \tit{schilderen}  `to paint'. \tit{Kopen} `to buy' takes an accusative DP as its object, illustrated in \ref{ex:kopen}. \tit{Schilderen} `to paint' can take an instrumental as its object., shown in \ref{ex:schilderen}.\footnote{\tit{Schilderen} also optionally takes an (accusative) object, but I am focussing on the instrumental object here.}

\ex.
\ag. Ik koop het schilderij.\\
 I buy the painting\\
 `I am buying the painting.'\label{ex:kopen}
\bg. Ik schilder met een kwast.\\
 I paint with a brush\\
 `I am painting with a brush.'\label{ex:schilderen}

I repeat the mismatching free relative in which \tit{waarmee} `met wat' appears in \ref{ex:mismatchwaarmee}. The predicate \tit{schildert} `paints' combines in the embedded clause with the instrumental object. The instrumental object forms a proper constituent within the embedded clause, and the it can be realized as the \tsc{r}-pronoun and postposition \tit{waarmee} `with what'.\footnote{I assume that the embedded clause modifies a phonologically empty noun here which is the DP argument of \tit{gekocht} `bought'.}

\exg. Ik heb gekocht [waarmee jij schildert].\\
 I have bought {where with} you paint\\
 `I bought what you are painting with.'\label{ex:mismatchwaarmee}

Next, we arrive at the mismatching free relative in which \tit{waarmee} `with what' cannot be used, but \tit{met wat} `with what' appears. The embedded clause predicate \tit{gekocht} `bought' combines with an accusative DP. The accusative object of a verb is always \tit{wat} `what', as I showed in \ref{ex:wat}. The instrumental only comes into the picture in the main clause, when \tit{schilder} `paint' combines with an instrumental object. At no point in the derivation does the instrumental object form a proper constituent, and \tit{waarmee} `with what' does not surface.

\exg. Ik schilder met [wat jij hebt gekocht].\\
 I paint with what you have bought\\
 `I paint with what you bought.'\label{ex:mismatchmetwat}

\ref{ex:summaryconst} summarizes what I showed in this section. \tit{Met wat} `with what' can never surface when \tit{met} `with' and \tit{wat} `what' form a proper constituent. It always becomes \tit{waarmee} `with what'. This is schematically shown in \ref{ex:waarmeefr}.
There are other contexts in which \tit{met wat} `with what' appears. This can be either when \tit{wat} `what' takes a complement, or when \tit{wat} is part of the a clause that \tit{met} `with' is not a part of. This last option is schematically showed in \ref{ex:metwatfr}, and it represents the mismatching free relative in \ref{ex:mismatchmetwat}.

\ex.\label{ex:summaryconst}
\a. [[met] [wat]] → [waarmee]\label{ex:waarmeefr}
\b. [met [wat [X]]]
\b. [met [[wat] [X]]]\label{ex:metwatfr}

\tit{Met} `with' is a preposition that combines with full DPs and animate pronouns. \tit{Wat} `what' is a wh-element that appears as subject or object.

In the next section I decompose \tit{waarmee} `with what' and \tit{met wat} `with what'. I show that both spell out the same set of features, but the distribution is different.



\section{Taking \tit{waarmee} and \tit{met wat} apart}

Along the way I introduce some background

Things to explain
*P pronoun-inanimate

\tit{r} is the locative
\tit{met} changes form

 The idea is that both expressions realize the same featu♦res, and \tit{waarmee} `with what' takes precedence when all features form a constituent. The morphemes I distinguish in \tit{waarmee} `with what' are \tit{w}, \tit{-aa-}, \tit{-r} and \tit{mee}. Within \tit{met wat} `with what' I distinguish \tit{met}, \tit{w}, \tit{a} and \tit{-t}. \ref{ex:decompose} shows this as well.

\ex.\label{ex:decompose}
\a. w -aa -r -mee
\b. met w -a -t

In this section I investigate the internal structure of \tit{waarmee} `with what' and \tit{met wat} `with what' to capture the phonological similarities and differences between the two forms. First, I identify \tit{w} and \tit{a} as morphemes that appear in both expressions. Putting these two aside, I concentrate on \tit{'rmee} `with it' and \tit{met 't} `with it'.

The elements \tit{w} and \tit{a} express the same syntactic structure in both expressions. The elements \tit{met} and \tit{'t} together also express the same features as \tit{mee} and \tit{'r} together, but the distribution differs. \tit{Met} expresses less structure than \tit{mee}, and \tit{'t} expresses more structure than \tit{'r}. \tit{Met} is shown to be an preposition, and \tit{mee} is a postposition. It also becomes clear that the \tit{'r} in fact corresponds to the locative \tit{r} in Dutch.




\subsection{Overlap: \tit{w-} and \tit{-a-}}


Let me start with the morphemes \tit{w} and \tit{a} that appear in both expressions. Exactly because they appear in both expression, I do not pay much attention to them. For \tit{w} I follow \citet{hachem2015} in that \tit{w} realizes a WP.\footnote{Throughout the paper, ⇔ indicates the pairing between a lexical tree and a phonological form in a lexical entry, and ⇒ indicates how a node in the syntactic structure is spelled out.}

\ex. \begin{forest}
[WP
    [W, roof]
]
{\draw (.east) node[right]{⇔ \tit{w}}; }
\end{forest}\label{ex:entryw}

I assume the morpheme \tit{a} to be related to deixis. Dutch distinguishes between proximal by using \tit{ie/i} and distal by using \tit{aa/a}, illustrated in \ref{ex:deixis}. I assume that lengthening or shortening of the vowel is a result of the final \tit{r}.

\ex.\label{ex:deixis}
\ag. h-ie-r\\
 here\\
\bg. d-aa-r\\
 there\\
\bg. d-i-t\\
 this\\
\bg. d-a-t\\
 that\\
 \z.
 \z.

The question remains open to why wh-elements only combine with the distal marker \tit{a}, and they do not combine with proximal marker \tit{i/ie}.

\ex.
\a. wat, waar
\b. *wit, *wier

For the purpose of this paper I simply let \tit{a} correspond to \tsc{deixP}.

\ex. \begin{forest}
[deixP
    [deix, roof]
]
{\draw (.east) node[right]{⇔ \tit{a}}; }
\end{forest}\label{ex:entrya}

I put \tit{w} and \tit{a} aside for now, assuming they spell out the same syntactic structure in \tit{waarmee} `with what' and \tit{met wat} `with what'. This leaves \tit{'r-mee} `with it' and \tit{met 't} `with it'.

\ex.
\ag. 'r -mee\\
there with\\
\b. met 't\\
with it\\





\subsection{Differences: \tit{'rmee} vs. \tit{met 't}}

Having put \tit{w-} and \tit{a} aside for now, the forms \tit{met 't} `with it` and \tit{'rmee} `with it' are left. Below I repeat examples from Section \ref{sec:rdefault} that show that \tit{met} `with` can express instrumental case, and \tit{'t} is an inanimate pronoun. \ref{ex:metdp} shows \tit{met} `with' combined with full DP. \ref{ex:tverb} shows \tit{'t} as the object of verbs.

\ex.
\ag. Ik schilder met een kwast.\\
 I paint with a brush\\
 `I am painting with a brush.'\label{ex:metdp}
 \bg. Ik zie 't.\\
  I see it\\
  `I see it.'\label{ex:tverb}

However, \tit{met} `with' and \tit{'t} `it' do not appear together if they form a proper constituent. Instead, \tit{'rmee} is used, as shown in \ref{ex:rmeeconst}.

\ex.\label{ex:rmeeconst}
\ag. Ik schilder 'r-mee.\\
 I paint there-with\\
 `I am painting with it.'\label{ex:jarmee}
\bg. *Ik schilder met 't.\\
 I paint with it\\
 `I am painting with it.'\label{ex:neemett}

In this section I set up an account that makes the ungrammaticality of \tit{met 't} and the appearance of \tit{'rmee} follow from spellout. The analysis accounts for the following two observations, taking \tit{met 't} as the point of departure. First, \tit{met} `with' changes from being a preposition to being a postposition, and its form changes into \tit{mee}. This process is restricted to inanimate pronouns, and it does not apply to full DPs and animate pronouns. Second, \tit{'t} is replaced by \tit{'r}, a morpheme that is associated with the locative in Dutch.

This section first discusses some basic facts about case and assumptions about Nanosyntax. Then I discuss the distinction between prepositiones and postpositiones, followed by a discussion on suppletion in pronouns. Finally, I specify the lexical entries for \tit{'rmee} `with it` and \tit{met 't} `with it` that account for the described facts.



\subsubsection{\tit{'t} vs. \tit{'r}}

Let me start with the lexical entry for \tit{'t}. \tit{'t} `it' can be used as subject (associated with nominative), direct object (associated in accusative) and indirect object (associated with dative), as shown in \ref{ex:tsubobj}.

\ex.\label{ex:tsubobj}
\ag. 't Staat in de hal.\\
 \tsc{3sg.n.nom} stands in the hallway\\
 `It is standing in the hallway.'\label{ex:tnoclitic}
\bg. Ik zie 't.\\
 I see \tsc{3sg.n.acc}\\
 `I see it.'
\bg. Ik heb 't een klap gegeven.\\
 I have \tsc{3sg.n.dat} a hit given\\
 `I gave it a hit.'

 Pronouns in other genders alternate between nominative (non-oblique) and accusative/dative (oblique) in these contexts, illustrated in \ref{ex:hemsubobj}.

 \ex.\label{ex:hemsubobj}
 \ag. Hij staat in de hal.\\
  \tsc{3sg.m.nom} stands in the hallway\\
  `He is standing in the hallway.'
 \bg. Ik zie hem.\\
  I see \tsc{3sg.m.acc}\\
  `I see it.'
 \bg. Ik heb hem een klap gegeven.\\
  I have \tsc{3sg.m.acc} a hit given\\ with both alternatives. I work the proposal out with't  realizing all cases.

For case, I follow \citet{caha2009} that case features case features are organized the containment relation in \ref{ex:casetree}. The higher, more complex cases contain the smaller, less complex cases.

 \ex. \label{ex:casetree}
 \begin{forest} boom
 [\tsc{comP}
     [\tsc{f5}]
     [\tsc{insP}
         [\tsc{f4}]
         [\tsc{datP}
             [\tsc{f3}]
             [\tsc{accP}
                 [\tsc{f2}]
                 [\tsc{nomP}
                     [\tsc{f1}]
                     [DP
                         [...,roof]
                     ]
                 ]
             ]
         ]
     ]
 ]
 \end{forest}

 Following the distinctions from \citet{cardinaletti1996}, \tit{'t} `it' is a weak pronoun. It is not a clitic, because it can occur in sentence initial position, shown in \ref{ex:tnoclitic}. It is not a strong pronoun, because it cannot be coordinated, as indicated in  \ref{ex:tcoordinated}. \ref{ex:datcoordinated} shows that \tit{'t} `it' needs to combine with \tit{da-}/\tit{di-} to be able to be coordinated.

 \ex.
 \ag. *Hij en 't staan in de hoek.\\
  he and it stand in the corner\\
  `He and it are standing in the corner.'\label{ex:tcoordinated}
 \bg. Hij en dit/dat staan in de hoek.\\
  he and this/that stand in the corner\\
  `He and it are standing in the corner.'\label{ex:datcoordinated}

I assume that the \tit{'t} contains the ontological category \tsc{thing} \citep{kayne2005}. The feature Σ indicates that the pronoun is a weak pronoun. I leave possible number and gender features out because they do not play a role in this paper. The morpheme \tit{'t} can act as nominative, accusative and dative, as I showed in \ref{ex:tsubobj}.\footnote{ Another possibility is to claim that \tit{'t} can only spell out \tsc{thing} and Σ and it combines with a zero suffix for the cases up to dative. This could be the same zero marker that full DPs combine with.

\ex.
\ag. De kast-∅ staat in de hal.\\
 the cabinet-\tsc{nom} stands in the hallway\\
 `The cabinet is standing in the hallway.'
\bg. Ik zie de kast-∅.\\
 I see the cabinet-\tsc{acc}\\
 `I see the cabinet.'
\bg. Ik heb de kast-∅ een klap gegeven.\\
 I have the cabinet-\tsc{dat} a hit given\\
 `I gave the cabinet a hit.'

The proposed account fares equally well with both alternatives. I work the proposal out with \tit{'t} realizing the cases up to the dative.} Taking this all together, \tit{'t} has the lexical entry given in \ref{ex:entryt}.

\ex. \begin{forest} boom
 [\tsc{dat}P
     [\tsc{f}3]
     [\tsc{acc}P
         [\tsc{f}2]
         [\tsc{nom}P
             [\tsc{f}1]
             [ΣP
                 [Σ]
                 [\tsc{thingP}
                     [\tsc{thing}, roof]
                 ]
             ]
         ]
     ]
 ]
 {\draw (.east) node[right]{⇔ \tit{'t}}; }
 \end{forest}\label{ex:entryt}

 This lexical entry can lexicalize the \tsc{datP}, but also the \tsc{accP} and \tsc{nomP}. This is due to the Superset Princple.

  \ex. The Superset Principle \citet{starke2009}: \\
  A lexically stored tree matches a syntactic node iff the lexically stored tree contains the syntactic node.

 In other words, a lexically stored structure does not have to be identical to the syntactic structure. It is enough if the syntactic structure is contained in the lexically stored tree. This has a as consequence that the lexical entry in \ref{ex:entryt} can also be inserted in \ref{ex:tacc} and \ref{ex:tnom}.

 \ex.
 \a. \begin{forest} boom
 [\tsc{acc}P
     [\tsc{f}2]
     [\tsc{nom}P
         [\tsc{f}1]
         [ΣP
             [Σ]
             [\tsc{thingP}
                 [\tsc{thing}, roof]
             ]
         ]
     ]
 ]
 \end{forest}\label{ex:tacc}
 \b. \begin{forest} boom
 [\tsc{nom}P
     [\tsc{f}1]
     [ΣP
         [Σ]
         [\tsc{thingP}
             [\tsc{thing}, roof]
         ]
     ]
 ]
 \end{forest}\label{ex:tnom}

Let me move on to \tit{'r}. \tit{'r} `there' can be used as a locative.

 \exg. Ik ben er al geweest.\\
  I am there already been\\
  `I have already been there.'

I follow \cite{baunaz2018} in assuming that the ontological category \tsc{location} contains \tsc{thing}.\footnote{\citet{baunaz2018} place in addition \tsc{person} between \tsc{thing} and \tsc{location}, which I left out here.}

\ex. \begin{forest} boom
[\tsc{loc}P
    [\tsc{location}]
    [\tsc{thingP}
        [\tsc{thing}, roof]
    ]
]
{\draw (.east) node[right]{⇔ \tit{'r}}; }
\end{forest}\label{ex:entryr}

Notice already here that, via the superset principle, \tit{'r} can be used to realize the feature \tsc{thing} as well, as it is contained in \tsc{locP}. Moreover, in a syntactic structure like in \ref{ex:rthing} the lexical entry \ref{ex:entryr} will be inserted and not \ref{ex:entryt}.

\ex.
\begin{forest} boom
 [\tsc{thingP}
     [\tsc{thing}, roof]
 ]
{\draw (.east) node[right]{⇒ \tit{'r}}; }
\end{forest}\label{ex:rthing}

This is due to the Elsewhere Condition. The idea is that when two lexical entries are both candidates for spellout, the most specific is inserted.

\ex. The Elsewhere Condition (\citealt{kiparsky1973}, formulated as in \citealt{caha2020}):\\
When two entries can spell out a given node, the more specific entry wins. Under the Superset Principle governed insertion, the more specific entry is the one which has fewer unused features

The syntactic structure in \ref{ex:entryr} only has \tsc{loc} as an unused feature, whereas in \ref{ex:entryt} Σ up to \tsc{f3} remain unused.


\subsubsection{\tit{Mee} vs. \tit{met}}

Now we come to mee and met. the crucial distinction between these elements is that \tit{mee} is postposition and \tit{met} is a preposition. i show: (i) relation between these types of elements, (ii) how that is being modeled with movement, (iii) that post goes before pre (iv) how that is modeled in nano.

I shows 't spells out all features. this does not always happen. languages also use prepositions/prefixes and postpositions/suffixes to express case. and the relation between them is not arbitrary.

\ex. The preposition/postposition hierarchy
\a. If the expression of a particular case in the Case sequence (below) involves a preposition, then all cases to its right do as well.
\b. The Case sequence: \tsc{nom – acc – gen – dat – ins – com} \hfill \citep{2009}

can be modeled in the sense that a dp moves to a certain height.

\exg. mit ein-em Löffel\\
with a-\tsc{dat.sg} spoon\\
`with a spoon' \hfill (German)

in german, dp moves above the dp. go back to dutch, hard to say above dative, because instrumental sometimes pre sometimes post.

There is variation, between languages but also within languages. another instance is bulgarian.

\ex.
\ag. Tazi duma m-i e nepoznata.\\
that word I-\tsc{dat} is unfamiliar\\
`That word is unfamiliar to me.'
\bg. *Tazi duma e nepoznata sina mi.\\
that word is unfamiliar son my\\
`That word is unfamiliar my son.'
\bg. Tazi duma e nepoznata na sina mi.\\
that word is unfamiliar to son my\\
`That word is unfamiliar to my son.'

Here pronouns can take a postposition, but full DPs cannot. In Dutch it was slightly different: inanimate pronouns can take the postposition, but animates and full DPs cannot.

give lexical entry for that.
say why it does not work with full DPs and with animates.


so what happens with these instead? they get the prepostiion. prepositions have to be encoded differently in the syntax, because they ARE different, they occur before the expression. as you saw the -mee has a unary bottom, so it can only arise as a result of movement. pre elements have a binary bottom.







i'm going to crucially rely on the order in the spellout algoritm




MOVE TO LATER

Spellout happens in a cyclic derivation, following a spellout algorithm \citep{starke2018}. After each instance of merge, spellout takes place. If no spellout exist for the phrase created by the newly added feature, evacuation movements specified in the spellout algorithm take place. The algorithm is given in \ref{ex:spellout}.

\ex. Merge F and \label{ex:spellout}
 \a. Spell out FP
 \b. If (a) fails, attempt movement of the spec of the complement of \tsc{f}, and retry (a)
 \b. If (b) fails, move the complement of \tsc{f}, and retry (a)

When a new match is found, it overrides previous spellouts.

\ex. Cyclic Override \citep{starke2018}:\\
Lexicalisation at a node XP overrides any previous match at a phrase contained in XP.





If not match can be found, a specifier is constructed.

\ex. Spec Formation \citep{starke2018}:\\
If Merge F has failed to spell out (even after backtracking), try to spawn a new derivation providing the feature F and merge that with the current derivation, projecting the feature F at the top node.

For case, I follow \citet{caha2009} that case features case features are organized the containment relation in \ref{ex:casetree}. The higher, more complex cases contain the smaller, less complex cases.



In other words, a lexically stored structure does not have to be identical to the syntactic structure. It is enough if the syntactic structure is contained in the lexically stored tree. This has a as consequence that the lexical entry in \ref{ex:entryt} can also be inserted in syntactic structures as \ref{ex:tacc} and \ref{ex:tnom}.



Nanosyntax distinguishes prepositiones and postpositiones by distinguishing these types of lexical entries. This is the crucial difference needed for the Dutch instrumental.

It differs per language and within a language how high a DP can move. Everything below is a postposition on the noun, everything higher is a preposition. Consider German full DP. Moves as high as above the dative, instrumental is expressed with a preposition.

but like i said, it also differs within a language, for example look at bulgarian here!





The same intuition will hold for Dutch.
-mee is going to be a postposition that will be able to attach to 'r, but it will not to full DPs and animates.








\subsubsection{Backtracking}

If the spellout procedure in \ref{ex:spellout} fails, backtracking takes place.

\ex. Backtracking \citep{starke2018}:\\
When spellout fails, go back to the previous cycle, and try the next option for that cycle.

\begin{table}[ht]
	\center
	\caption {Iron Ossetic pronoun and noun}
	\begin{minipage}{0.56\linewidth}
		\begin{tabularx}{\textwidth}{ccccccc}
		\toprule
              & 1.\tsc{sg}  & head    \\
		\midrule
    \tsc{nom} & æz          & sær-Ø   \\
    \tsc{acc} & mæn-Ø       & sær-Ø   \\
    \tsc{gen} & mæn (⁇?)    & sær-y   \\
    \tsc{ins} & mæn-æj      & sær-æj  \\
    \tsc{dat} & mæn-æn      & sær-æn  \\
    \tsc{all} & mæn-mæ      & sær-mæ  \\
    \tsc{ade} & mæn-yl      & sær-yl  \\
    \tsc{eq}  & mæn-au      & sær-au  \\
    \tsc{com} & mæn-imæ     & sær-imæ \\
    \bottomrule
\end{tabularx}
\label{tbl:ossetic}
\end{minipage}
\end{table}

now the question: is \tit{æz} a suppletive nominative or is \tit{mæn} the suppletive stem from the accusative on? ossetic has a way to tell: suspended affixation.

What we see in (3-a) is an ordinary coordination of two phrases, each marked
by a plural marker and a case marker. Suspended affixation is then illustrated in
(3-b). It is a type of a coordination where the comitative case marker is located
only on the second conjunct, with no change in interpretation. The first conjunct
simply bears no case marker in such cases.


\ex.
\ag. bæx-t-imæ æmæ gæl-t-imæ\\
horse-pl-com and ox-pl-com\\
\bg. bæx-tæ æmæ gæl-t-imæ\\
horse-pl and ox-pl-com\\
`with horses and oxen' \hfill (Iron Ossetic, Erschler 2012: 165)

doing this with 1\tsc{sg} shows that the bare stem is \tit{mæn}.

\ex.
\ag. mæn æmæ Zauyr-æn\\
 1.\tsc{sg} and Zaur-\tsc{dat}\\
\bg. *æz æmæ Zauyr-æn\\
 1.\tbf{sg} and Zaur-\tsc{dat}\\
 `me and Zaur' \hfill (Belyaev 2014: 39)

\tit{mæn} is actually the caseless form, \tit{æz} is marked for nominative.


my conclusion: Dutch has a suppletive form for nominative, accusative and dative. Then higher cases do not combine with that suppletive form, but with the unmarked form.

't
't
't
'r-van
'r-mee
'r-mee



\subsubsection{The lexical entries}



Continuing with \tit{met} `with', which I have shown in this paper can be used to express instrumental. It can also be used as comitative, as shown in \ref{ex:metinscom}.

\exg. Ik dans met hem.\\
 I dance with him\\
 `I am dancing with him.'\label{ex:metinscom}

Case features are in a containment relation and instrumental contains all case features down to \tsc{f1}. Some of these features are expressed by the preposition, and other by the DP they combine with. As pronouns in Dutch can realize up to \tsc{f3}, the preposition starts from \tsc{f4}.

 \ex. \begin{forest} boom
 [\tsc{com}P
     [\tsc{f}5]
     [\tsc{f}4]
 ]
 {\draw (.east) node[right]{⇔ \tit{met}}; }
 \end{forest}\label{ex:entrymet}

In sum, the syntactic structure that \tit{met wat} `with what' realizes is the one in \ref{ex:stmetwat}.

\ex. [ [ [ [ \tsc{thing} ] \tsc{f1} ] \tsc{f2} ] \tsc{f3} \tsc{f4} ]\label{ex:stmetwat}

Let us now turn to \tit{'rmee} `with it', which lexicalizes the same structure as \tit{met 't} `with it', but with a different distribution.

So far, \tit{'r} `there' only realizes the feature \tsc{thing} from the structure in \ref{ex:stmetwat}. This leaves \tsc{f1} to \tsc{f4} to be realized by \tit{mee}. I give the lexical tree in \ref{ex:entrymee}.

\ex. \begin{forest} boom
[\tsc{com}P
    [\tsc{f}5]
    [\tsc{insP}
        [\tsc{f}4]
        [\tsc{dat}P
            [\tsc{f}3]
            [\tsc{acc}P
                [\tsc{f}2]
                [\tsc{nom}P
                    [\tsc{f}1]
                    [ΣP
                        [Σ]
                    ]
                ]
            ]
        ]
    ]
]
{\draw (.east) node[right]{⇔ \tit{mee}}; }
\end{forest}\label{ex:entrymee}

Note that there are two differences between the lexical entry in \ref{ex:entrymet} and \ref{ex:entrymee}. First, the lexical entry for \tit{mee} `with' in \ref{ex:entrymee}, in addition to the one for \tit{met} `with' in \ref{ex:entrymet}, contains the features \tsc{f1}, \tsc{f2} and \tsc{f3}.
Second, the lexical entry in \ref{ex:entrymet} has a binary bottom, and the lexical entry in \ref{ex:entrymee} has a unary bottom. \tit{Met} `with' is a preposition (preposition), so it has a binary bottom, and \tit{mee} `with' is a postposition, so it needs a unary bottom. In the next section I show how this works in a derivation.\footnote{For completeness, \tit{mee} `with' has to be blocked from combining with full DPs and animate pronouns (reference). For now I assume that these pronouns spell out an additional feature \tit{σ} that is not realized by \tit{mee} `with' but only by \tit{met} `with'. Full DPs combine in addition with a zero marker for dative case. This way \tit{mee} `with' does not compete with \tit{met} `anymore'. I left this features out in this paper, at is does not contribute to the main point.}$^,$\footnote{Giving \tit{met} `with' and \tit{mee} `with' two distinct lexical entries has as a consequence that the phonological overlap between them seems like a coincidence. Moreover, in almost all cases the preposition does not change form when it combines with an \tsc{r}-pronoun, e.g. \tit{in}. If this proposal is on the right track, elements as \tit{in} can be used as either a preposition and as a postposition. A lexical entry as in \ref{ex:presuf} would be a candidate for such an element.

\ex. \begin{forest} boom
[XP
    [XP
        [X,roof]
    ]
    [YP
        [Y]
    ]
]
{\draw (.east) node[right]{⇔ \tit{in}}; }
\end{forest}\label{ex:presuf}

I leave it to future research to determine whether this is a feasible solution.}$^,$\footnote{A topic related to this paper is the different positioning of identical adpositions in Dutch (see \citet{caha2010} for an account of German and Dutch and \citet{pretorius2017} for Afrikaans). In \ref{ex:dutchin}, \tit{in} changes meaning dependening on whether it proceeds or follows the DP, it is respectively locational or directional.

\ex.\label{ex:dutchin}
\ag. Ik klim in de boom.\\
 I climb in the tree\\
 `I am climbing in the tree.'
\bg. Ik klim de boom in.\\
 I climb the tree in\\
 `I am climbing into the tree.'

In \ref{ex:dutchin}, the movement of the adposition is driven by movement, and it is meaningful. The movement I discuss in this paper with \tsc{r}-pronouns is driven by spellout, and it is meaningless.}

Table \ref{tbl:withfeatures} gives an overview of the distribution of the features in \tit{met wat} `with what' and \tit{waarmee} `with what'. In both cases, \tit{w} and \tit{a} are the realization of the same morpheme. The difference between the two expressions lies in the distribution of the case features. \tit{Mee} `with' realizes all of them, and \tit{met} `with' only the top two. The residues are realized by respectively \tit{'r} `there' and \tit{'t} `it'.

\begin{table}[ht]
	\center
	\caption {\tit{met wat} and \tit{waarmee}}
	\begin{minipage}{0.56\linewidth}
		\begin{tabularx}{\textwidth}{ccccccc}
		\toprule
    \tsc{wh}  & \tsc{deix}                     & \tsc{f}4  & \tsc{f}3 & \tsc{f}2  & \tsc{f}1  & \tsc{thing} \\
		\midrule
    \tit{w}   & \multicolumn{1}{|c|}{\tit{a}}  & \tit{met} & \multicolumn{4}{|c}{\tit{'t}}                \\\hline
    \tit{w}   & \multicolumn{1}{|c|}{\tit{a}}  & \multicolumn{4}{c|}{\tit{mee}}               & \tit{'r}  \\
    \bottomrule
\end{tabularx}
\label{tbl:withfeatures}
\end{minipage}
\end{table}

In the next section I put all features back together in a derivation and I show how \tit{waarmee} `with what' surfaces when all features form a constituent. \tit{Met wat} `with what' appears when the functional sequence is disrupted.








% I do not care whether \tit{w} and \tit{a} are merged before or after.
%
% WHY DOES MEE MOVE? I DO NOT KNOW.
% BUT IT DOES.
% AND SO DOES WAAR.
% waar maybe because it is remnant movement?
%
%
%
% Either way, it combines with accusative/dative (oblique), which can be seen on the pronouns. But only for the comitative, because for the instrumental we are getting an ʀ-pronoun. Full DPs do not show any marking, leading me to postulate the zero marker up to the dative.
%
% \ex.
% \ag. Ik dans met de man.\\
%  I dance with the man\\
%  `I am dancing with the man.'
% \bg. Ik schilder met een kwast.\\
%  I paint with een brush\\
%  `I am painting with een brush.'
%
% \ex. \begin{forest} boom
% [\tsc{datP}
%    [\tsc{f}3]
%    [\tsc{acc}P
%        [\tsc{f}2]
%        [\tsc{nom}P
%            [\tsc{f}1]
%            [\phantom{x}]
%        ]
%    ]
% ]
% {\draw (.east) node[right]{⇔ \tit{∅}}; }
% \end{forest}»


\section{Showing it in a derivation}

Repeat spellout algorithm

spellout
backtracking
complex spec



I first show how \tit{'rmee} `with it' is constructed. I leave out \tit{w} and \tit{a}, because it unnecessarily complicates the story.\footnote{I assume that the WP and \tsc{deixP} appear lower in the structure than the case features, so the functional sequence is as given in \ref{ex:fseq}.

\ex. [ [ [ [ [ [ [ \tsc{thing} ] \tsc{deix} ] W ] \tsc{f1} ] \tsc{f2} ] \tsc{f3} ] \tsc{f4} ]\label{ex:fseq}
\z.

\phantom{x}

}

We start with \tsc{thing}. The two candidates here are \ref{ex:entryt} and \ref{ex:entryr}. Following the Elsewhere Condition, \ref{ex:entryr} wins the competition because it contains less unused material.

\ex.
\begin{forest} boom
 [\tsc{thingP}
     [\tsc{thing}, roof]
 ]
{\draw (.east) node[right]{⇒ \tit{'r}}; }
\end{forest}\label{ex:thingspellout}

In the next step, \tsc{f}1 is merged. \ref{ex:entryr} is no longer a candidate because it does not contain \tsc{f1}. \ref{ex:entryt} still is a candidate, because it contains all features in \ref{ex:thingf1}. The spellout is overridden and the structure is realized as \tit{'t}.

\ex. \begin{forest} boom
[\tsc{nom}P
   [\tsc{f}1]
   [ΣP
       [Σ]
       [\tsc{thingP}
           [\tsc{thing}, roof]
       ]
   ]
]
{\draw (.east) node[right]{⇒ \tit{'t}}; }
\end{forest}\label{ex:thingf1}

Then \tsc{f}2 is merged. This structure can still be realized by \tit{'t}.

\ex. \begin{forest} boom
[\tsc{accP}
   [\tsc{f}2]
   [\tsc{nom}P
       [\tsc{f}1]
       [ΣP
           [Σ]
           [\tsc{thingP}
               [\tsc{thing}, roof]
           ]
       ]
   ]
]
{\draw (.east) node[right]{⇒ \tit{'t}}; }
\end{forest}

In the next step \tsc{f3} is merged, the structure is still spelled out as \tit{'t}.

\ex. \begin{forest} boom
[\tsc{datP}
    [\tsc{f}3]
    [\tsc{accP}
       [\tsc{f}2]
       [\tsc{nom}P
           [\tsc{f}1]
           [ΣP
               [Σ]
               [\tsc{thingP}
                   [\tsc{thing}, roof]
               ]
           ]
       ]
    ]
]
{\draw (.east) node[right]{⇒ \tit{'t}}; }
\end{forest}

Then \tsc{f4} is merged, as shown in \ref{ex:f4no}. \ref{ex:entryt} can no longer spell the structure out, because it does not contain \tsc{f4}. Moreover, there is no candidate to spell out the structure as it is, so evacuation movement takes place. According to the spellout algorithm, the first evacuation movement to be tried is spec-to-spec movement. However, there is no specifier in this structure, so this does not apply. The second movement option is complement movement. The complement of \tsc{f4} moves to the specifier of \tsc{insP}, resulting in the structure in \ref{ex:f4comp}. The lexicon does not contain an entry with \tsc{insP} which contains only \tsc{f4}.\footnote{\tit{Met} `with' is not a candidate, because the syntactic structure has a unary bottom and the lexical structure has a binary bottom.}

\ex.
\a. \begin{forest} boom
[\tsc{insP}
    [\tsc{f}4]
    [\tsc{datP}
        [\tsc{f}3]
        [\tsc{accP}
           [\tsc{f}2]
           [\tsc{nom}P
               [\tsc{f}1]
               [ΣP
                   [Σ]
                   [\tsc{thingP}
                       [\tsc{thing}, roof]
                   ]
               ]
           ]
        ]
    ]
]
{\draw (.east) node[right]{⇒ }; }
\end{forest}\label{ex:f4no}
\b. \begin{forest} boom
[\tsc{datP}
    [\tsc{datP}
        [\tsc{f}3]
        [\tsc{accP}
           [\tsc{f}2]
           [\tsc{nom}P
               [\tsc{f}1]
               [ΣP
                   [Σ]
                   [\tsc{thingP}
                       [\tsc{thing}, roof]
                   ]
               ]
           ]
        ]
    ]
    {\draw (.east) node[right]{⇒ \tit{'t}}; }
    [\tsc{insP}
        [\tsc{f}4]
    ]
    {\draw (.east) node[right]{⇒ }; }
]
\end{forest}\label{ex:f4comp}

As no match can be found, backtracking is triggered. The derivation goes back to the previous cycle, and the next option for that cycle is tried. In this case, that is after the merge of \tsc{f3}. There is no specifier, so there is no spec-to-spec movement. Complement movement is tried, showed in \ref{ex:f3comp}. However, there is no match in the lexicon for an \tsc{datP} that contains only \tsc{f3}.

\begin{forest} boom
[\tsc{accP}
    [\tsc{accP}
       [\tsc{f}2]
       [\tsc{nom}P
           [\tsc{f}1]
           [ΣP
               [Σ]
               [\tsc{thingP}
                   [\tsc{thing}, roof]
               ]
           ]
       ]
    ]
    {\draw (.east) node[right]{⇒ \tit{'t}}; }
    [\tsc{datP}
        [\tsc{f}3]
    ]
    {\draw (.east) node[right]{⇒ }; }
]
\end{forest}\label{ex:f3comp}

Backtracking proceeds further, into the cycle in which \tsc{f1} was merged. Again there is no specifier to be moved, and complement movement does not give any results.

The derivation ends up backtracking into the cycle in which Σ was merged. At this stage, \tsc{thing} was realized as \tit{'r}. Here there is again no specifier, but complement movement provides a structure that is fit for a lexical entry, namely: \ref{ex:entrymee}.

\ex. \begin{forest} boom
[ΣP
   [\tsc{thingP}
       [\tsc{thing}, roof]
   ]
   {\draw (.east) node[right]{⇒ \tit{'r}}; }
   [ΣP
       [Σ]
   ]
   {\draw (.east) node[right]{⇒ \tit{mee}}; }
]
\end{forest}

The feature \tsc{f}1 is merged again. The structure cannot be spelled out as a whole, so spec-to-spec movement is tried. In that configuration \tsc{f1} can be realized together with Σ as \tit{mee}.

\ex. \begin{forest} boom
[\tsc{nomP}
   [\tsc{thingP}
       [\tsc{thing}, roof]
   ]
   {\draw (.east) node[right]{⇒ \tit{'r}}; }
   [\tsc{nom}P
       [\tsc{f}1]
       [ΣP
           [Σ]
       ]
   ]
   {\draw (.east) node[right]{⇒ \tit{mee}}; }
 ]
\end{forest}

The same happens for \tsc{f2}, \tsc{f3} and \tsc{f4}, and the complete structure is realizes as shown in \ref{ex:spellout'rmee}.

\ex. \begin{forest} boom
[\tsc{insP}
    [\tsc{thingP}
       [\tsc{thing}, roof]
    ]
    {\draw (.east) node[right]{⇒ \tit{'r}}; }
    [\tsc{insP}
       [\tsc{f}4]
       [\tsc{dat}P
           [\tsc{f}3]
           [\tsc{acc}P
               [\tsc{f}2]
               [\tsc{nom}P
                   [\tsc{f}1]
                   [ΣP
                       [Σ]
                   ]
               ]
           ]
       ]
    ]
    {\draw (.east) node[right]{⇒ \tit{mee}}; }
]
\end{forest}\label{ex:spellout'rmee}

Skipping over the irrelevant details of how \tit{w} and \tit{a} end up in their positions, the structure for \tit{waarmee} `with what' look as in \ref{ex:spelloutwaarmee}.\footnote{I assume that WP and \tsc{deixP} are both complex specifiers that are created after \tsc{thing} is spelled out in \ref{ex:thingspellout}. In after each merge after that, backtracking always takes place, the complex specifier is detached from the structure and the case features are spelled out together with or as a postposition on \tsc{thing}.}

\ex. \begin{forest} boom
[\tsc{WP}
    [\tsc{WP}
        [\tsc{W}, roof]
    ]
    {\draw (.east) node[right]{⇒ \tit{w}}; }
    [\tsc{deixP}
        [\tsc{deixP}
            [\tsc{deix}, roof]
        ]
        {\draw (.east) node[right]{⇒ \tit{a}}; }
        [\tit{insP}
            [\tsc{thingP}
               [\tsc{thing}, roof]
            ]
            {\draw (.east) node[right]{⇒ \tit{'r}}; }
            [\tsc{insP}
               [\tsc{f}4]
               [\tsc{dat}P
                   [\tsc{f}3]
                   [\tsc{acc}P
                       [\tsc{f}2]
                       [\tsc{nom}P
                           [\tsc{f}1]
                           [ΣP
                               [Σ]
                           ]
                       ]
                   ]
               ]
            ]
            {\draw (.east) node[right]{⇒ \tit{mee}}; }
        ]
    ]
]
\end{forest}\label{ex:spelloutwaarmee}

A consequence of analyzing \tit{mee} `with' as a postposition is that \tit{r} and \tit{mee} always form a constituent to the exclusion of \tit{w} and \tit{a}. At first sight this seems problematic, because it is possible for \tit{waar} `where', stranding \tit{mee} `with'. I repeat the relevant example in \ref{ex:meestranded}.

\exg. Ik heb gekocht waar jij mee schildert.\\
 I have bought waar you with paint\\
 `I bought what you are painting with.'\label{ex:meestranded}

There is no constituent in \ref{ex:spelloutwaarmee} that contains \tit{waar} but not \tit{mee}. To resolve this situation I follow \citet{noonan2017dutch} in assuming that the phrase containing the adposition (here \tit{mee}) syntactically moves to a position higher in the structure. The movement of the adposition has the typical distribution of that of verbal particles (cf. \citealt{riemsdijk1978,noonan2017dutch}, and it could possibly be triggered by the feature Σ, associated with weak pronouns.
With \tit{mee} having moved out, the WP only contains features that are realized as \tit{waar}, and it moves to the left edge of the clause, resulting in the surface order in \ref{ex:meestranded}.

So far I showed how \tit{waarmee} `with what' is derived if all syntactic features form a constituent. Next I address how this is blocked when the features do not form a proper constituent, and how the spellout is then \tit{met wat} `with what'.

I start in a situation in which \tit{wat} `what' is part of a syntactic structure with the rest of a relative clause as a sister.

\ex. \begin{forest} boom
[CP
    [\tsc{WP}
        [\tsc{WP}
            [\tsc{W}, roof]
        ]
        {\draw (.east) node[right]{⇒ \tit{w}}; }
        [\tsc{deixP}
            [\tsc{deixP}
                [\tsc{deix}, roof]
            ]
            {\draw (.east) node[right]{⇒ \tit{a}}; }
            [\tsc{accP}
               [\tsc{f}2]
               [\tsc{nom}P
                   [\tsc{f}1]
                   [ΣP
                       [Σ]
                       [\tsc{thingP}
                           [\tsc{thing}, roof]
                       ]
                   ]
               ]
            ]
            {\draw (.east) node[right]{⇒ \tit{'t}}; }
        ]
    ]
    [CP
        [...,roof]
    ]
]
\end{forest}

Only after the wh-element is moved to the left edge of the clause, case features higher than the accusative are merged.

When a dative feature is merged here, it will not be able to be spelled out. The only option to realize the features is to form to spawn a new derivation, to create a complex specifier. And that is exactly what forming the preposition \tit{met} `with' is.


\section{Conclusion and discussion}

In this paper, I discussed the distribution of \tit{waarmee} `met wat' and \tit{met wat} `with what' is mismatching free relatives. I showed that \tit{waarmee} `with what' appears when all features form a proper constituent, and \tit{met wat} `with what' if they do not.

The described pattern follows from a core assumptions in Nanosyntax: phrasal spellout spells out constituents. Looking more into detail, \tit{waarmee} `with what' takes precedence over \tit{met wat} `with what' because \tit{mee} has the structure of a postposition. Following the spellout algorithm, \tit{'t} `it' is replaced by \tit{'r} `there', so that it can combine with \tit{mee} `with'. The preposition \tit{met} `with' is only used if there is no other option to spell out the features.

This proposal is in several aspects in accordance with earlier work.
Just like \citet{riemsdijk1978}, I claim that \tsc{r}-pronouns originate as the complement of P. He argued that due to some kind of suppletion \tit{'t} `it' changes form to \tsc{'r}, which is coincidentally also the locative in Dutch. In my proposal, this suppletion follows naturally from the regular spellout algorithm. Also, it is not a coincidence that the locative appears, as it is an item with little features that spells out \tsc{thing}.
The current proposal differs from \citet{riemsdijk1978} in that it is not the whole complement of P is moved. Instead, only part of the complement of P is extracted, which is generally allowed, also in non-preposition stranding languages \citep{abels2003diss}.



\printbibliography

\end{document}
