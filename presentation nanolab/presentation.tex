\documentclass[xcolor=dvipsnames,10pt]{beamer}

\usepackage{../fenna-files/packages}
\usepackage{../fenna-files/commands}
\bibliography{../fenna-files/references}{}

\geometry{paperwidth=140mm,paperheight=105mm}
%
% Choose how your presentation looks.
%
% For more themes, color themes and font themes, see:
% http://deic.uab.es/~iblanes/beamer_gallery/index_by_theme.html
%
\mode<presentation>
{
	\usetheme{Berlin}      % or try Darmstadt, Madrid, Warsaw, ...
	\definecolor{goethe}{rgb}{0,0.37,0.66}
	\usecolortheme[named=goethe]{structure}
	%\usefonttheme{serif}  % or try serif, structurebold, ...
	\setbeamertemplate{navigation symbols}{}
	\setbeamertemplate{caption}[numbered]
	\resetcounteronoverlays{exx}
}



%gets rid of bottom navigation bars
\setbeamertemplate{footline}{}

%gets rid of navigation symbols
%\setbeamertemplate{navigation symbols}{}


\addtobeamertemplate{navigation symbols}{}{%
	\usebeamerfont{footline}%
	\usebeamercolor[fg]{footline}%
	\hspace{1em}%
	\insertframenumber/\inserttotalframenumber
}


\title{The \tsc{r}-pronoun and postposition \tit{waar-mee} in Dutch}
\author{Fenna Bergsma}
\date{Nanolab\\ \today}
\institute{Goethe-Universität Frankfurt}



\begin{document}

\begin{frame}
	\titlepage

\centering{
	\includegraphics[width=0.4\textwidth]{dfg-logo}
	\hspace{2cm}
	\includegraphics[width=0.3\textwidth]{goethe-logo}}
\end{frame}

% \section{Introduction}

\begin{frame}

\exg. Ik schilder met een kwast.\\
 I paint with a brush\\
 `I am painting with a brush.'\label{ex:metdp}

 \pause

\exg. Ik zie 't.\\
 I see it\\
 `I see it.'\label{ex:tverb}

\pause

\exg. *Ik schilder met 't.\\
 I paint with it\\
 `I am painting with it.'

\end{frame}


\begin{frame}

\exg. *Ik schilder met 't.\\
 I paint with it\\
 `I am painting with it.'\label{ex:neemett}

\pause

\exg. Ik schilder r -mee.\\
 I paint there -with\\
 `I am painting with it.'\label{ex:jarmee}

\end{frame}


\begin{frame}

\centering{\Large{Why *\tit{met 't}?}}

\end{frame}


\begin{frame}

Instrumental in Dutch
\ex. *met t

\ex. r mee

\pause

\begin{itemize}
	\item r-pronouns and locatives syncretic \pause
	\item \tit{met} becomes \tit{mee} \pause
	\item position of adposition
\end{itemize}

\end{frame}

% \section{R-pronouns and constituency}

% \subsection{Distribution of \tsc{r}-pronouns}

\begin{frame}

\exg. Ik zie haar/hem.\\
 I see her/him\\
 `I see her/him.'\label{ex:aniobj}

\exg. Ik zie 't.\\
 I see it\\
 `I see it.'\label{ex:inaniobj}

\end{frame}


\begin{frame}

\exg. Ik schilder samen met haar/hem.\\
 I paint together with her/him\\
 `I am painting together with her/him.'\label{ex:prepani}

\exg. *Ik schilder met 't.\\
 I paint with it\\
 `I am painting with it.'\label{ex:prephet}

\exg. Ik schilder 'r -mee.\\
 I paint there -with\\
 `I am painting with it.'\label{ex:preper}

\exg. *Ik schilder mee 'r.\\
 I paint with there\\
 `I am painting with it.'\label{ex:erprep}

\end{frame}


\begin{frame}

\ex.
\ag. Ik zit 'r op.\\
 I sit there on\\
 `I am sitting on it.
\bg. *Ik zit op 't.\\
 I sit on it\\
 `I am sitting on it.

\ex.
 \ag. Hij zwemt 'r in.\\
  he swims it-in\\
  `He is swimming there it.'
 \bg. *Hij zwemt in 't.\\
  he swims in it\\
  `He is swimming in it.'

\end{frame}



\begin{frame}

\exg. Wat zie jij?\\
 what see you\\
 `What do you see?'\label{ex:wat}

\exg. *Met wat schilder jij?\\
 with what paint you\\
 `What are you painting with?'\label{ex:metwat}

\exg. Waar -mee schilder jij?\\
 where with paint you with\\
 `What are you painting with?'\label{ex:waar-mee}

\end{frame}


\begin{frame}

\ex.\label{ex:headed}
\ag. Ik schilder met de kwast waar -mee jij ook schildert.\\
 I paint with the brush where with you also paint\\
 `I am painting with the brush that you are painting with too.'
\bg. *Ik schilder met de kwast met wat jij ook schildert.\\
 I paint with the brush with what you also paint\\
 `I am painting with the brush that you are painting with too.'

\ex.\label{ex:headless}
\ag. Ik schilder waar -mee jij ook schildert.\\
 I paint where with you also paint\\
 `I am painting with what you are painting with too.'
\bg. *Ik schilder met wat jij ook schildert.\\
 I paint with what you also paint\\
 `I am painting with what you are painting with too.'

\end{frame}

\begin{frame}

\ex.
\ag. Ik heb gekocht waar -mee jij schildert.\\
 I have bought where with you paint\\
 `I bought what you are painting with.'\label{ex:gekochtwaar-mee}
\bg. *Ik heb gekocht met wat jij schildert.\\
 I have bought with what you paint\\
 `I bought what you are painting with.'\label{ex:gekochtmetwat}

\pause

\ex.\label{ex:schildermet}
\ag. *Ik schilder waar -mee jij hebt gekocht.\\
 I paint where with you have bought\\
 `I paint with what you bought.'\label{ex:schilderwaar-mee}
\bg. Ik schilder met wat jij hebt gekocht.\\
 I paint with what you have bought\\
 `I paint with what you bought.'\label{ex:schildermetwat}


\end{frame}


\begin{frame}

\ex.
\ag. Ik schilder 'r -mee.\\
 I paint there with\\
 `I am painting with it.'\label{ex:const1}
\bg. Waar -mee schilder jij?\\
where with paint you with\\
 `What are you painting with?'\label{ex:const2}
\bg. Ik schilder met de kwast [waar -mee jij ook schildert].\\
 I paint with the brush where with you also paint\\
 `I am painting with the brush that you are painting with too.'\label{ex:const3}
\bg. Ik schilder [waar -mee jij ook schildert].\\
 I paint where with you also paint\\
 `I am painting with what you are painting with too.'\label{ex:const4}

\pause

\ex.
\a. [[ik] [[schilder] ['r -mee]]]\label{ex:const1stage}
\b. [[jij] [[schilder] [waar-mee]]]\label{ex:const2stage}
\b. [[jij] [[ook] [[schilder] [waar-mee]]]]\label{ex:const3stage}

\end{frame}


\begin{frame}

\ex.
\ag. Ik koop het schilderij.\\
 I buy the painting\\
 `I am buying the painting.'\label{ex:kopen}
\bg. Ik schilder met een kwast.\\
 I paint with a brush\\
 `I am painting with a brush.'\label{ex:schilderen}

 \pause

\end{frame}

\ex.
\ag. Ik koop het schilderij.\\
 I buy the painting\\
 `I am buying the painting.'\label{ex:kopen}
\bg. Ik schilder met een kwast.\\
 I paint with a brush\\
 `I am painting with a brush.'\label{ex:schilderen}

\pause

\exg. Ik heb gekocht [waar -mee jij schildert].\\
 I have bought where with you paint\\
 `I bought what you are painting with.'\label{ex:mismatchwaar-mee}

\pause

\exg. Ik schilder met [wat jij hebt gekocht].\\
I paint with what you have bought\\
`I paint with what you bought.'\label{ex:mismatchmetwat}

\begin{frame}

\exg. [Met [wat [voor [potloden]]] teken jij?\\
 with what for pencils draw you\\
 `What kind of pencils do you with?'\label{ex:watwasfur}
\exg. Ik wil graag thee [met [wat [suiker]]].\\
 I want please tea with some sugar\\
 `I would like to have tea with some sugar.'\label{ex:watindef}

\pause

\ex.\label{ex:summaryconst}
\a. [[met] [wat]] → [waar-mee]\label{ex:waar-meefr}
\b. [met [wat [X]]]\label{ex:metwatx}
\b. [met [[wat] [X]]]\label{ex:metwatfr}

\end{frame}


\begin{frame}

\ex.\label{ex:decompose}
\ag. w -aa -r -mee\\
\tsc{w} -\tsc{deix} -\tsc{loc} -\tsc{ins}\\
\bg. met w -a -t\\
\tsc{ins} \tsc{w} -\tsc{deix} -\tsc{n.sg}\\
\z.
\z.

\end{frame}

\begin{frame}

\ex. \begin{forest}
[\tsc{wP}
    [W, roof]
]
{\draw (.east) node[right]{⇔ \tit{w}}; }
\end{forest}\label{ex:entryw}

\ex. \begin{forest}
[\tsc{deixP}
    [\tsc{deix}, roof]
]
{\draw (.east) node[right]{⇔ \tit{a}}; }
\end{forest}\label{ex:entrya}

\end{frame}


\begin{frame}{Case}

\ex.\label{ex:tsubobj}
\ag. 't Staat in de hal.\\
 \tsc{3sg.n.nom} stands in the hallway\\
 `It is standing in the hallway.'\label{ex:tnoclitic}
\bg. Ik zie 't.\\
 I see \tsc{3sg.n.acc}\\
 `I see it.'
\bg. Ik heb 't een klap gegeven.\\
 I have \tsc{3sg.n.dat} a hit given\\
 `I gave it a hit.'

\ex.\label{ex:hemsubobj}
\ag. Hij staat in de hal.\\
\tsc{3sg.m.nom} stands in the hallway\\
`He is standing in the hallway.'
\bg. Ik zie hem.\\
I see \tsc{3sg.m.acc}\\
`I see it.'
\bg. Ik heb hem een klap gegeven.\\
I have \tsc{3sg.m.dat} a hit given\\
`I gave him a hit.'

\end{frame}




\begin{frame}

\ex.
\ag. *Hij en 't staan in de hoek.\\
 he and it stand in the corner\\
 `He and it are standing in the corner.'\label{ex:tcoordinated}
\bg. Hij en dit/dat staan in de hoek.\\
 he and this/that stand in the corner\\
 `He and it are standing in the corner.'\label{ex:datcoordinated}

\end{frame}


\begin{frame}

\ex. \begin{forest} boom
 [\tsc{datP}
     [\tsc{f3}]
     [\tsc{accP}
         [\tsc{f2}]
         [\tsc{nomP}
             [\tsc{f1}]
             [ΣP
                 [Σ]
                 [\tsc{thingP}
                     [\tsc{thing}, roof]
                 ]
             ]
         ]
     ]
 ]
 {\draw (.east) node[right]{⇔ \tit{'t}}; }
 \end{forest}\label{ex:entryt}

\end{frame}


\begin{frame}

\exg. Ik ben 'r al geweest.\\
 I am there already been\\
 `I have already been there.'

\ex. \begin{forest} boom
[\tsc{loc}P
   [\tsc{location}]
   [\tsc{thingP}
       [\tsc{thing}, roof]
   ]
]
{\draw (.east) node[right]{⇔ \tit{'r}}; }
\end{forest}\label{ex:entryr}

\end{frame}


\begin{frame}

	\begin{tabularx}{\textwidth}{ccccccc}
	\toprule
						& \tsc{n.sg} \\
	\midrule
	\tsc{nom} & 't         \\
	\tsc{acc} & 't         \\
	\tsc{dat} & 't         \\
	\tsc{ins} & -r -mee    \\
	\bottomrule
\end{tabularx}


\end{frame}


\begin{frame}

\ex. \begin{forest} boom
[\tsc{insP}
    [\tsc{f4}]
    [\tsc{datP}
        [\tsc{f3}]
        [\tsc{accP}
            [\tsc{f2}]
            [\tsc{nomP}
                [\tsc{f1}]
                [ΣP
                    [Σ]
                ]
            ]
        ]
    ]
]
{\draw (.east) node[right]{⇔ \tit{-mee}}; }
\end{forest}\label{ex:entrymee}

\end{frame}


\begin{frame}

\ex. \begin{forest} boom
[\tsc{insP}
    [\tsc{f4},roof]
]
{\draw (.east) node[right]{⇔ \tit{met}}; }
\end{forest}\label{ex:entrymet}

\end{frame}


\begin{frame}

\begin{forest} boom
 [\tsc{thingP}
     [\tsc{thing}, roof]
 ]
{\draw (.east) node[right]{⇒ \tit{'r}}; }
\end{forest}\label{ex:thingspellout}

\pause

\begin{forest} boom
[ΣP
   [Σ]
   [\tsc{thingP}
       [\tsc{thing}, roof]
   ]
]
{\draw (.east) node[right]{⇒ \tit{'t}}; }
\end{forest}\label{ex:thingf1}

\end{frame}

\begin{frame}

	\begin{forest} boom
	[\tsc{nomP}
	   [\tsc{f1}]
	   [ΣP
	       [Σ]
	       [\tsc{thingP}
	           [\tsc{thing}, roof]
	       ]
	   ]
	]
	{\draw (.east) node[right]{⇒ \tit{'t}}; }
	\end{forest}

\end{frame}


\begin{frame}

\begin{forest} boom
[\tsc{datP}
    [\tsc{f3}]
    [\tsc{accP}
       [\tsc{f2}]
       [\tsc{nomP}
           [\tsc{f1}]
           [ΣP
               [Σ]
               [\tsc{thingP}
                   [\tsc{thing}, roof]
               ]
           ]
       ]
    ]
]
{\draw (.east) node[right]{⇒ \tit{'t}}; }
\end{forest}

\end{frame}




\begin{frame}

\begin{forest} boom
[\tsc{insP}
    [\tsc{f4}]
    [\tsc{datP}
        [\tsc{f3}]
        [\tsc{accP}
           [\tsc{f2}]
           [\tsc{nomP}
               [\tsc{f1}]
               [ΣP
                   [Σ]
                   [\tsc{thingP}
                       [\tsc{thing}, roof]
                   ]
               ]
           ]
        ]
    ]
]
{\draw (.east) node[right]{⇒ }; }
\end{forest}\label{ex:f4no}

\end{frame}


\begin{frame}

\begin{forest} boom
[ΣP
	 [\tsc{thingP}
			 [\tsc{thing}, roof]
	 ]
	 {\draw (.east) node[right]{⇒ \tit{'r}}; }
	 [ΣP
			 [Σ]
	 ]
	 {\draw (.east) node[right]{⇒ \tit{-mee}}; }
]
\end{forest}

\end{frame}


\begin{frame}

\begin{forest} boom
[\tsc{nomP}
    [\tsc{f1}]
    [ΣP
       [\tsc{thingP}
           [\tsc{thing}, roof]
       ]
       [ΣP
           [Σ]
       ]
    ]
]
{\draw (.east) node[right]{⇒}; }
\end{forest}\label{ex:f1again}

\end{frame}


\begin{frame}

\begin{forest} boom
[\tsc{nomP}
   [\tsc{thingP}
       [\tsc{thing}, roof]
   ]
   {\draw (.east) node[right]{⇒ \tit{'r}}; }
   [\tsc{nomP}
       [\tsc{f1}]
       [ΣP
           [Σ]
       ]
   ]
   {\draw (.east) node[right]{⇒ \tit{-mee}}; }
 ]
\end{forest}\label{ex:f1spec}

\end{frame}


\begin{frame}

\begin{forest} boom
[\tsc{insP}
    [\tsc{thingP}
       [\tsc{thing}, roof]
    ]
    {\draw (.east) node[right]{⇒ \tit{'r}}; }
    [\tsc{insP}
       [\tsc{f4}]
       [\tsc{datP}
           [\tsc{f3}]
           [\tsc{accP}
               [\tsc{f2}]
               [\tsc{nomP}
                   [\tsc{f1}]
                   [ΣP
                       [Σ]
                   ]
               ]
           ]
       ]
    ]
    {\draw (.east) node[right]{⇒ \tit{-mee}}; }
]
\end{forest}\label{ex:spellout'rmee}

\end{frame}


\begin{frame}

\begin{forest} boom
[\tsc{\tsc{wP}}
		[\tsc{\tsc{wP}}
				[\tsc{w}, roof]
		]
		{\draw (.east) node[right]{⇒ \tit{w}}; }
		[\tsc{deixP}
				[\tsc{deixP}
						[\tsc{deix}, roof]
				]
				{\draw (.east) node[right]{⇒ \tit{a}}; }
				[\tsc{insP}
						[\tsc{thingP}
							 [\tsc{thing}, roof]
						]
						{\draw (.east) node[right]{⇒ \tit{'r}}; }
						[\tsc{insP}
							 [\tsc{f4}]
							 [\tsc{datP}
									 [\tsc{f3}]
									 [\tsc{accP}
											 [\tsc{f2}]
											 [\tsc{nomP}
													 [\tsc{f1}]
													 [ΣP
															 [Σ]
													 ]
											 ]
									 ]
							 ]
						]
						{\draw (.east) node[right]{⇒ \tit{-mee}}; }
				]
		]
]
\end{forest}\label{ex:spelloutwaar-mee}

\end{frame}




\begin{frame}

\exg. Ik heb gekocht waar jij mee schildert.\\
 I have bought waar you with paint\\
 `I bought what you are painting with.'\label{ex:meestranded}

\exg. *Ik heb gekocht waar jij schildert mee.\\
 I have bought waar you paint with\\
 `I bought what you are painting with.'\label{ex:meeend}

\end{frame}


\begin{frame}

\ex.
\a. \tit{met} --- \tit{mee}
\b. \tit{tot} --- \tit{toe}

\ex. CVt
\pause

\ex.
\a. \tit{in} --- \tit{in}
\b. \tit{van} --- \tit{van}
\b. \tit{op} --- \tit{op}

\end{frame}




\appendix

\begin{frame}{References}

	\scriptsize{
		\printbibliography
	}

\end{frame}



\end{document}
