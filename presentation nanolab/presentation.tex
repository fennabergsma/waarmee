\documentclass[xcolor=dvipsnames,10pt]{beamer}

\usepackage{../fenna-files/packages}
\usepackage{../fenna-files/commands}
\bibliography{../fenna-files/references}{}

\usepackage{multicol}

\geometry{paperwidth=140mm,paperheight=105mm}
%
% Choose how your presentation looks.
%
% For more themes, color themes and font themes, see:
% http://deic.uab.es/~iblanes/beamer_gallery/index_by_theme.html
%
\mode<presentation>
{
\usetheme{Berlin}      % or try Darmstadt, Madrid, Warsaw, ...
\definecolor{goethe}{rgb}{0,0.37,0.66}
\usecolortheme[named=goethe]{structure}
	% \usefonttheme{serif}  % or try serif, structurebold, ...
\setbeamertemplate{navigation symbols}{}
\setbeamertemplate{caption}[numbered]
\resetcounteronoverlays{exx}
}

\renewcommand{\eachwordone}{\sffamily}
\renewcommand{\eachwordtwo}{\sffamily}
\renewcommand{\eachwordthree}{\sffamily}

%gets rid of bottom navigation bars
\setbeamertemplate{footline}{}

%gets rid of navigation symbols
%\setbeamertemplate{navigation symbols}{}


\addtobeamertemplate{navigation symbols}{}{%
\usebeamerfont{footline}%
\usebeamercolor[fg]{footline}%
\hspace{1em}%
\insertframenumber/\inserttotalframenumber
}


\title{The \tsc{r}-pronoun and postposition \tit{waar-mee} in Dutch}
\author{Fenna Bergsma}
\date{Syntax colloquium\\ \today}
\institute{Goethe-Universität Frankfurt}



\begin{document}

\begin{frame}
\titlepage

\centering{
\includegraphics[width=0.4\textwidth]{dfg-logo}
\hspace{2cm}
\includegraphics[width=0.3\textwidth]{goethe-logo}}
\end{frame}

% \section{Introduction}

\begin{frame}

\exg. Jij schildert \tbf{met} \tbf{een} \tbf{kwast}.\\
 you paint with a brush\\
 `You are painting with a brush.'\label{ex:metwat}

\pause

\exg. \tbf{Wat} zie jij?\\
 what see you\\
 `What do you see?'\label{ex:wat}

\pause

\exg. *\tbf{Met} \tbf{wat} schilder jij?\\
 with what paint you\\
 `What are you painting with?'\label{ex:metwat}

\pause

\exg. \tbf{Waarmee} schilder jij?\\
 {where with} paint you with\\
 `What are you painting with?'\label{ex:waar-mee}

\end{frame}


\begin{frame}

\exg. Womit malst du?\\
 {where with} paint you\\
 `What are you painting with?'

\pause

\exg. Waa\tbf{r}mee schilder jij?\\
 {where with} paint you with\\
 `What are you painting with?'\label{ex:waar-mee}

\end{frame}




\begin{frame}

\exg. Worauf kletterst du?\\
 {where on} climb you\\
 `What are you climbing onto?'

\exg. Wo bist du \tbf{(d)r}auf geklettert?\\
 where are you on climbed\\
 `What did you climb on?'

\pause

\exg. Waa\tbf{r} ben jij op geklommen?\\
 where are you on climbed\\
 `What did you climb on?'


\end{frame}


\begin{frame}

\exg. *Ik schilder \tbf{met} \tbf{t}.\\
 I paint with it\\
 `I am painting with it.'

\pause

\exg. Ik schilder \tbf{r} \tbf{-mee}.\\
 I paint \tsc{r} -with\\
 `I am painting with it.'\label{ex:jarmee}

\end{frame}


\begin{frame}

\centering{\Large{Why *\tit{met wat}?}}

\vspace{3em} \citep{riemsdijk1978}


\end{frame}


\begin{frame}

\centering{
*\tit{met wat} \hspace{1em} | \hspace{1em} \tit{waar-mee}}

\vspace{1em}

\pause

\begin{itemize}
\item positions are reversed \pause
\item \tit{met} becomes \tit{mee} \pause
\item \tit{wat} becomes \tit{waar}, which is syncretic with locative \citep{koopman1994}
\end{itemize}

\vspace{1em}\pause

\tsc{r}-pronouns are the result of regular spellout mechanisms

\pause

as long as all features form a proper constituent

\end{frame}


\begin{frame}

\ex.\label{ex:headed}
\ag. Ik koop de kwast \tbf{waar} \tbf{-mee} jij schildert.\\
 I buy the brush where with you paint\\
 `I buy the brush that you are painting with.'
\bg. *Ik koop de kwast \tbf{met} \tbf{wat} jij schildert.\\
 I koop the brush with what you paint\\
 `I buy the brush that you are painting with.'

\end{frame}


\begin{frame}

\ex.
\ag. \tbf{Waar} \tbf{-mee} schilder jij?\\
 where with paint you with\\
 `What are you painting with?'\label{ex:const2}
\bg. Ik koop de kwast [\tbf{waar} \tbf{-mee} jij schildert].\\
 I buy the brush where with you paint\\
 `I koop the brush that you are painting with.'\label{ex:const3}

\pause

\ex. [[jij] [[schilder] [\tbf{waar-mee}]]]\label{ex:const2stage}


\end{frame}



\begin{frame}

\ex.
\ag. Ik heb gekocht \tbf{waar} \tbf{-mee} jij schildert.\\
 I have bought where with you paint\\
 `I bought what you are painting with.'\label{ex:gekochtwaar-mee}
\bg. *Ik heb gekocht \tbf{met} \tbf{wat} jij schildert.\\
 I have bought with what you paint\\
 `I bought what you are painting with.'\label{ex:gekochtmetwat}

\pause

\ex.\label{ex:schildermet}
\ag. *Ik schilder \tbf{waar} \tbf{-mee} jij hebt gekocht.\\
 I paint where with you have bought\\
 `I paint with what you bought.'\label{ex:schilderwaar-mee}
\bg. Ik schilder \tbf{met} \tbf{wat} jij hebt gekocht.\\
 I paint with what you have bought\\
 `I paint with what you bought.'\label{ex:schildermetwat}


\end{frame}


\begin{frame}

\ex.
\ag. Ik koop \tbf{het} \tbf{schilderij}.\\
 I buy the painting\\
 `I am buying the painting.'\label{ex:kopen}
\bg. Ik schilder \tbf{met} \tbf{een} \tbf{kwast}.\\
 I paint with a brush\\
 `I am painting with a brush.'\label{ex:schilderen}

 \pause

\exg. Ik heb gekocht [\tbf{waar} \tbf{-mee} jij schildert].\\
 I have bought where with you paint\\
 `I bought what you are painting with.'\label{ex:mismatchwaar-mee}

 \pause

\exg. Ik schilder \tbf{met} [\tbf{wat} jij hebt gekocht].\\
 I paint with what you have bought\\
 `I paint with what you bought.'\label{ex:mismatchmetwat}

\end{frame}



\begin{frame}

\exg. [\tbf{Met} [\tbf{wat} [voor [potloden]]] teken jij?\\
 with what for pencils draw you\\
 `What kind of pencils are you drawing with?'\label{ex:watwasfur}

\pause

\exg. Ik wil graag thee [\tbf{met} [\tbf{wat} [suiker]]].\\
 I want please tea with some sugar\\
 `I would like to have tea with some sugar.'\label{ex:watindef}

\pause

\ex.\label{ex:summaryconst}
\a. [[met] [wat]] → [waar-mee]\label{ex:waar-meefr}
\b. [met [wat [X]]]\label{ex:metwatx}
\b. [met [[wat] [X]]]\label{ex:metwatfr}

\end{frame}


\begin{frame}

\ex. w -aa \tbf{-r} \tbf{-mee}

\ex. \tbf{met} w -a \tbf{-t}

\pause

\begin{itemize}
\item late insertion (phonology after syntax)\pause
\item individual features (no feature bundles)
\end{itemize}


\end{frame}

\begin{frame}

\ex. \begin{forest}
[\tsc{wP}
    [W, roof]
]
{\draw (.east) node[right]{⇔ \tit{w}}; }
\end{forest}\label{ex:entryw}

\citep{hachem2015}

\pause


\ex. \begin{forest}
[\tsc{deixP}
    [\tsc{deix}, roof]
]
{\draw (.east) node[right]{⇔ \tit{a}}; }
\end{forest}\label{ex:entrya}

\citep[cf.][]{lander2016}

\end{frame}


\begin{frame}

\ex.\label{ex:tsubobj}
\ag. T staat in de hal.\\
 \tsc{3sg.n.nom} stands in the hallway\\
 `It is standing in the hallway.'\label{ex:tnoclitic}
\bg. Ik zie t.\\
 I see \tsc{3sg.n.acc}\\
 `I see it.'
\bg. Ik heb t een klap gegeven.\\
 I have \tsc{3sg.n.dat} a hit given\\
 `I gave it a hit.'

\end{frame}

\begin{frame}

\label{ex:casetree}
\ex. \begin{forest} boom
	[\tsc{insP}
			[\tsc{f4}]
			[\tsc{datP}
					[\tsc{f3}]
					[\tsc{accP}
							[\tsc{f2}]
							[\tsc{nomP}
									[\tsc{f1}]
									[X
											[..,roof]
									]
							]
					]
			]
	]
\end{forest}

\citep{caha2009}

\end{frame}



\begin{frame}

\ex.
\ag. *Hij en t staan in de hoek.\\
 he and it stand in the corner\\
 `He and it are standing in the corner.'\label{ex:tcoordinated}
\bg. Hij en dit/dat staan in de hoek.\\
 he and this/that stand in the corner\\
 `He and it are standing in the corner.'\label{ex:datcoordinated}

\pause

\exg. T staat in de hal.\\
 \tsc{3sg.n.nom} stands in the hallway\\
 `It is standing in the hallway.'\label{ex:tnoclitic}

\end{frame}



\begin{frame}

\label{ex:strengthtree}
\ex. \begin{forest} boom
			[CP
					[C]
					[ΣP
							[Σ]
							[IP
									[I]
									[X
											[..,roof]
									]
							]
					]
			]
\end{forest}

\citep{cardinaletti1996}

\end{frame}


\begin{frame}

\ex. \begin{forest} boom
 [\tsc{datP}
     [\tsc{f3}]
     [\tsc{accP}
         [\tsc{f2}]
         [\tsc{nomP}
             [\tsc{f1}]
             [ΣP
                 [Σ]
								 [IP
								 		[I]
		                 [\tsc{thingP}
		                     [\tsc{thing}, roof]
										]
                 ]
             ]
         ]
     ]
 ]
 {\draw (.east) node[right]{⇔ \tit{t}}; }
 \end{forest}\label{ex:entryt}


\citep{cardinaletti1996,kayne2005,caha2009}

\end{frame}


\begin{frame}

\ex. \tbf{The Superset Principle} \citep{starke2009}: \\
	A lexically stored tree matches a syntactic node iff the lexically stored tree contains the syntactic node.

\begin{multicols}{2}

\ex. \begin{forest} boom
	[\tsc{accP}
			[\tsc{f2}]
			[\tsc{nomP}
					[\tsc{f1}]
					[ΣP
							[Σ]
							[IP
									[I]
									[\tsc{thingP}
											[\tsc{thing}, roof]
									]
							]
					]
			]
	]
\end{forest}\label{ex:tacc}

\ex. \begin{forest} boom
	[\tsc{nomP}
			[\tsc{f1}]
			[ΣP
					[Σ]
					[IP
							[I]
							[\tsc{thingP}
									[\tsc{thing}, roof]
							]
					]
			]
	]
\end{forest}\label{ex:tnom}

\end{multicols}

\end{frame}




\begin{frame}

\exg. Ik ben r al geweest.\\
 I am there already been\\
 `I have already been there.'

\pause

\ex. \begin{forest} boom
[\tsc{loc}P
   [\tsc{location}]
   [\tsc{thingP}
       [\tsc{thing}, roof]
   ]
]
{\draw (.east) node[right]{⇔ \tit{r}}; }
\end{forest}\label{ex:entryr}

\citep{baunaz2018}

\pause

\ex. \tbf{The Elsewhere Condition} (\citealt{kiparsky1973}, formulated as in \citealt{caha2020}):\\
When two entries can spell out a given node, the more specific entry wins. Under the Superset Principle governed insertion, the more specific entry is the one which has fewer unused features.

\pause

\ex. \begin{forest} boom
 [\tsc{thingP}
     [\tsc{thing}, roof]
 ]
{\draw (.east) node[right]{⇒ \tit{r}}; }
\end{forest}

\end{frame}


\begin{frame}

\centering{
\begin{tabular}{cc}
\toprule
						& \tsc{n.sg} \\
\midrule
\tsc{nom} & t         \\
\tsc{acc} & t         \\
\tsc{dat} & t         \\
\tsc{ins} & r -mee    \\
\bottomrule
\end{tabular}
}

\end{frame}


\begin{frame}

\ex. \begin{forest} boom
[\tsc{insP}
    [\tsc{f4}]
    [\tsc{datP}
        [\tsc{f3}]
        [\tsc{accP}
            [\tsc{f2}]
            [\tsc{nomP}
                [\tsc{f1}]
                [ΣP
                    [Σ]
										[IP
												[I]
										]
                ]
            ]
        ]
    ]
]
{\draw (.east) node[right]{⇔ \tit{-mee}}; }
\end{forest}\label{ex:entrymee}

\pause


\ex. \begin{forest} boom
[\tsc{insP}
    [\tsc{f4},roof]
]
{\draw (.east) node[right]{⇔ \tit{met}}; }
\end{forest}\label{ex:entrymet}

\end{frame}


\begin{frame}

[W [\tsc{deix} [\tsc{f4} [\tsc{f3} [\tsc{f2} [\tsc{f1} [Σ [I [\tsc{thing}] ] ] ] ] ] ] ] ]

\end{frame}


\begin{frame}

\ex. Merge F and \label{ex:spellout}
	 \a. Spell out FP.
	 \b. If (a) fails, attempt movement of the spec of the complement of \tsc{f}, and retry (a).
	 \b. If (b) fails, move the complement of \tsc{f}, and retry (a).

\pause

\ex. \tbf{Cyclic Override} \citep{starke2018}:\\
	Lexicalisation at a node XP overrides any previous match at a phrase contained in XP.

\pause

\ex. \tbf{Backtracking} \citep{starke2018}:\\
	When spellout fails, go back to the previous cycle, and try the next option for that cycle.\label{ex:backtracking}

\pause

\ex. \tbf{Spec Formation} \citep{starke2018}:\\
	If Merge F has failed to spell out (even after backtracking), try to spawn a new derivation providing the feature F and merge that with the current derivation, projecting the feature F at the top node.\label{ex:specformation}

\end{frame}


\begin{frame}

\ex. \begin{forest} boom
 [\tsc{thingP}
     [\tsc{thing}, roof]
 ]
{\draw (.east) node[right]{⇒ \tit{r}}; }
\end{forest}\label{ex:thingspellout}

\pause

\ex. \begin{forest} boom
[IP
   [I]
   [\tsc{thingP}
       [\tsc{thing}, roof]
   ]
]
{\draw (.east) node[right]{⇒ \tit{t}}; }
\end{forest}\label{ex:thingf1}

\end{frame}


\begin{frame}

\ex. \begin{forest} boom
	   [ΣP
	       [Σ]
				 [IP
				 		[I]
			       [\tsc{thingP}
			           [\tsc{thing}, roof]
			       ]
					]
	   ]
	{\draw (.east) node[right]{⇒ \tit{t}}; }
\end{forest}

\end{frame}


\begin{frame}

\ex. \begin{forest} boom
	[\tsc{nomP}
	   [\tsc{f1}]
	   [ΣP
	       [Σ]
				 [IP
				 		[I]
			       [\tsc{thingP}
			           [\tsc{thing}, roof]
			       ]
					]
	   ]
	]
	{\draw (.east) node[right]{⇒ \tit{t}}; }
\end{forest}

\end{frame}


\begin{frame}

\ex. \begin{forest} boom
[\tsc{datP}
    [\tsc{f3}]
    [\tsc{accP}
       [\tsc{f2}]
       [\tsc{nomP}
           [\tsc{f1}]
           [ΣP
               [Σ]
							 [IP
							 		[I]
		               [\tsc{thingP}
		                   [\tsc{thing}, roof]
		               ]
							 ]
           ]
       ]
    ]
]
{\draw (.east) node[right]{⇒ \tit{t}}; }
\end{forest}

\end{frame}




\begin{frame}

\ex. \begin{forest} boom
[\tsc{insP}
    [\tsc{f4}]
    [\tsc{datP}
        [\tsc{f3}]
        [\tsc{accP}
           [\tsc{f2}]
           [\tsc{nomP}
               [\tsc{f1}]
               [ΣP
                   [Σ]
									 [IP
									 		[I]
		                   [\tsc{thingP}
		                       [\tsc{thing}, roof]
		                   ]
										]
               ]
           ]
        ]
    ]
]
{\draw (.east) node[right]{⇒ }; }
\end{forest}\label{ex:f4no}

\end{frame}


\begin{frame}

\ex. \begin{forest} boom
	[\tsc{datP}
	    [\tsc{datP}
	        [\tsc{f3}]
	        [\tsc{accP}
	           [\tsc{f2}]
	           [\tsc{nomP}
	               [\tsc{f1}]
	               [ΣP
	                   [Σ]
										 [IP
										 		[I]
			                   [\tsc{thingP}
			                       [\tsc{thing}, roof]
												 ]
	                   ]
	               ]
	           ]
	        ]
	    ]
	    {\draw (.east) node[right]{⇒ \tit{'t}}; }
	    [\tsc{insP}
	        [\tsc{f4}]
	    ]
	    {\draw (.east) node[right]{⇒ }; }
	]
\end{forest}

\end{frame}


\begin{frame}

\ex. \begin{forest} boom
[IP
	 [\tsc{thingP}
			 [\tsc{thing}, roof]
	 ]
	 {\draw (.east) node[right]{⇒ \tit{r}}; }
	 [IP
			 [I]
	 ]
	 {\draw (.east) node[right]{⇒ \tit{-mee}}; }
]
\end{forest}

\pause

\citep{abels2003diss}

\end{frame}


\begin{frame}

\ex. \begin{forest} boom
[ΣP
    [Σ]
    [IP
       [\tsc{thingP}
           [\tsc{thing}, roof]
       ]
       [IP
           [I]
       ]
    ]
]
{\draw (.east) node[right]{⇒}; }
\end{forest}\label{ex:f1again}

\end{frame}


\begin{frame}

\ex. \begin{forest} boom
[ΣP
   [\tsc{thingP}
       [\tsc{thing}, roof]
   ]
   {\draw (.east) node[right]{⇒ \tit{r}}; }
   [ΣP
       [Σ]
       [IP
           [I]
       ]
   ]
   {\draw (.east) node[right]{⇒ \tit{-mee}}; }
 ]
\end{forest}\label{ex:f1spec}

\end{frame}


\begin{frame}

\ex. \begin{forest} boom
[\tsc{insP}
    [\tsc{thingP}
       [\tsc{thing}, roof]
    ]
    {\draw (.east) node[right]{⇒ \tit{r}}; }
    [\tsc{insP}
       [\tsc{f4}]
       [\tsc{datP}
           [\tsc{f3}]
           [\tsc{accP}
               [\tsc{f2}]
               [\tsc{nomP}
                   [\tsc{f1}]
									 [ΣP
									 		[Σ]
		                   [IP
		                       [I]
											]
                   ]
               ]
           ]
       ]
    ]
    {\draw (.east) node[right]{⇒ \tit{-mee}}; }
]
\end{forest}\label{ex:spelloutrmee}

\end{frame}


\begin{frame}

\ex. \begin{forest} boom
[\tsc{\tsc{wP}}
		[\tsc{\tsc{wP}}
				[\tsc{w}, roof]
		]
		{\draw (.east) node[right]{⇒ \tit{w}}; }
		[\tsc{deixP}
				[\tsc{deixP}
						[\tsc{deix}, roof]
				]
				{\draw (.east) node[right]{⇒ \tit{a}}; }
				[\tsc{insP}
						[\tsc{thingP}
							 [\tsc{thing}, roof]
						]
						{\draw (.east) node[right]{⇒ \tit{r}}; }
						[\tsc{insP}
							 [\tsc{f4}]
							 [\tsc{datP}
									 [\tsc{f3}]
									 [\tsc{accP}
											 [\tsc{f2}]
											 [\tsc{nomP}
													 [\tsc{f1}]
													 [ΣP
													 		 [Σ]
															 [IP
																	 [I]
															 ]
													 ]
											 ]
									 ]
							 ]
						]
						{\draw (.east) node[right]{⇒ \tit{-mee}}; }
				]
		]
]
\end{forest}\label{ex:spelloutwaar-mee}

\end{frame}



\begin{frame}

\ex. \begin{forest} boom
[CP
    [\tsc{\tsc{wP}}
        [\tsc{\tsc{wP}}
            [\tsc{w}, roof]
        ]
        {\draw (.east) node[right]{⇒ \tit{w}}; }
        [\tsc{deixP}
            [\tsc{deixP}
                [\tsc{deix}, roof]
            ]
            {\draw (.east) node[right]{⇒ \tit{a}}; }
            [\tsc{datP}
                [\tsc{f3}]
                [\tsc{accP}
                   [\tsc{f2}]
                   [\tsc{nomP}
                       [\tsc{f1}]
                       [ΣP
                           [Σ]
													 [IP
													 		[I]
		                           [\tsc{thingP}
		                               [\tsc{thing}, roof]
		                           ]
														]
                       ]
                   ]
                ]
            ]
            {\draw (.east) node[right]{⇒ \tit{t}}; }
        ]
    ]
    [CP
        [..,roof]
    ]
]
\end{forest}

\end{frame}


\begin{frame}

\begin{itemize}
	\item \tsc{r}-pronouns are nothing special, but just a result of regular spellout \pause
	\item \tit{r} is ambiguous between location and (caseless) thing \pause
	\item \tit{met} and \tit{mee} are separate lexical entries
		\begin{itemize}
			\item so their phonological realization can differ
			\item so they can differ between between being pre and post
		\end{itemize}
\end{itemize}

\end{frame}


\begin{frame}

\exg. Ik zie \tbf{t}.\\
 I see it\\
 `I see it.'\label{ex:inaniobj}

\exg. *Ik schilder \tbf{met} \tbf{t}.\\
 I paint with it\\
 `I am painting with it.'\label{ex:prephet}

\exg. Ik schilder \tbf{r} \tbf{-mee}.\\
 I paint where -with\\
 `I am painting with it.'\label{ex:preper}

\pause

\exg. Ik zie \tbf{haar/hem}.\\
 I see her/him\\
 `I see her/him.'\label{ex:aniobj}

\exg. Ik schilder samen \tbf{met} \tbf{haar/hem}.\\
I paint together with her/him\\
`I am painting together with her/him.'\label{ex:prepani}

\pause

\ex. [\tsc{f4} [\tsc{f3} [\tsc{f2} [\tsc{f1} [\tbf{phi} [Σ [I [\tsc{thing}] ] ] ] ] ] ] ]

\end{frame}



\begin{frame}

\exg. Ik heb gekocht waar mee jij schildert.\\
 I have bought where with you paint\\
 `I bought what you are painting with.'\label{ex:meestranded}

\pause

\exg. Ik heb gekocht waar jij mee schildert.\\
 I have bought where you paint\\
 `I bought what you are painting with.'\label{ex:meestranded}

\pause

\ex. \begin{forest} boom
[\tsc{insP}
    [\tsc{thingP}
       [\tsc{thing}, roof]
    ]
    {\draw (.east) node[right]{⇒ \tit{r}}; }
    [\tsc{\tbf{insP}}
       [\tsc{f4}]
       [\tsc{datP}
           [\tsc{f3}]
           [\tsc{accP}
               [\tsc{f2}]
               [\tsc{nomP}
                   [\tsc{f1}]
									 [ΣP
									 		[Σ]
			                   [IP
			                       [I]
			                   ]
			               ]
								]
           ]
       ]
    ]
    {\draw (.east) node[right]{⇒ \tit{-mee}}; }
]
\end{forest}

\citep{noonan2017dutch}

\end{frame}


\begin{frame}

\begin{tabular}{ l | l | l }
\textsc{prep} \textit{what} & \textit{where} \tsc{post} & translation \\
\hline
 aan wat & waar\textbf{aan} & `on whom/what' \\
 achter wat & waar\textbf{achter} & `behind whom/what' \\
 bij wat & waar\textbf{bij} & `at whom/what' \\
 door wat & waar\textbf{door} & `by whom/what' \\
 langs wat & waar\textbf{langs} & `alongside whom/what \\
 met wat & waar\underline{\textbf{mee}} & `with whom/what' \\
 naar wat & waar\textbf{naar} & `to whom/what' \\
 op wat & waar\textbf{op} & `on whom/what' \\
 over wat & waar\textbf{over} & `about whom/what' \\
 tot wat & waar\underline{\textbf{toe}} & `to whom/what' \\
 uit wat & waar\textbf{uit} & `out of whom/what' \\
 van wat & waar\textbf{van} & `of whom/what' \\
 voor wat & waar\textbf{voor} & `for whom/what' \\
\end{tabular}

\end{frame}


\begin{frame}

\ex.\label{ex:dutchin}
\ag. Ik dans in het bos.\\
 I dance in the forest\\
 `I am dancing in the forest.'
\bg. Ik dans het bos in.\\
 I dance the forest in\\
 `I am dancing into the forest.'

\pause

\exg. Ik dans r -in.\\
 I dance \tsc{r} in\\
 `I dance it in.'\\
  *`I dance into it.'

\pause

\exg. *Ik ben r -in gedanst.\\
 I am \tsc{r} in danced\\
 `I dancing into it.'

\pause

\exg. Ik heb r -in gedanst.\\
 I have \tsc{r} in danced\\
 `I dancing in it.'


\end{frame}

\begin{frame}

\centering{
	\begin{tabular}{ccc}
	\toprule
					& DP				& \tsc{r}	\\
	\midrule
		pre		& \tsc{loc} & *     	\\
		post	& \tsc{dir}	& \tsc{loc} \\
	\bottomrule
\end{tabular}
		}

\vspace{1em}
\pause

verbal particles in e.g. \tbf{mee}lopen `walk along', \tbf{mee}zingen `sing along'

\vspace{1em}
	\pause

\centering{
	\begin{tabular}{ccc}
	\toprule
					& \tit{mee}	& \tit{met}	\\
	\midrule
		pre		& verb		 	& noun     	\\
		post	& noun			& *					 \\
	\bottomrule
	\end{tabular}
	}


\end{frame}


\end{document}
